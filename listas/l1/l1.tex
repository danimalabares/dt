\input{/Users/daniel/github/config/preamble-por.sty}
%\input{/Users/daniel/github/config/thms-por.sty}

\begin{document}
\bibliographystyle{alpha}

\begin{minipage}{\textwidth}
	\begin{minipage}{1\textwidth}
		Topologia Diferencial\hfill Daniel González Casanova Azuela
		
		{Prof. Vinicius Ramos\hfill\href{https://github.com/danimalabares/dt}{github.com/danimalabares/dt}}
	\end{minipage}
\end{minipage}\vspace{.2cm}\hrule

\vspace{10pt}
{\huge Lista 1}
\begin{thing1}{Problem 1}\label{p:1}\leavevmode
Let \(G(k,n)\) be the set of dimension \(k\) vector subspaces of \(\mathbb{R}^n\). Construct a smooth structure on \(G(k,n)\) and compute its dimension.
\end{thing1}

\begin{proof}[Solution]\leavevmode
	(Proof in \cite{gui2}.) First we topologize \(G(n,k)\) as follows. We identify it with the quotient \(W/ \sim\), the set of \(k\)-frames (sets of \(k\) linearly independent vectors of \(\mathbb{R}^n\)) modulo the equivalence relation of spanning the same vector space. Since \(W \subset (\mathbb{R}^n)^k\) it has a subspace topology, and \(G(n,k)\) has a quotient topology.

	Now fix \(V \in G(n,k)\). To construct a chart consider the set
	\[W_V=\{U \in G(n,k):\text{orthogonal projection \(U \to V\) is bijective} \}\]
and the function
\begin{align*}
	\rho_V: W_V &\longrightarrow \operatorname{Hom}(V,V^\perp) \\
	U &\longmapsto \pi_{U,V^\perp}\circ\pi^{-1}_{U,V}
\end{align*}
where \(\pi_{X,Y}\) is the orthogonal projection from \(X\) to \(Y\). Since the set \(\operatorname{Hom}(V,V^\perp)\) is the spaces of matrices of  \(\dim V \times \dim V^\perp=k (n-k)\), we may write \(\operatorname{Hom}(V,V^\perp)=\mathbb{R}^{k(n-k)}\).

To complete the proof we must confirm several facts: (1) \(G(n,k)\) is Hausdorff and second countable, (2) \(W_V\) is open for all  \(V\), (3) \(\rho_V\) is a homeomorphism for all \(V\), (4) transition maps \(\rho_{V}\circ \rho_{V'}^{-1}\) are smooth.
	
\begin{enumerate}
\item By thm 7.7 in \cite{tus}, it's enough to show that the quotient projection \(q:W \to W/\sim\) is an open map and that the graph of \(\sim\) is closed.
\item To see that \(W_V\) is open define \(\widetilde{W}_V\) to be the set of \(k\)-frames  \(\{u_i\}_{i=1}^k\) of \(\mathbb{R}^n\) such that the orthogonal projection from \(\operatorname{span}(u_i)\) onto \(V\) is bijective. Then \(\widetilde{W}_V/\sim=W_V\). Since \(G(n,k)=W/\sim\) is equipped with the quotient topology, it's enough to show that \(q^{-1}\left(q \left(\widetilde{W}_V\right) \right)=\widetilde{W}_V\) is open, where \(q:W \to W/\sim\) is the quotient map.

	Fix \(\{u_i\}_{i=1}^n \in \widetilde{W}_V\). It is clear from elementary properties of euclidean space that there is a neighbourhood of every \(u_i\) such that the vector space obtained by choosing one vector in each of these neighbourhoods orthogonally-projects bijectively onto \(V\). Since we are using the product topology on \(W\), it follows that \(\widetilde{W}_V\) is open.
%	the set of orthogonal projections \(\{\pi(u_i)\}_{i=1}^k\) onto \(V\) generate \(V\).
\item %To see that \(\rho_V\) is injective suppose that \(U\) and \(U'\) are such that \(\rho_V(U)=\rho_V(U')\), that is, \(\pi_{U,V^\perp} \circ \pi^{-1}_{U,V}=\pi_{U',V^\perp} \circ \pi^{-1}_{U',V}\). Now given a base \(\{v_i\}\) of \(V\) we obtain bases \(\{u_i\}\) of  \(U\) and \(\{u_i'\}\) of \(U'\) (because the projections \(\pi_{U,V},\pi_{U',V}\) are isomorphisms). The projections of these bases onto \(V^\perp\) coincide by hypothesis, i.e. \(\pi_{U,V^\perp}(u_i)=\pi_{U',V^\perp}(u_i')\). Suppose \(u \in U\setminus U'\). Projecting \(u=\sum \lambda^i u_i\) to \(V^\perp\) gives a linear combination of \(\pi_{U,V^\perp}(u_i)=\pi_{U',V^\perp}(u_i')\), yielding an element of \(U'\).

	%First we check surjectivity. Fix \(T: V \to V^\perp\) linear and bases \(\{v_i\}_{i=1}^k\) of \(V\) and \(\{v_j'\}_{j=1}^{n-k}\) of \(V^\perp\). Since \(\pi_{U,V}\) is a bijection for any \(U \in W_V\), t. Any vector space \(U\) projecting

	%Any linear map from \(V \cong \mathbb{R}^k\) to \(V^\perp \cong \mathbb{R}^{n-k}\)
\item 
\end{enumerate}



	%First consider the set of size \(k\) orthonormal frames (=linearly independent sets) of \(R^n\). It is a subset of \(S^{n-1}\times\ldots\times S^{n-1}\), so that it has a subspace topology. {\color{2}But how to give it a smooth structure?}
\end{proof}

\begin{thing1}{Problem 2}\label{p:2}\leavevmode
Let \(M\) and $N$ be manifolds of dimension $m$ and $n$, respectively, and let \(f:M \to N\) be a smooth function whose rank is \(k\) for every point in an open set \(\tilde{U} \subset M\). Prove that for each point \(p \in \tilde{U}\), there exist charts \((U, \phi)\) and \((V,\psi)\) centered at \(p\) and \(f(p)\) such that \(f(U) \subset V\) and
\[\psi \circ f \circ \phi^{-1}(x_1,\ldots,x_k,x_{x+1},\ldots,x_m)=(x_1,\ldots,x_k,0,\ldots,0).\]
\end{thing1}

\begin{proof}[Solution]\leavevmode
	(Adapted from the proof of thm B.4 in \cite{tus}.) Since \(D_{p}f\) has rank \(k\) at \(p\), we may assume the first \(k\) columns are linearly independent. Define a locally invertible map of \(\mathbb{R}^m\) to itself by
\[G(x_1,\ldots,x_k,y_1,\ldots,y_{m-k})=\Big(f_1(x,y),\ldots,f_k(x,y),y\Big).\]
where \(f_i\) are the coordinate functions of $f$ for some charts of \(p\) and \(f(p)\). \(G\) is locally invertible since it has nonsingular derivative at \(p\). Notice that \(f \circ G^{-1}\) maps 
\[(x,y) \mapsto \Big(x,f_{k+1} \circ G^{-1}(x,y),\ldots,f_{n} \circ G^{-1}(x,y)\Big).\]
{\color{6}Notice that \(f \circ G^{-1}\) does not depend on \(y\) in a neighbourhood of \(p\):} since \(G\) is locally a diffeomorphism, the rank of \(f \circ G^{-1}\) must be the same as that of \(f\), and its derivative is
\[D_q(f \circ G^{-1})=\begin{pmatrix}\operatorname{Id}&0\\ \frac{\partial (f \circ G^{-1})_i}{\partial x^j} &\frac{\partial (f \circ G^{-1})_i}{\partial y^j}\end{pmatrix},\quad\text{ for }  k \leq i \leq n\]
so that the matrix \(\frac{\partial (f \circ G^{-1})_i}{\partial y^j}\) must be singular \textbf{for all $q$ in a neighbourhood of $p$} (here we use that $f$ has \textbf{constant} rank \(k\)). This allows us to define the function of \(R^n\) to itself
\[F(x,y)=\Big(x,y_1-f_{k+1}\circ G^{-1}(x),\ldots,y_n-f_n \circ G^{-1}(x) \Big) \]
which is locally invertible: its derivative is
\[D_{f(p)}F(x,y)=\begin{pmatrix}\operatorname{Id}&0\\ *&\operatorname{Id}\end{pmatrix}\]
using that \(f_i \circ G^{-1}\) does not depend on \(y\) near $p$. Thus we may restrict our domains as necessary to obtain open sets  \(U \ni p\) and \(V \ni f(p)\) such that
\[F \circ \hat{f} \circ G^{-1}(x,y)=F\Big(x,f_{k+1} \circ G^{-1}(x,y),\ldots,f_{n} \circ G^{-1}(x,y)\Big)=(x,0)\]
where \(\hat{f}=(f_1,\ldots,f_n)\) is the coordinate representation of $f$ with which we started.

\end{proof}
\iffalse
\begin{proof}[Ideas using \cite{lee}]\leavevmode
	A linear map with kernel collapses to zero a piece of its domain. Thus the image of the map may be smaller than the domain. Then the rank says how big is the image, how small is the kernel.

	We can chose coordinates such that a linear map of rank \(k\) is seen as the identity in the first \(k\times k\) block and 0 elsewhere: these are given by (1) a basis of the domain composed by a basis of the kernel and the preimage of a basis of the image, and (2) the same basis of the image completed to any basis of the codomain.

	This exercise is the non-flat analogoue for the latter linear case: show that a smooth map of rank \(k\) has coordinate representation \((x_1,\ldots,x_m) \mapsto (x_1,\ldots,x_k,0,\ldots,0)\).

	Proceeding in analogy, we want to find (1) a chart of the domain composed by a chart of the zero-set of \(f \circ \psi\) and the preimage of a chart of the image, and (2) a chart for the codomain which is just the completion of a chart for the image.

	Now the image of $f$ may not be a manifold (for this we need $f$ to be an embedding). But instead $f$ has rank \(k\). If we are allowed to use the level-set theorem, we know at once that \(f^{-1}(c)\) is a \(m-k\)-dimensional submanifold of \(M\) for any \(c \in f(M)\). This gives us a chart for \(f^{-1}(c)\).

	


\end{proof}\fi

\clearpage\begin{thing1}{Problem 3}\label{p:3}\leavevmode
Let \(M\) be a compact manifold. Prove that does not exist a submersion \(F: M \to \mathbb{R}^k\), \(k>0\).
\end{thing1}

\begin{proof}[Solution]\leavevmode
%{\color{6}How to prove that a submersion is open?}
	Since \(M\) is compact any real-valued function is bounded, so the composition of \(F\) with the modulus function \(\|\cdot\| \circ F\) is bounded and so is \(F\). Now let \(x_0 \in M\) a point such that \(\|F(x)\|\) is maximum over \(M\).

Let \(\gamma\) be the curve in \(\mathbb{R}^k\) given by \(\gamma(t)=F(x_0)+tF(x_0)\). It corresponds to a vector at \(F(x_0)\) pointing in the direction of \(F(x_0)\), so that the norm of points on \(\gamma\) for positive \(t\) is larger than that of \(F(x_0)\)

Since \(D_{x_0}F\) is surjective, there is a vector such that its image under \(D_{x_0}F\) is \([\gamma]\). Namely, \([F^{-1}\circ \gamma]\) for some local inverse of \(F\). (Indeed: \(F_*[F^{-1}\circ \gamma]=\frac{d}{dt}\Big|_{t=0}F \circ F^{-1} \circ \gamma\).)

Now let \(x_1:=F^{-1} \circ \gamma(t_1)\) for some \(t_1>0\). Then \(F(x_1)\) has a bigger norm than \(F(x_0)\):
\begin{align*}
\|F(x_1)\|&=\|\gamma(t_1)\|=\|F(x_0)+t_1F(x_0)\|>\|F(x_0)\|
\end{align*}
but \(\|F(x_0)\|\) is maximum.
\end{proof}


\bibliography{bib.bib}

\end{document}
