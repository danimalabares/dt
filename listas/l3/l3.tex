\input{/Users/daniel/github/config/preamble-por.sty}
%\input{/Users/daniel/github/config/thms-por.sty}

\begin{document}
\bibliographystyle{alpha}

\begin{minipage}{\textwidth}
	\begin{minipage}{1\textwidth}
		Topologia Diferencial\hfill Daniel González Casanova Azuela
		
		{Prof. Vinicius Ramos\hfill\href{https://github.com/danimalabares/dt}{github.com/danimalabares/dt}}
	\end{minipage}
\end{minipage}\vspace{.2cm}\hrule

\vspace{10pt}
{\huge Lista 3}

\vspace{1em}
\begin{thing1}{Problema 1}\label{prob:1}\leavevmode
Seja \(f: X \to Y\) un difeomorfismo entre duas variedades orientadas conexas. Prove que \(df_x\) preserva orientação para um ponto \(x \in X\) se, e somente se, \(df_x\) preserva orientação para todo ponto \(x \in X\).
\end{thing1}

\begin{proof}\leavevmode
Pelo teorema da função inversa orientado,  existe uma vizinhança \(U\) de  \(x\) na qual \(\det d_zf\) é positivo para todo \(z \in U\).

Como \(X\) é conexa, podemos ligar \(x\) com qualquer outro ponto \(y \in X\) mediante uma curva \(\gamma\). Pegue em cada ponto \(\gamma(t)\) uma vizinhança na qual \(\det d_{z}f\) é positivo em qualquer ponto \(z\) da vizinhança. Como \(\operatorname{Im}\gamma\) é compacto, temos uma quantidade finita de abertos onde \(\det df\) é positivo.
\end{proof}

\end{document}
