\input{/Users/daniel/github/config/preamble-por.sty}
%\input{/Users/daniel/github/config/thms-por.sty}

\begin{document}
\bibliographystyle{alpha}

\begin{minipage}{\textwidth}
	\begin{minipage}{1\textwidth}
		Topologia Diferencial\hfill Daniel González Casanova Azuela
		
		{Prof. Vinicius Ramos\hfill\href{https://github.com/danimalabares/dt}{github.com/danimalabares/dt}}
	\end{minipage}
\end{minipage}\vspace{.2cm}\hrule

\vspace{10pt}
{\huge Lista 3}

\vspace{1em}
\begin{thing1}{Problema 1}\label{prob:1}\leavevmode
Seja \(f: X \to Y\) un difeomorfismo entre duas variedades orientadas conexas. Prove que \(df_x\) preserva orientação para um ponto \(x \in X\) se, e somente se, \(df_x\) preserva orientação para todo ponto \(x \in X\).
\end{thing1}

\begin{proof}\leavevmode
Considere
\begin{align*}
	D: X &\longrightarrow \mathbb{R} \\
	x &\longmapsto \det d_xf
\end{align*}
É uma funcão contínua que nunca pode ser zero. Como é positiva em \(x\), deve ser positiva sempre.
\iffalse
	Pelo teorema da função inversa orientado,  existe uma vizinhança \(U\) de  \(x\) na qual \(\det d_zf\) é positivo para todo \(z \in U\).

Como \(X\) é conexa, podemos ligar \(x\) com qualquer outro ponto \(y \in X\) mediante uma curva \(\gamma\). Pegue em cada ponto \(\gamma(t)\) uma vizinhança na qual \(\det d_{z}f\) é positivo em qualquer ponto \(z\) da vizinhança. Como \(\operatorname{Im}\gamma\) é compacto, temos uma quantidade finita de abertos onde \(\det df\) é positivo.\fi
\end{proof}

\begin{thing1}{Problem 2}\label{prob:2}\leavevmode
Seja \(X\) uma variedade orientável. Prove que a orientação induzida em \(X \times X\) é independente da orientação de \(X\).
\end{thing1}

\begin{proof}\leavevmode
	A orientação de \(X \times X\) está dada como segue: uma base \((\beta_1,\beta_2)\) do espaço tangente \(T_{(x,y)} X \times X\) é orientada se \(\beta_1\) e \(\beta_2\) são bases orientadas de \(X\).

Agora considere a mesma construção usando \(-X\). A base \((\tilde{\beta_1},\tilde{\beta_2})\) de \(-X \times -X\) é orientada se \(\tilde{\beta_1}\) e \(\tilde{\beta_2}\) são bases orientadas de \(-X\).

Porém, é equivalente que \((\beta_1,\beta_2)\) seja orientada em \(X\times X\) e que \((\tilde{\beta_1},\tilde{\beta_2})\) seja orientada em \(-X \times -X\): tanto a transformação que manda \(\beta_1 \mapsto \tilde{\beta_1}\) quanto a transformação que manda \(\beta_2 \mapsto \tilde{\beta_2}\) tem determinante negativo, de modo que a transformação que manda \((\beta_1,\beta_2)\mapsto (\tilde{\beta_1},\tilde{\beta_2})\) tem determinante positivo!
\end{proof}

\begin{thing1}{Problem 3}\label{prob:3}\leavevmode
Prove que \(\mathsf{SO}(n)\) é uma variedade orientável e calcule a sua dimensão. Usando teoria da interseção prove que \(\chi(\mathsf{SO}(n))=0\).
\end{thing1}

\begin{proof}\leavevmode
	First notice that \(\mathsf{SO}(n)\) is one of the connected components of \(\mathsf{O}(n)\). Indeed, \(\mathsf{SO}(n)=\det^{-1}(1)\) for the submersion \(\det:\mathsf{O}(n)\to \{\pm 1\}\), making into a codimension-0 submanifold of  \(\mathsf{O}(n)\) since \(\dim \{\pm 1\}=0\). This means that computing the dimension of \(\mathsf{SO}(n)\) is the same as computing the dimension of \(\mathsf{O}(n)\).

	Now observe that a matrix in \(\mathsf{O}(n)\) is the same as an orthonormal frame of \(\mathbb{R}^n\): the column vectors of any  \(A \in \mathsf{O}(n)\) unitary and mutually orthogonal since \(A A ^{\mathbf{T}}=\operatorname{Id}\) says \(\sum_{k}a_{ik}a_{jk}=\delta_{ij}\) for every \(i,j\). 

	We can compute the dimension of \(\mathsf{O}(n)\) as follows. Take a vector \(v_1\) in \(S^{n-1}\), then a unitary vector in the orthogonal complement of  \(v_1\), i.e. a vector in \(S^{n-2}\), and so on until we choose either of the two vectors in \(S^0\). This means that we are choosing points in \(S^{n-1}\times S^{n-2} \times \ldots \times S^0\), which gives \(\dim \mathsf{O}(n)=\sum_{i=0}^{n-1}i\). Gauss could tell at very early age that this number is \(\frac{n(n-1)}{2}\).

	To orient \(\mathsf{SO}(n)\) just notice that it acts on itself homogeneously (by orientation-preserving diffeomorphisms). Taking a basis at the identity matrix and moving it around our manifold using this action generates a smooth global choice of local orientations; i.e. a global orientation.

	The fact that \(\chi(\mathsf{SO}(n))=0\) is immediate from the fact that its tangent bundle is trivial: there is a nowhere vanishing vector field (the orbit of any nonzero vector), giving the result by Poincaré-Hopf theorem.
\iffalse
	\(\mathsf{SL}(n)\) is \(\det^{-1}(1)\) for \(\det:\mathsf{GL}(n)\to \mathbb{R}\). And \(\mathsf{O}(n)\) is \(F^{-1}(\operatorname{Id})\) for \(F: \mathsf{GL}(n)\to \mathsf{GL}(n)\), \(A \mapsto  A A^{\mathbf{T}}\). And then \(\mathsf{SO}(n)=\mathsf{SL}(n)\cap \mathsf{O}(n)\).

To compute the dimension of \(\mathsf{SO}(n)\) consider the embedding \(\mathsf{O}(n)\hookrightarrow \mathsf{GL}(n)\). If \(\mathsf{O}(n)\pitchfork \mathsf{SL}(n)\) we get that \(\mathsf{SO}(n)\) has the same codimension inside \(\mathsf{O}(n)\) as the codimension of \(\mathsf{SL}(n)\) inside \(\mathsf{GL}(n)\), the latter being 1 since it is a level-set of a real valued function. {\color{2}But what is the dimension of \(\mathsf{O}(n)\)?}

To give \(\mathsf{SO}(n)\) an orientation we can use the preimage orientation. That is\fi
\end{proof}

\begin{thing1}{Problem 4}\label{prob:4}\leavevmode
Seja \(\Sigma\) uma superfície de gênero \(g\). Construa um campo vetorial em \(\Sigma\) com um único zero de índice \(2-2g\).
\end{thing1}

\begin{proof}[Solution]\leavevmode

\end{proof}

\begin{thing1}{Problem 5}\label{prob:5}\leavevmode
Seja \(A\) uma matriz de \(n \times n\) com coeficientes inteiros e seja \(f:\mathbb{R}^2/\mathbb{Z}^2 \to \mathbb{R}^2/\mathbb{Z}^2\) tal que \(f(x)=Ax\). Calcule o grau de \(f\).
\end{thing1}

\begin{proof}\leavevmode
	To compute the degree of \(f \) it's enough to compute the numer of preimages of \([0] \in \mathbb{R}^n/\mathbb{Z}^n\). This is the same as counting the number of points with integer coordinates in \(P:=A\Big( [0,1)^n \Big)\).

	In fact, this number is the volume of \(P\). To check this we may






\iffalse
	First consider the case for \(n=1\). Take the class \([0] \in \mathbb{R}/\mathbb{Z}\) and let's check how many preimages it has in the fundamental domain \([0,1) \subset \mathbb{R}\). Our matrix \(A\) is only a number, say \(a\). So we have \(ax \in [0]=\{0+n:n \in \mathbb{Z}\}\). This just says \(ax \in \mathbb{Z}\) which happens when \(x\) is a rational number with  denominator \(a\), and there's  \(|a|\) such numbers in \([0,1)\).

	Now for the case  \(n=2\) suppose \(A=\begin{pmatrix} a & b\\c & d \end{pmatrix} \), and let's look for points \(\vec{x}=(x,y) \in [0,1)^2\) such that \(A\vec{x} \in [\vec{0}]=\mathbb{Z}^2\). This means that
	\[A \vec{x}=\begin{pmatrix} a & b\\c & d \end{pmatrix} \begin{pmatrix} x\\y \end{pmatrix} =\begin{pmatrix} ax+by\\cx+dy \end{pmatrix} \in \mathbb{Z}^2.\]
	 Substracting these two conditions (that \(ax+by \in \mathbb{Z}\) and that \(cx+dy \in \mathbb{Z}\)) we obtain that
	 \[x(a-c) + y(b-d) \in \mathbb{Z}\]
	Suppose for a second that \(y=0\): we go back to a case similar to the 1-dimensional, namely there are \(|a-c|\) solutions. The same works whenever \(y(b-d) \in \mathbb{Z}\), for which there are \(|b-d|\) choices. So there's \(|a-c||b-d|\) solutions.
\fi
\end{proof}


\begin{thing1}{Problem 6}\label{prob:6}\leavevmode
Prove que \(\mathbb{R}P^{2n+1}\) é orientável e que \(\mathbb{R}P^{2n}\) não é orientável.
\end{thing1}

\begin{proof}\leavevmode
	First notice that \(-\operatorname{Id}\) preserves orientation iff \(n\) is odd. This map is a composition of \(n\) reflections, one about every axis of \(\mathbb{R}^{n+1}\supset S^n\). Each of these reflections is orientation-reversing, and composing a map with an orientation-reversing map reverses orientation by the chain rule.

	Now recall that \(\mathbb{R}P^{n}=S^{n}/-\operatorname{Id}\). Suppose \(\mathbb{R}P^{n}\) is orientable, so that the quotient map is orientation-preserving since it is a submersion: the determinant of its differential is a nowhere-zero continuous function on a connected manifold, so it cannot be positive somewhere and negative elsewhere.

	Choose an oriented basis of the tangent space of \(\mathbb{R}P^{n}\) at \([e_1]\). Pull back the basis using quotient map, this produces a basis at each of the preimages, namely \(e_1\) and \(-e_1\). These two bases must be in the same orientation of \(S^{n}\) since the quotient map is orientation-preserving and they are mapped to the same basis in the quotient. However, this only happens when \(-\operatorname{Id}\) is orientation-preserving.
\end{proof}



\end{document}
