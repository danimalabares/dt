\input{/Users/daniel/github/config/preamble-por.sty}
%\input{/Users/daniel/github/config/thms-por.sty}

\begin{document}
\bibliographystyle{alpha}

\begin{minipage}{\textwidth}
	\begin{minipage}{1\textwidth}
		Topologia Diferencial\hfill Daniel González Casanova Azuela
		
		{Prof. Vinicius Ramos\hfill\href{https://github.com/danimalabares/dt}{github.com/danimalabares/dt}}
	\end{minipage}
\end{minipage}\vspace{.2cm}\hrule

\vspace{10pt}
{\huge Lista 4}

\vspace{1em}

\begin{thing1}{Problem 1}\label{prob:1}\leavevmode
Para \(k=\mathbb{R}\) ou \(\mathbb{C}\), seja \(f: kP^n \longrightarrow \mathbb{R}\) definida por
\[f\Big([z_0:\ldots:z_n]\Big)=\frac{\sum_{j=0}^nj|z_j|^2}{\sum_{j=0}^n |z_j|^2}.\]
Prove que \(f\) é uma função de Morse e calcule o índice de todos os pontos críticos de \(f\) para  \(k=\mathbb{R}\) e \(k=\mathbb{C}\).
\end{thing1}

	 \iffalse Primeiro considere a carta coordenada definida em \(U_0=\{[1:x_1:\ldots:x_n]\}\) como \(\varphi_0([1:x_1:\ldots:x_n])=(x_1,\ldots,x_n)\). (Supondo que \(f\) está bem definida, coisa que não comprovei.) Nessas coordenadas a nossa função fica
	\[f \circ \varphi_0^{-1}(x_1,\ldots,x_n)=\frac{x_1^2+2x_2^2+\ldots+nx_n^2}{1+x_1^2+\ldots+x_n^2}\]
E a derivada (o gradiente) dela acaba sendo dado por
\begin{align*}\frac{\partial }{\partial x_i}(f \circ \varphi_0^{-1})&=\frac{\Big(1+\sum_{j=1}^n x_j^2\Big)(2ix_i)-\Big(\sum_{j=1}^njx_j^2\Big)(2x_i)}{\left(1+\sum_{j=1}^nx_j^2\right)^2 }\\
	&=\frac{2x_i\left(i \left(1+\sum_{j=1}^nx_j^2\right)-\sum_{j=1}^njx_j^2\right) }{\left(1+\sum_{j=1}^nx_j^2\right) ^2}
	\end{align*}
	Podemos ver que quando \(x_i=0\) para toda \(i=1,\ldots,n\), i.e. no ponto \([1:0:\ldots:0]\), o gradiente se anula, i.e. esse é um ponto crítico.

\begin{remark}\leavevmode
Fico meio travado para comprovar se poderia ter outros pontos críticos, i.e. se o numerador da eq. anteior poderia ser zero em algum outro ponto. Vou continuar analizando o ponto crítico que já encontrei.
\end{remark}

	Para analizar a Hessiana nessa carta coordenada calculei que
	\[\frac{\partial^2}{\partial x_i\partial x_j}(f \circ \varphi_0^{-1})\Big|_{(1,0,\ldots,0)}=2i\delta_{ij}\]
Rascunho da conta: o \(\delta_{ij}\) aparece quando derivamos \(2x_i\) no numerador respeito a \(x_j\). Isso é o único que sobrevive quando avaliamos em \((1,0,\ldots,0)\)! (E o denominador fica igual a 1.) Obtemos que a Hessiana não tem autovalores negativos neste ponto crítico, e concluimos que ele tem índice 0.

	\begin{align*}
	\frac{\partial^2}{\partial x_i\partial x_j}(f \circ \varphi^{-1})&=\frac{\left(1+\sum_{k=1}^nx_k^2\right) 2 \delta_{ij}-2x_i 2\delta_{ij}}{\left(1+\sum_{k=1}^nx_k^2\right)^4}\\
	&=\frac{2\delta_{ij}}{\left(1+\sum_{k=1}^n x_k^2\right)^4}\left(\left(1+\sum_{k=1}^nx_k^2\right) -2x_i\right) 
	\end{align*}
	avaliando no ponto crítico \((0,\ldots,0)\) obtemos \(D^2_{\vec{0}}(f\circ \varphi^{-1})=2\operatorname{Id}\). Ou seja, o índice de \([1:0:\ldots:0]\) é \(0\) (a Hessiana não tem autovalores negativos nesse ponto).\fi


\begin{proof}[Solution]\leavevmode
	\textbf{(Caso \(k=\mathbb{R}\).)} Fixe \(r \in \{0,1,2,\ldots,n\}\). Consideremos a carta coordenada \(\varphi_r([x_0:\ldots:x_n])	= (x_0,\ldots,\widehat{x_r},\ldots,x_n)\) definida em \(U_r=\{x_r \neq 0\}=\{x_r=1\}\). Nossa função fica
	\[f \circ \varphi_r^{-1}(x_0,\ldots,\widehat{x_r},\ldots,x_n)=\frac{r+\sum_{i\neq r}ix_i^2}{1+\sum_{i\neq r}x_i^2}\]
Para facilitar notação defina \(\vec{x}:=(x_0,\ldots,x_{r-1},1,x_{r+1},\ldots,x_n)\). A nossa função se escreve como
\[(f \circ \varphi_r^{-1})(x_0,\ldots,\widehat{x_r},\ldots,x_n)=\frac{r+\sum_{i\neq r}ix_i^2}{\|\vec{x}\|^2}\]
Derivemos:
\begin{align*}
\frac{\partial }{\partial x_i}(f \circ \varphi_r^{-1})&=\frac{\|\vec{x}\|^2\cdot 2ix_i+\frac{\partial \|\vec{x}\|^2}{\partial x_i}\left(r+\sum_{j \neq r}ix_i^2\right) }{\|\vec{x}\|^4}\\
&=2x_i\cdot  \frac{i \|\vec{x}\|^2+\left(r+\sum_{j \neq r}jx_j^2\right) }{\|\vec{x}\|^4}
\end{align*}
É claro que se \(x_i=0\) para toda \(i\) temos um ponto crítico, i.e. em \(\vec{0}\in \mathbb{R}^n\)%\(p:=(0,\ldots,\underbrace{1}_{r\text{-th place} },\ldots,0)\)
. (Note que não podemos ter outros pontos críticos, pois se \(x_k \neq 0\) para alguma \(k\), a derivada parcial nessa variável não pode se anular: o numerador do quociente é uma soma de números positivos!)


Derivando de novo e avaliando em \(\vec{0}\):
\[\frac{\partial ^2}{\partial x_i\partial x_j}(f \circ \varphi_r^{-1})\Big|_{\vec{0}}=2\delta_{ij}(i+r),\qquad r \neq  i,j\]
Isso fica pela regra do produto: obtemos a derivada de \(2x_i\) multiplicada pelo quociente, somado com \(2x_i\) multiplcado pela derivada do quociente. É só notar que, por um lado, a derivada de \(2x_i\) respeito de \(x_j\) é \(2\delta_{ij}\) e o lado direito avaliado em \(\vec{0}\) da \(i\). Por outro lado, a derivada do quociente em \(\vec{0}\) se anula.

Então a matriz de segundas derivadas fica
\[2\begin{pmatrix} r & 0 &0 & \cdots &0\\
0 & 1+r & 0& \cdots & 0\\
0 & 0 & 2+r & \cdots & 0\\
\vdots & & &\ddots &\vdots \\
0 & 0 & 0 & \cdots & n+r\end{pmatrix}\]
que é não singular e não tem autovalores negativos. Concluimos que os pontos \([0:\ldots:\underbrace{1}_{r\text{-th place} }:\ldots:0]\) são críticos de índice 0 para cada \(r=0,\ldots,n\).

\textbf{(Caso \(k=\mathbb{C}\).)} Fixe \(r \in \{0,1,2,\ldots,n\}\). Consideremos a carta coordenada \(\varphi_r([z_0:\ldots:z_n])	= (z_0,\ldots,\widehat{z_r},\ldots,z_n)\) definida em \(U_r=\{z_r \neq 0\}=\{z_r=1\}\). Nossa função fica
	\[f \circ \varphi_r^{-1}(z_0,\ldots,\widehat{z_r},\ldots,z_n)=\frac{r+\sum_{i\neq r}iz_i\overline{z_i}}{1+\sum_{i\neq r}z_i\overline{z_i}}\]
Para facilitar notação defina \(\vec{z}:=(z_0,\ldots,z_{r-1},1,z_{r+1},\ldots,z_n)\). A nossa função se escreve como
\[(f \circ \varphi_r^{-1})(z_0,\ldots,\widehat{z_r},\ldots,z_n)=\frac{r+\sum_{i\neq r}iz_i\overline{z_i}}{\|\vec{z}\|^2}\]
Derivemos:
\begin{align*}
\frac{\partial }{\partial z_i}(f \circ \varphi_r^{-1})&=\frac{\|\vec{z}\|^2\cdot 2i\overline{z_i}+\frac{\partial \|\vec{z}\|^2}{\partial z_i}\left(r+\sum_{j \neq r}iz_i\overline{z_i}\right) }{\|\vec{z}\|^4}\\
&=2\overline{z_i}\cdot  \frac{i \|\vec{z}\|^2+\left(r+\sum_{j \neq r}jz_j\overline{z_j}\right) }{\|\vec{z}\|^4}
\end{align*}
E analogamente
\[\frac{\partial }{\partial \overline{z}_i}(f \circ \varphi_r^{-1})=2z_i\cdot  \frac{i \|\vec{z}\|^2+\left(r+\sum_{j \neq r}jz_j\overline{z_j}\right) }{\|\vec{z}\|^4}
\]
Como antes, a derivada só pode ser zero quando \(z_i=0 \iff \overline{z_i}=0\) \(\forall i\), i.e. no ponto \(\vec{0}\in \mathbb{C}^n\).

Para calcular a Hessiana note que as parciais cruzadas \(\frac{\partial^2 }{\partial z_i\partial \overline{z}_j}\) são as únicas que podem sobrevivir! De fato, quando derivamos de novo y avaliamos em \(\vec{0}\),
\[\frac{\partial^2}{\partial z_i \partial z_j }(f \circ \varphi_r^{-1})\Big|_{\vec{0}}=\frac{\partial^2 }{\partial \bar{z}_i\partial \bar{z}_j}(f \circ \varphi_r^{-1})\Big|_{\vec{0}}=0\]
e
\[\frac{\partial^2 }{\partial z_i\partial \bar{z}_j}(f \circ \varphi_r^{-1})\Big|_{\vec{0}}=\frac{\partial^2 }{\partial \bar{z}_i\partial z_j}(f \circ \varphi_r^{-1})\Big|_{\vec{0}}=2\delta_{ij}(i+r), \qquad  r\neq i,j\]
Essa matriz é
\[2\left(\begin{array}{@{}c|c@{}}
0&\begin{matrix} r & 0 &0 & \cdots &0\\
0 & 1+r & 0& \cdots & 0\\
0 & 0 & 2+r & \cdots & 0\\
\vdots & & &\ddots &\vdots \\
0 & 0 & 0 & \cdots & n+r\end{matrix}\\
\hline
\begin{matrix} r & 0 &0 & \cdots &0\\
0 & 1+r & 0& \cdots & 0\\
0 & 0 & 2+r & \cdots & 0\\
\vdots & & &\ddots &\vdots \\
0 & 0 & 0 & \cdots & n+r\end{matrix}&0\end{array}\right)\]
que é não singular e não tem autovalores negativos, concluindo, como no caso real, que os pontos \([0:\ldots:\underbrace{1}_{r\text{-th place} }:\ldots:0]\) são críticos não degenerados de índice 0.

\end{proof}
\end{document}
