\input{/Users/daniel/github/config/preamble-por.sty}
%\input{/Users/daniel/github/config/thms-por.sty}

\begin{document}
\bibliographystyle{alpha}

\begin{minipage}{\textwidth}
	\begin{minipage}{1\textwidth}
		Topologia Diferencial\hfill Daniel González Casanova Azuela
		
		{Prof. Vinicius Ramos\hfill\href{https://github.com/danimalabares/dt}{github.com/danimalabares/dt}}
	\end{minipage}
\end{minipage}\vspace{.2cm}\hrule

\vspace{10pt}
{\huge Lista 1}

\vspace{1em}
\iffalse
\begin{defn}[\(C^k\) topology]\leavevmode
It's a topology on \(C^\infty (X,Y)\) generated by the subsets
\[M_k(U):=\left\{ f \in C^\infty(X,Y) : j^kf \in U  \right\}\]
For any open subset \(U \subset J^k(X,Y)\).

\textbf{Idea.} Put a topology on an infinite-dimensional space coming from finite-dimensional approximations. Infinite-dimensional is a whole Taylor series, a smooth function, and finite dimensional is a finite polynomial, a jet!

But there's something missing! The point \(x\). Because a jet is not a global linear approximation right? So the real definition is
\[M_k(U)=\{f \in C^\infty (X,Y): j^kf(x) \in U \forall x \in X\}\]
But how is the topology on \(J^k(X,Y)\)? Or better yet, what is \(J^k(X,Y)\)? Points are functions from \(X\) to \(Y\) that map \(x\) to \(y\) and agree up to degree \(k\) on their derivative. But it's better to think of it as a bundle (over \(X, Y\) or \(X \times Y\)):  a function gives for every point of \(X\) a \(k\)-jet based on the point; so a thing that is in the fiber \(J^k(X,Y)_{x,f(x)}\). The topology of this bundle is given by local charts using that the \(k\)-polynomials are a vector space.

So, an open set \(U \subset J^k(X,Y)\) is (a union of) coordinate charts of the kind explained above. So you can just think that there is a natural way of saying how close or far away are jets from each other, because jets are simple.
\end{defn}
\fi
\begin{thing1}{Problem 1}\label{p:1}\leavevmode
Let  \(\Phi=\{\varphi_i,U_i\}_{i \in I}\) be a locally finite atlas of \(X\), \(\mathcal{K}=\{K_i\}_{i \in I}\) a family of compact sets \(K_i \subset U_i\), \(\Psi=\{\psi_i,V_i\}\) an atlas of \(Y\), \(\varepsilon=\{\varepsilon_i\}_{i \in I}\subset \mathbb{R}^+\) and \(f\in C^\infty(X,Y)\) such that \(f (K_i) \subset V_i\). Let
\begin{align*}
W^k(f;\Phi,\Psi,\mathcal{K},\varepsilon)&=\left\{ g \in C^\infty(X,Y)  :\begin{aligned}
&\qquad  \qquad g(K_i)\subset V_i,\\& \|D^r(\psi_i \circ f \circ \varphi_i^{-1})(x)-D^r(\psi_i \circ g \circ \varphi_i^{-1})(x)\|<\varepsilon_i,\\
&\qquad \forall x \in K_i,\qquad \forall i \in I, \qquad \forall r\text{ s.t. } \; 0 \leq r\leq k
\end{aligned}\right\}
\end{align*}\
Prove that the collection of all sets of this form is a base for the \(C^k\) topology on \(C^\infty(X,Y)\).
\end{thing1}

\begin{proof}[Solution]\leavevmode
Fix \(U\) a \(C^\infty\)-open set and  \(f \in U\). We need to show that there is a set of the form \(W^k(f;\Phi,\Psi,\mathcal{K},\varepsilon)\) contained in \(U\).

Since \(f \in U\), by definition of the \(C^\infty\) topology there must be a set
\[M(V)=\{g \in C^\infty(X,Y): j^kg(x) \in V \;\forall x \in X\} \subset U\]
for some  \(V\) open in \(J^k(X,Y)\). We need to construct a set \(W\) contained in \(M(V)\). Fix for now any atlases  \(\Phi,\Psi\) and set of compact sets \(\mathcal{K}\) and set of numbers \(\varepsilon\). To get the contention \(W:=W^k(f;\Phi,\Psi,\mathcal{K},\varepsilon)\subset M(V)\) we must show that any \(g \in W\) satisfies \(j^kg(X) \subset V\), which ``is the definition of being in \(V\)". Indeed, the open 

\end{proof}

\begin{thing1}{Problem 2}\label{p:2}\leavevmode
A function is said to be \textit{\textbf{closed}} if the image of every closed set is closed. Prove that the set \(\{f \in C^\infty(\mathbb{R},\mathbb{R}): f \text{ is closed} \}\) is closed in the \(C^\infty\) topology.
\end{thing1}

\begin{proof}[Solution]\leavevmode
Choose a non-closed function $f$ and around it find a \(C^k\)-neighbourhood of non-closed functions, for some \(k\). This means that when we put the \(C^k\) glasses, we see less clearly, we can barely distinguish functions up to degree \(k\) (further than that they could be different but we won't notice).

If $f$ is non-closed then it cannot be a submersion: a submersion is open. Then there is a point where \(d_pf\) is not surjective. I want a neighbourhood of functions that have the same derivative as $f$ (but they are still \(k\)-close to $f $). So they can be different in second derivative. So all these functions are the same function in \(k=1\) topology. Do I not need \(C^0\) topology for that? To distinguish them. But \(C^0\) topology gives me the functions that map \(x \mapsto y=f(x)\).
\end{proof}

\begin{thing1}{Problem 3}\label{prob:3}\leavevmode
Let \(X\) and \(Y\) be manifolds and \(\ell \geq  k\). Prove that there exists a natural fiber bundle \(J^\ell(X,Y) \to J^k(X,Y)\) and compute the dimension of its fiber.
\end{thing1}

\begin{proof}[Solution]\leavevmode
Consider the map
\begin{align*}
	\pi: J^\ell(X,Y) &\longrightarrow J^{k}(X,Y) \\
	j^\ell f(p) &\longmapsto j^k f(p)
\end{align*}
for any smooth function \(f \in C^\infty(X,Y)\) and \(p \in X\).

First notice that \(\pi\) is a submersion. Fix a point \(p \in X\) and a jet \(\sigma:= j^\ell f(p)\). After fixing local coordinates on \(X\) and \(Y\), \(\sigma\) has a coordinate representation \[\Big((x_1,\ldots,x_n), (y_1,\ldots,y_m),T_\ell(\sigma)\Big)\] where \(T_\ell\) gives the Taylor polynomials of the coordinate functions of \(f\) with respect to the chosen coordinates. Then the coordinates of \(\pi\sigma\) are given simply by composing \(T_\ell\) with \(T_k\), which basically means ``forgetting" the coefficients of the Taylor polynomials for degrees above \(k\). This just says that the differential will be the identity in the coordinates of the points in \(X\) and \(Y\), and also for the first \(k\) coordinates of the Taylor polynomials. The rest of the matrix will have zeroes, but it will be full rank since the dimension of \(J^k(X,Y)\) is smaller than that of  \(J^\ell(X,Y)\).

Now let's look at \(\pi^{-1}(\sigma)\), it's the manifold given by all the \(\ell\)-jets that coincide with \(\sigma\) up to order \(k\). Computing its dimension is analogous of computing the dimension of jet spaces in general: it will be the product of the dimensions of \(X\), \(Y\), and the dimension of a certain polynomial space. For the polynomial coordinates we need to consider how the Taylor expansion of the coordinate functions can vary in degrees between  \(k+1\) and \(\ell\). It is \(m \sum_{r=k+1}^\ell \binom{r+n-1}{n-1}\) according to the following combinatorial argument.

For each coordinate function of \(f\), we consider its Taylor polynomial at \(p\), which is a polynomial in  \(n\) variables. We need to put a number (a coordinate) at every monomial. Every monomial is determined by the exponents we put in each indeterminate. The exponents should add up to the degree of the monomial, say \(r\) where \(k<r\leq \ell\). Thus different monomials are different choices of \(n\) nonnegative integers that add up to \(r\). That's like putting \(r\) balls in \(n\) boxes, which is like putting \(n-1\) ``divisions" among the \(r\) objets. So it's a choice of \(n-1\) things among \(r+n-1\) thins. Taking into account the \(m\) polynomials, this gives the above number.

To conclude we must show that \(\pi\) admits local trivializations. Let \(\tau \in J^k(X,Y)_{p,q}\). An open neighbourhood \(U\) of \(\tau\) is a product of neighbourhoods of \(p\), \(q\) and \(T_k\tau\) in their respective spaces. The inverse image \(\pi^{-1}(U)\) only differs from \(U\) in the polynomial part, where now we consider polynomials up to degree \(\ell\) instead of only \(k\). A diffeomorphism \(\pi^{-1}(U) \cong U \times F\) is given as 
\begin{align*}
	 \pi^{-1}(U) &\longrightarrow U \times F \\
	 (p,q,T_\ell (\sigma)) &\longmapsto (p,q,T_k(\sigma)),
\end{align*}

\end{proof}

\begin{thing1}{Problem 4}\label{p:4}\leavevmode
Let \(M\) be a non-compact manifold.
\begin{enumerate}[label=(\alph*)]
\item Prove that multiplication by scalar \(\mathbb{R}\times C^\infty (M,\mathbb{R})\to C^\infty (M,\mathbb{R})\) is not continuous in the \(C^\infty \) topology.
\item Prove that addition and multiplication of functions are continuous in the \(C^\infty \) topology.
\end{enumerate}
\end{thing1}

\begin{proof}[Solution]\leavevmode
\begin{enumerate}[label=(\alph*)]
\item Fix \(f \in C^\infty (M,\mathbb{R})\). Then we have a multiplication map \(\mu_f:\mathbb{R}\to C^\infty (M,\mathbb{R}\). So maybe this is not continuous, i.e. there is a convergent sequence of numbers \((a_n)\) but \((a_nf)\) does not converge to \(a_0f\). Convergence means that for every open neighbourhood \(U\) of  $f$ there is \(N \in \mathbb{N}\) st \(\forall  n>N\), \(a_nf \in U\).

	Fix an open neighbourhood \(U\) of \(f\). Then every \(g \in U\) is in a \(C^k\)-open neighbourhood contained in \(U\). This means that the \(k\)-jet of every function in \(W\) is "\(k\)-polynomially-close" to \(g\).

\item 
	First notice (thanks to ChatGPT) that addition map
	\[A: C^\infty(M,\mathbb{R}) \times C^\infty(M,\mathbb{R}) \to C^\infty(M,\mathbb{R}),\qquad  (f,g)\mapsto  f+g\]
	induces a bundle map
\[\tilde{A}:J^{k}(M,\mathbb{R})\times J^{k}(M,\mathbb{R})\longrightarrow J^{k}(M,\mathbb{R})\]
which is smooth. Indeed, addition of two sections \(j^kf\) and \(j^kg\) is smooth since addition of real numbers is smooth. (To define this map formally we fix a point of \(M\) and map two \(k\)-jets at \(x\) to their sum at \(x\), which in coordinates gives a polynomial whose coefficients are sums of the coefficients of the original polynomials.)

Then we show that addition of smooth functions is \(C^\infty\)-continuous like this: Let \(U \subset C^\infty(M,\mathbb{R})\) open, then \(A^{-1}(U)\) is some set in \(C^\infty(M,\mathbb{R}) \times C^\infty (M,\mathbb{R})\). Take a point \((f,g)\) in there. Since \(f+g\) is in the open set \(U\), there is a \(k\) such that \(f+g \in M(V)\) for some \(V\) open in \(J^k(M,\mathbb{R})\). The preimage \(\tilde{A}^{-1}(V):=V_1\times V_2\) is open in \(J^k(M,\mathbb{R})\times J^{k}(M,\mathbb{R})\). Consider the \(C^\infty\)-open set \(M(V_1) \times M(V_2) \ni (f,g)\).

To conclude we must check that \(M(V_1)\times M(V_2) \subset A^{-1}(M(V))\). This means that the sum of any pair of smooth functions in \(M(V_1) \times M(V_2)\) remains in \(M(V)\). But any two smooth induce a sum of \(k\)-jets that remains in \(V\) by construction.
\iffalse
	Then an open set \(V \subset J^{k}(M,\mathbb{R})\) gives a pair of \(C^k\)-open sets \(V_1\) and \(V_2\) given as the preimage under \(\tilde{A}\). 

	\textbf{(Dani.)} Let \(U\) be a \(C^\infty\)-open set in \(C^\infty(X,Y)\) and fix \((f,g) \in +^{-1}(U)\), that is \(f+g \in U\). We must show there are neighbourhoods of \(f\) and of \(g\) such that addition of any pair of functions taken from these neighbourhoods remains in \(U\).

	%A neighbourhood of \(f\) or \(g\) in the \(C^\infty\) topology is a union of open set of different \(C^k\) topologies. To fix an open neighbourhood of $f$ is the same as fixing an open neighbourhood for some \(C^k\). These, in turn, are given by open sets of the jet bundle \(J^k(X,\mathbb{R})\).

	Since \(U\) is open in \(C^\infty\), for \(f+g \in U\) there is a \(k\) and a \(V\) open in \(C^k\) such that \(f + g \in V \subset U\). This means that the \(k\)-jets of functions in \(V\) remain are close to \(f+g\).

	Consider the translations \(V_1=V-g\) and \(V_2=V-f\). Perhaps translating is a diffeomorphism. If so, these are neighbourhoods of \(f\) and \(g\), respectively. Pick \(f' \in V_1\) and \(g' \in V_2\).

	
	\textbf{(cite{gui2}.)} This question is in fact Coro. 3.8. The following result was proved in class:

	\begin{thing6}{Proposition 3.5}[\cite{gui2}]\label{prop:3.5}\leavevmode
Let \(X,Y,Z\) be smooth manifolds. Let \(\phi:Y \to Z\) be smooth. Then the mapping \(\phi_*:C^\infty (X,Y) \to C^\infty(X,Z)\) given by \(f \mapsto \phi \circ f\) is a continuous mapping in the \(C^\infty\) topology.
	\end{thing6}
	To prove that addition is continuous it's enough to notice that \(+: \mathbb{R} \times \mathbb{R} \to \mathbb{R}\) is smooth and that in fact addition of functions in \(C^\infty(X,Y)\) is the same as the induced map or pushforward \(+_*:C^\infty(X,\mathbb{R} \times \mathbb{R} \to C^\infty(X,\mathbb{R})\) mapping \((f,g) \mapsto f+g\).\fi

	Multiplication of functions is analogous; this time we should use a bundle map \(\tilde{M}\) which is also continuous since multiplication of \(k\)-jets is locally multiplication of polyonimals.
\end{enumerate}
\end{proof}

\begin{thing1}{Problem 5}\label{prob:5}\leavevmode
Let \(X\) be a submanifold of \(\mathbb{R}^n\) and \(k \geq  \operatorname{codim}X\). Prove that almost every subspace of dimension \(k\) intersects \(X\) transversally, i.e. the set of all subsets of dimension \(k\) that don't intersect \(X\) transversally has measure zero.
\end{thing1}

\begin{proof}[Solution]\leavevmode
Let \(\mathcal{T}\subset \operatorname{Gr}(n,k)\) be the set of all subspaces of dimension \(k\) that don't intersect \(X\) transversally. To use Sard's theorem it's enough to show that it is the set of critical points of a smooth function. Consider
\begin{align*}
	d: \operatorname{Gr}(n,k) &\longrightarrow \mathbb{Z} \\
	V &\longmapsto \dim (V \cap T_pX)
\end{align*}
where \(p\) is 
\end{proof}



\bibliography{bib.bib}
\end{document}
