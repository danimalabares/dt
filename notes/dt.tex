\input{/Users/daniel/github/config/preamble-por.sty}%available at github.com/danimalabares/config
%\input{/Users/daniel/github/config/thms-por.sty}%available at github.com/danimalabares/config

\newcommand{\rightlooparrow}{\mathbin{
    \vbox{\openup-10.25pt\halign{\hss$##$\hss\cr\circ\cr\longrightarrow\cr}}
}}

\begin{document}
\bibliographystyle{alpha}

\begin{minipage}{\textwidth}
	\begin{minipage}{1\textwidth}
		Topologia diferencial\hfill Daniel González Casanova Azuela
		
		{\small Prof. Vinicius Ramos\hfill\href{https://github.com/danimalabares/k3}{github.com/danimalabares/dt}}
	\end{minipage}
\end{minipage}\vspace{.2cm}\hrule

\vspace{10pt}
{\huge Topologia Diferencial}
\tableofcontents
\section{Aula 1}

\subsection{Plano do curso, bibliografia}

Cronograma
\begin{enumerate}
	\item[0.] Revisão de variedades.
	\item Transversalidade: Sard, top. forte, fraca, aproximação.
	\item  Teoria da interseção e indice.
	\item Teoria de Morse.
	\item Tópicos adicionais (possiveis): h-cobordismo, top. de baixa dimensão, Poincaré \(n\geq 5\).
\end{enumerate}

Bibliografía: \cite{milnordt} (intuição),  \cite{gui} (tranqui, tem muito), \cite{hirsch} (pesado, tem tudo, e importante ler, usa Análise Funcional). 

\subsection{Resumo da aula 1}

\begin{enumerate}
\item Revisão de vriedades, espaço topológico, 2-enumerável, 2-contável, Hausdorff, loc. euclidiano, dimensão é fixa nas componentes conexas, def. de carta, atlas, atlas \(C^k\), atlas maximal. \textbf{Obs.} Existem atlas que não contém sub atlas \(C^k\).
\item \textbf{Teorema.} \(k=1,\ldots, +\infty\) tuda \(C^k\)-variedade é  \(C^k\)-difeomorfa a uma \(C^\infty\)-variedade.
\item \textbf{Teorema.} \(1 \leq  \ell \leq  k \leq +\infty\), se \(M,N\) são \(C^k\)-variedades, \(C^\ell\)-difeomorfas, então \(M\) e $N$ são \(C^k\)-difeomorfas. {\color{2}No será \(\ell?\)}
\item \textbf{Partições da unidade}. Definição. \textbf{Exercício:} toda variedade topológica é paracompacta. \textbf{Teorema:} \(M\) variedade \(C^\infty\) e \(\{ U_i\}\) cobertura, então existe \(C^\infty\) partição da unidade subordinada. 
\end{enumerate}


\section{Aula 2}

\subsection{Lembre}

Dada uma variedade suave \(M\). Definimos como velocidades de curvas ou como derivações: \(T_pM\) é um espaço vetorial de dimensão $n$, onde para \(p \in U\), \((U, \varphi)\) carta, \(\varphi=(x^1,\ldots,x^n\) com base \(\left\{ \frac{\partial }{\partial x_1}\Big|_{p},\ldots,\frac{\partial }{\partial x^n}\Big|_{p} \right\} \). O \textit{\textbf{espaço cotangente}} é
\[T^*_p M=(T_pM)^* =\operatorname{Hom}(T_pM,\mathbb{R}).\]
A base dual é \(\left\{ dx^1|_{p},\ldots,dx^n|_{p} \right\} \) dada por
\[ dx^i|_{p}=\left(\frac{\partial }{\partial x^j}\right)\Big|_{p}=\delta_i^j=\begin{cases}
	1\qquad &\text{se } i=j \\
	0\qquad &\text{se não} 
\end{cases}\]

e ai extendemos por linearidade a todos os demais covetores.

\begin{remark}\leavevmode
	Note que mudando de carta a gente muda de base---não tem uma base canônica do espaço cotantente.
\end{remark}

\subsection{Fórmula de mudança de bases}

\begin{thing6}{Fórmula de mudança de bases}[Exercício]\leavevmode
\((U,\varphi),(V,\psi), p \in U \cap V\), \(\varphi=(x^1,\ldots,x^n\), \(\psi(y^1,\ldots,y^n\) com bases
\[\left\{ \frac{\partial }{\partial x_1}\Big|_{p},\ldots,\frac{\partial }{\partial x^n}\Big|_{p} \right\}, \qquad \left\{ \frac{\partial }{\partial y_1}\Big|_{p},\ldots,\frac{\partial }{\partial y^n}\Big|_{p} \right\},\]
mostre que
\[\frac{\partial }{\partial x^j}=\sum_{i=1}^n \frac{\partial y^i}{\partial x^j}\frac{\partial }{\partial y^i}\]

\end{thing6}

\subsection{Fibrado tangente}
\(M\) variedade,
\[TM:=\bigsqcup_{p \in M}T_pM.\]

Note que para toda carta \((U,\varphi)\) existe uma bijeção
\begin{align*}
	\phi^{-1}:U \times \mathbb{R}^n &\longrightarrow \pi^{-1}(U) \\
	\Big(p,(v_1,\ldots,v_n) \Big) &\longmapsto \sum_{i=1}^n v_i\frac{\partial }{\partial x^i}
\end{align*}
usando essa bijeção, topologizamos \(TM\). Mas ainda, induz uma estrutura de variedade topológica com cartas dadas pelas \(\phi\). Mas exatamente, as cartas são
\begin{align*}
	\phi_{(U,\varphi)}: \pi^{-1}(U) &\longrightarrow \varphi(U)\times\mathbb{R}^n \subset \mathbb{R}^{2n} \\
	\sum v_i \frac{\partial }{\partial x^i}\Big|_{p} &\longmapsto \Big(\varphi(p),(v_i) \Big)
\end{align*}
e a mudança de coordenadas também é \(C^\infty\), i.e. esa estrutura é diferenciável.

\begin{remark}\leavevmode
	Se variedade é \(C^k\), o fibrado tangente é \(C^{k-1}\).
\end{remark}

A gente vai fazer isso mesmo com o fibrado cotangente:
\[T^*  M= \bigsqcup_{p \in M} T^*_pM.\]
O mesmo procedimento mostra que \(T^* M\) é uma \(C^\infty\)-variedade de dimensão \(2n\).

\begin{remark}\leavevmode
	Para todo \(p \in M\) existe \(U \ni p\) vizinhança tal que \(\pi_1(U) \cong U \times \mathbb{R}^n\). {\color{2}Mas \(TM \not\cong M \times \mathbb{R}^n\) em geral}; nesse caso dizemos que \(M\) é \textit{\textbf{paralelizável}}.
\end{remark}

\begin{thing6}{Casos onde \(TM \cong M \times \mathbb{R}^n\)}\leavevmode
\begin{enumerate}
\item \(M \cong \mathbb{R}^n\), \(TM \cong \mathbb{R}^n \times \mathbb{R}^n\).
\item  \(M= S^1\), \(TS^1 \cong S^1 \times \mathbb{R}\).
\item  \(M\) 3-variedade orientável, então \(TM \cong M \times \mathbb{R}^3\). (Difícil mas verdadeiro.) \textbf{Hint.} Usando quaternios não é difícil obter uma base global.
\end{enumerate}
\end{thing6}

\subsection{Imersões e mergulhos}

Até agora definimos funções suaves, mas não o que é a diferencial delas.

\begin{defn}\leavevmode
	\(M,N\) variedades suaves e \(f:M \to N\) suave. A \textit{\textbf{derivada de $f$}} é
	\[Df_p:T_pM \to T_{f(p)}N,\]
uma aplicacão linear que pode ser definida usando a definição do espaço tangente de curvas ou de derivações. Se pensamos que \(v\) é uma clase de equivalência de curvas,
\(Df_p[\gamma]=[f \circ \gamma].\)
Se \(v: C^\infty(M) \to \mathbb{R}\) é uma derivação, a definição é o pus
rward
\begin{align*}
	Df_pv: C^\infty(N) &\longrightarrow \mathbb{R} \\
	(Df_pv)g &\longmapsto v(g \circ f).
\end{align*}
Tem outra forma de definir, que usando cartas coordenadas, onde \(Df_p\) está dada como uma matriz em termos das bases locais: em cartas \((U,\varphi),(V,\psi)\) de \(p\) e \(f(p)\), \(\varphi=(x^1,\ldots,x^n)\) e \(\psi=(y^1,\ldots,y^n)\). A notação fica
\[Df_p\left(\frac{\partial }{\partial x^j}|_{p}\right) =\sum_{i=1}^n\frac{\partial f_i}{\partial x^j}|_{p}\frac{\partial }{\partial y^i}|_{f(p)}\]
onde \(\frac{\partial f_i}{\partial x^j}\) é definida como
\[D(\psi \circ f \circ \varphi^{-1})_{ij}=\frac{\partial }{\partial x^j}(\psi \circ \varphi \circ \varphi^{-1})\]
\end{defn}

\begin{defn}\leavevmode
Seja \(f:M \to N\) uma função suave. \(f\) é uma \textit{\textbf{imersão em $p$}} se a derivada \(Df_p\) é injetiva. \(f\) é uma \textit{\textbf{submersão em $p$}} se \(Df_p\) é sobrejetiva. \(f\) é um \textit{\textbf{mergulho}} se é uma imersão injetiva tom inversa \(g:f(M) \to M\) contínua.
\end{defn}

\begin{example}\leavevmode
	O exemplo mas fácil é o caso das incusões em variedades produto:
	\begin{align*}
		M &\longrightarrow M \times N \\
		p &\longmapsto (p,q)
	\end{align*}
	E as projecões:
	\begin{align*}
		M \times N &\longrightarrow M \\
		(p,q) &\longmapsto p
	\end{align*}
	Outros exemplos de submersões são as projeções dos fibrados tangente e cotangente.
\end{example}

Para ver por que na definição de mergulho pedimos que a inversa seja contínua, considere o seguinte contraexemplo: \(\mathbb{R} \to \mathbb{R}^2\) uma curva que tem um ponto límite demais: a topologia no domínio é uma linha, mas a topologia no contradomínio e de um outro espaço, mas $f$ é um mergulho injetivo! A inversa de $f$ não é contínua (não manda limites em limites).

\begin{remark}\leavevmode
	Se \(f:M \to N\) é um mergulho, então \(f(M)\) herda uma estrutura de variedade diferenciável e $f$ é um difeomorfismo entre \(M\) e \(f(M)\).
\end{remark}

\begin{upshot}\leavevmode
	Merhulo são as treis condições que precisamos para que a imagem de $f(M)$ tenha estrutura diferenciável e \(f\) um difeomorphismo entre \(M\) e \(f(M)\). O lance é usar o teorema da função inversa. \(f(M)\) é chamada de uma \textit{\textbf{subvariedade}} de $N$.
\end{upshot}
Uma definição alternativa de \textit{\textbf{subvariedade}} é que para cada ponto \(p \in  Q \subset M\), \(Q\) subespaço topológico, existe uma carta de $N$ tal que \(\varphi(U \cap Q)=\mathbb{R}^k\). (Misha's). Tem uma terceira definição: \(Q\) é a imagem de um mergulho; para isso pode usar a inclusão como o mergulho.
In Misha's handouts:
\begin{thing4}{Exercise 2.23}\label{exer:2.23}\leavevmode
Let \(N_1,N_2\) be two manifolds and let \(\varphi_i:N_i\to M\) be smooth embeddings. Suppose that the image of \(N_1\) coincides with that of \(N_2\). Show that \(N_1\) and \(N_2\) are isomorphic.
\end{thing4}

\begin{thing5}{Remark 2.10}\leavevmode
By the above problem, in order to define a smooth structure on $N$, it sufficies to embed $N$ into \(\mathbb{R}^n\). As it will be clear in the next handout, every manifold is embeddable into \(\mathbb{R}^n\) (assuming it admits partition of unity). Therefore, in place of a smooth manifold, we can use ``manifolds that are smoothly embedded into \(\mathbb{R}^n\)".
\end{thing5}

\begin{thing3}{Notação}\leavevmode
Se \(f:M \to N\) é uma imersão escrevemos \(M \rightlooparrow N\), se é mergulho \(M \hookrightarrow N\) e se é submersão \(f: M \twoheadrightarrow N\).
\end{thing3}
Uma \textit{\textbf{subvariedade imersa}} é a imagem de uma imersão (que pode nem ser variedade…)

\begin{remark}\leavevmode
	\(Q \subset M\) subvariedade, então existe uma inclusão natural \(T_qQ \subset T_q M\) (linear injetiva) para todo \(q \in Q\). Claro, a derivada da inclusão \(\iota:Q \to M\), i.e. \(D\iota_q:T_qQ \to T_qM\).
\end{remark}
kj
Dado \(q \in Q\), existe \((U,\varphi)\) carta de \(M\) tal que \(\varphi|_{U \cap Q}\) é uma carta de \(Q\), é só botar a base \(\left\{\frac{\partial}{\partial x_1}\Big|_{p},\ldots,\frac{\partial}{\partial x^n}\Big|_{p}\right\}\) dentro da base de \(M\).

\subsubsection{Valores regulares}
\begin{defn}\leavevmode
	Seja \(f:M \to N\) \(C^\infty\), um ponto \(y \in N \) é dito \textit{\textbf{valor regular}} se $f$ é uma submersão em $x$ para todo \(x \in f^{-1}(y)\) i.e. \(Df_x\) é sobrejetiva para todo \(x \in f^{-1}(y)\).
\end{defn}

\begin{thm}[Do valor regular]\leavevmode
Se \(y \) é um valor regular de $f$, então \(f^{-1}(y)\) é uma subvariedade de \(M\) de dimensão \(\dim M- \dim N\). (Se \(f^{-1}(y)\neq \varnothing\).)
\end{thm}

\begin{remark}\leavevmode
	Isso é só outra encarnação do teorema da função implícita.
\end{remark}

\begin{proof}\leavevmode
\(x \in f^{-1}(y):=Q\). Pega cartas \(\varphi\) de $x$ e \(\psi\) de $y$. Supondo que \(f(U) \subset V\), e que \(x,y\) tem coordenadas 0.
\[\begin{tikzcd}
	U \subset M \arrow[r,"f"]\arrow[d,swap,"\varphi"]&  V \subset N\arrow[d, "\psi"]\\
	\mathbb{R}^m \arrow[ r, swap,"\Phi:\psi \circ f \circ \varphi^{-1}"]& \mathbb{R}^n
\end{tikzcd}\]
Note que \(\Phi(0)=0\) e que \(\Phi^{-1}(0)=\varphi(f^{-1}(y) \cap U)\).
\begin{claim}\leavevmode
	\(\Phi^{-1}(0)\) é uma subvariedade.
\end{claim}
Para tudo ficar claro vamos reescrever o teorema de função implícita.  \(\Phi'(0)\) é sobrejetiva. Temos que
\begin{align*}
	: \mathbb{R}^m &\longrightarrow \mathbb{R}^n \times \mathbb{R}^{m-n} \\
	z &\longmapsto \Phi(z)
\end{align*}
A ideia é que existe uma vizinhança \(W\) de \(0 \in \mathbb{R}^m\) e um difeomorfismo \(\eta:W \to W^{\smile}\) tal que
\begin{align*}
	\phi \circ \eta: W \subset \mathbb{R}^n \times \mathbb{R}^{m-n} &\longrightarrow \mathbb{R}^n \\
	(x_1,x_2) &\longmapsto x_1
\end{align*}
\end{proof}

\subsection{Fibrados vetoriais}
Um fibrado vetorial é uma coisa que generaliza os fibrados tangente e cotangente.
\begin{defn}\leavevmode
Sejam \(E, M\) variedades e \(\pi: E \to M\) submersão sobrejetiva. Dizemos que \(\pi\) é um \textit{\textbf{fibrado vetorial}} se para todo \(p \in M\), \(\pi^{-1}(p)=E_p\) possui uma estrutura de espaço vetorial tal que para todo \(p \in M\) existe \(U \ni p\) aberto e um difeomorfismo \(\varphi: \pi^{-1}(U) \to U \times \mathbb{R}^n\) tal que o seguinte diagrama comuta
\[\begin{tikzcd}
\pi^{-1}(U)\arrow[rr,"\varphi"]\arrow[dr,swap,"\pi"]&&U \times \mathbb{R}^n\arrow[dl,"\operatorname{pr}_1"]\\
&U
\end{tikzcd}\]
e
\[\varphi|_{E_p}:E_p \to \{ p\}\times \mathbb{R}^n\]
é um isomorfismo.
\end{defn}

\begin{example}\leavevmode
	\(TM,T^* M,TM \oplus  TM, TM \otimes TM, \Lambda^{k}(TM),\Lambda^{k}(T^*M),\operatorname{Sym}^k(TM)\).
\end{example}

\subsection{Seções}

\begin{defn}\leavevmode
	Uma \textit{\textbf{seção}} de \(\pi:E \to M\) é \(s:M \to E\) suave tal que \(\pi \circ s = \operatorname{id}\)
\[\begin{tikzcd}
E\arrow[d,"\pi",swap]\\
M \arrow[u, bend right,swap,"s"]
\end{tikzcd}\]
Uma seção de TM é uma função \(X: M \to TM\) tal que \(X(p) \in T_pM\), um \textit{\textbf{campo vetorial}}.
\end{defn}

\begin{thm}[da bola cabeluda]\leavevmode
\(M = S^{n}\), \(n\) par, \(X:M \to TM\) campo vetorial, então existe \(p \in M\) tal que \(X(p)=0 \in T_pM\).
\end{thm}

\begin{thing6}{Notação}\leavevmode
\(\Gamma(E) = \{ \text{seções de \(\pi:E \to M\)} \}\), \(\Gamma(TM)=\mathfrak{X}(M)\), \(\Gamma(T^*M)=\Omega^{1}(M)\), \(\Gamma(\Lambda^{k}(T^*M))=\Omega^{k}(M)\).

Para qualquer espaço vetorial \(V\),
\[\operatorname{Sym}^2(V^*)=\{f:V \times V \to \mathbb{R},\text{ bilinear, }f(x,y)=f(y,x) \}\subset V^* \otimes V^*.\]
E para fibrado vetorial \(E\),
 \[\operatorname{Sym}^2(E)=\bigsqcup_{p \in M}\operatorname{Sym}^2(E^*_p).\]
\end{thing6}

\begin{defn}\leavevmode
	Uma \textit{\textbf{métrica Riemanniana}} em \(E\) é uma seção \(s: M \to \operatorname{Sym}^2(E)\) tal que \(s(p):E_p \times E_p \to \mathbb{R}\) é positiva definida, i.e. \(s(p)(x,x)>0\) se \(x \neq  0\).
\end{defn}

\begin{remark}[Aprox.]\leavevmode
	Todo fibrado vetorial tem uma métrica Riemanniana: usando a métrica euclidiana dada em cada carta, usamos uma partição da unidade para extender a uma seção global, somar e notar que fica positiva definida.
\end{remark}

É muito fácil construir seções do fibrado cotangente: para \(f \in C^\infty(M)\), a diferencial \(df :M \to T^*M\) é uma seção do fibrado cotangente, i.e. \(df \in \Gamma(T^*M)\) porque
\[df_p=Df_p:T_pM \to T_{f(p)}\mathbb{R}\]

\begin{exercise}\leavevmode
	Qualquer seção é um mergulho de \(M\) em \(E\).
\end{exercise}

\begin{thing6}{Mais uma}\leavevmode
$g$ uma métrica Riemanniana em \(TM\).
\[g_p:T_pM \times T_p M \to \mathbb{R}\]

\begin{align*}
	g_p^\sharp :T_pM &\longrightarrow  T^*_pM\\
	v &\longmapsto g(v,\cdot)
\end{align*}
Então o \textit{\textbf{gradiente}} de \(f\) é
\[(g^\sharp _p)^{-1}(df_p):=\operatorname{grad}_pf\]
\end{thing6}

\section{Aula 3: Teorema de Sard}

\subsection*{Teorema da função implícita (aula pasada)}

Uma função \(f:\mathbb{R}^n \to \mathbb{R}^m\) suave tal que \(f(0)=0\) e  \(f'(0)\) é sobrejetiva (\(\implies n \geq m\)). Então existe uma vizinhança de \(0 \in \mathbb{R}^n\) \(U\) e \(\tilde{U}\) \textbf{e um difeomorfismo} \(\varphi: U \to \tilde{U}\) tal que
\begin{align*}
	f \circ \varphi: \mathbb{R}^m \times \mathbb{R}^{n-m} &\longrightarrow \mathbb{R}^m \\
	(x,y) &\longmapsto x
\end{align*}

\begin{proof}\leavevmode
Parecido como a prova de \cite{tus} no teorema do valor regular, usando uma matrix com um \(*\), a identidade, e uma matriz invertível.
\end{proof}

\subsection{Transversalidade: Teorema de Sard}

A prova do teorema de Sard é muito técnica. Porém, a parte difícil é só análise em \(\mathbb{R}^n\).

Pegue \(a=(a_1,\ldots,a_n),b=(b_1,\ldots,b_n) \in \mathbb{R}^n\), defina um \textit{\textbf{cubo}} como sendo
\[c(a,b)=\prod_{i=1}^n ]a_i,b_i[ \subset \mathbb{R}^n.\]
Note que \(\operatorname{Vol}(a,b)=\prod_{i=1}^n(b_i-a_i).\)

\begin{defn}\leavevmode
\(S \subset \mathbb{R}^n\) possui \textit{\textbf{medida nula}} se \(\forall  \varepsilon >0\) existe \(\{c_i\}_{i=1}^\infty\) cubos (ou bolas) tais que \[S \subset \bigcup_{i=1}^\infty \operatorname{Vol}(c_i)<\varepsilon\]
\end{defn}

\begin{prop}\leavevmode
	\begin{enumerate}
	\item Uma união enumerável de conjuntos de medida nula tem medida nula.
	\item \(f : \mathbb{R}^n \to \mathbb{R}^n\) \(C^1\) e \(S \subset \mathbb{R}^n\) tem medida nula, então \(f(S)\) tem medida nula.
	\end{enumerate}
\end{prop}

\begin{proof}\leavevmode
\begin{enumerate}
\item \(\{S_i\}\) enumerável de medida nula, para cada $i$ você pode escolher cubos \(C^i_1,C^i_2,\ldots\) que cobren \(S_i\) e tal que a soma dos volumes deles é menor do que \(\sum_j \operatorname{Vol}(C^i_j)<\frac{\varepsilon}{2^i}\). Vai ver que a soma dos volumeis variando tanto \(i\) como \(j\) da \(\varepsilon\).

\item (Foto)
\end{enumerate}
\end{proof}

\begin{defn}\leavevmode
	\(X\) variedade diferenciável. \(S \subset X\). Dizemos que \(S\) tem \textit{\textbf{medida nula}} se \(\exists \{U_i\}_{i=1}^\infty\) cobertura aberta de \(S\), i.e. \(\bigcup_{i=1}^\infty S_i \supset S\), e cartas \(\varphi_i:U_i\to \mathbb{R}\) e \(S_i \subset U_i\) e \(\varphi(S)\) tem medida nula.

	O más bien: sólo el chiste es que cada conjunto tiene medida en \(\mathbb{R}^n\) cuando proyectas con cualquier carta.
\end{defn}

\begin{coro}\leavevmode
	\begin{enumerate}
	\item \(\{S_i\}_{i=1}^\infty\).  \(S_i \subset X\) medida nula, entao \(\bigcup_{i \in \mathbb{N}}S_i\) tem medida nula.
	\item \(X^n, Y^n\) variedades, \(f:X \to Y\) suave, \(S \subset X\) medida nula. Então \(f(S)\) tem medida nula.
	\end{enumerate}
\end{coro}

\begin{prop}\leavevmode
\(Y^n\) variedade, \(X^m \subset Y^n\) subvariedade de dimensão \(m <n\). Então \(X\) tem medida nula.
\end{prop}

\begin{proof}\leavevmode
É simplesmente levar para \(\mathbb{R}^n\): considera \(X_i\) como a parte de \(X\) que está den'de cada  \(U_i\) no atlas de \(Y\) e vai ver que ele tem dimensão menor. Daí é só provar que subespaços (acho que lineares) de dimensão menor em \(\mathbb{R}^n\) tem dimensão menor.
\end{proof}

\begin{coro}[Minisard]\leavevmode
\(X^m,Y^n\) variedades \(m<n\)  e \(f:X \to Y\) suave. Então \(f(X)\) tem medida nula.
\end{coro}

\begin{proof}\leavevmode
Aqui se usa o corolário: usar a inclusão \(\iota:X \to X \times \mathbb{R}^{n-m}, \qquad x \mapsto (x,0)\), compor com \(\tilde{f}:X \times R^{n-m}\to Y\), \((x,y) \mapsto  f(x)\). Então \(\tilde{f}(i(X))=f(X)\). O lance é que \(\iota(X)\) é uma subvariedade de codimensão positiva e com medida nula. Daí f(X) também.
\end{proof}

\begin{coro}[Versão fácil do teorema de mergulho de Whitney]\leavevmode
	Se \(X^n\) variedade diferenciável compacta, então existem 
	\[X \hookrightarrow \mathbb{R}^{2n+1},\qquad  X\rightlooparrow \mathbb{R}^{2n}\]
\end{coro}

\begin{thm}[Difícil de Sard]\leavevmode
\[X \hookrightarrow \mathbb{R}^{2n},\qquad  X\rightlooparrow \mathbb{R}^{2n-1}\]
\end{thm}

\begin{proof}\leavevmode
\begin{enumerate}[label=\textbf{Step \arabic*}]
\item Mergulhar a variedade num espaço euclidiano \textit{grande}. Pegue um atlas finito \(\{(U_i,\varphi_i)_{i=1}^k\}\), note que \(\varphi_i:U_i \to \mathbb{R}^n\) são mergulhos.

\begin{thing6}{Ideia}\leavevmode
\begin{align*}
	\Phi: X &\longrightarrow \mathbb{R}^n\times \mathbb{R}^n \times\ldots\times \mathbb{R}^n \subset \mathbb{R}^{nk} \\
	p &\longmapsto (\varphi_1(p),\varphi_2(p),\ldots
\end{align*}
Isso não da. Para fazer bem precisamos de uma partição da unidade \(\{\rho_i\}_{i=1}^k\) subordinada a \(\{U_i\}_{i=1}^k\) sobertura. Defina \(\rho_i\varphi_i:X \to \mathbb{R}^n\) como sendo zero fora do conjunto bom; note que essa função não é mais um mergulho, mas tudo bem. Agora faça \(X \to (\mathbb{R}^n)^k\times \mathbb{R}^k=\mathbb{R}^{nk+k}\)
\begin{align*}
	\Phi: X &\longrightarrow (\mathbb{R}^n)^k \times\mathbb{R}^k = \mathbb{R}^{nk+k} \\
	p &\longmapsto \Big((\rho_1 \varphi_1)(p),\ldots,\Big(\rho_k \varphi_k)(p)\Big)
\end{align*}
\begin{exercise}[Importante]\leavevmode
	Mostre que \(\Phi\) é uma imersão injetiva.
\end{exercise}
\end{thing6}

\item \textbf{Afirmação:}
	\[X \hookrightarrow  \mathbb{R}^n \implies \begin{cases}
		X \hookrightarrow \mathbb{R}^{N-1}\qquad &\text{ se $N>2n+1$}  \\
		X \rightlooparrow \mathbb{R}^{N-1} \qquad &\text{se $N>2n$.} 
	\end{cases}\]
\begin{proof}[Prova da afirmação]\leavevmode
Vamos projetar a variedade mergulhada em \(\mathbb{R}^n\) no plano ortogonal a algum vetor \(a \in \mathbb{R}^n\). Resulta que
\begin{exercise}\leavevmode
	\begin{align*}
	g: X \times X \times \mathbb{R} &\longrightarrow \mathbb{R}^N \\
	(x,y,t) &\longmapsto \operatorname{pr}_a \circ f
\end{align*}
é injetiva.
\end{exercise}
\end{proof}

\item \textbf{Ideia:} ver que em quase todo ponto podemos projetar.

Considere agora o mapa pusforward que pega um vetor tangente e manda mediante $f$:
\begin{align*}
	h: TX &\longrightarrow \mathbb{R}^N \\
	(x,v) &\longmapsto (Df)_xv
\end{align*}
Agora note que
\begin{thing4}{Afirmação}\leavevmode
\(a \not\in \operatorname{Im}(h) \iff \operatorname{pr}_a \circ f\) é uma imersão \(\iff\) \(D(\operatorname{pr}_a \circ f)_a\) é injetiva para toda $x$.
\end{thing4}

\item A prova termina usando minisard: as imagens de $g$ e de \(h\) tem medida nula. Mesmo a união delas. Então existe um ponto fora dessa união.
\end{enumerate}
\end{proof}

\begin{defn}\leavevmode
Sejam \(X^m, Y^k\) variedades, \(f:X \to Y\) suave, dizemos que
\begin{enumerate}[label=(\alph*)]
\item \(x \in X\) é \textit{\textbf{ponto crítico}} se o posto de \(Df_x\) é menor do que \(\operatorname{min}(m,n)\).
\item \(x \in X\) é \textit{\textbf{ponto regular}} se posto \(Df_x=\operatorname{min}(m,n)\).
\item \(y \in Y\) é \textit{\textbf{valor crítico}} se existe um ponto crítico tal que \(f(x) = y\).
\item \(y \in Y\) é \textit{\textbf{valor regular}} se \(\forall x \in f^{-1}(y)\), \(x\) é valor regular.
\end{enumerate}
\end{defn}

\begin{thm}[Sard]\leavevmode
\(f:X \to Y\) suave. Então \(\{\text{valores críticos} \}\) tem medida nula.
\end{thm}

\begin{remark}\leavevmode
\begin{enumerate}
\item Teorema vale se \(f\) é \(C^\ell\), onde \(\ell>\operatorname{max}(m-n,0)\). 

	. \end{enumerate}
\end{remark}

\begin{proof}\leavevmode
\begin{enumerate}[label=\textbf{Step \arabic*}]
\item \textbf{Redução para a versão local.} Supomos que \(X= \mathbb{R}^{m}, Y=\mathbb{R}^n\). \(f:U \subset \mathbb{R}^m \to \mathbb{R}^n\), \(U\) aberto.
	\[\operatorname{Crit}f= \{x \in 0:\text{posto } f'(x) < \operatorname{min}(m,n)\}\]
Então \(f(\operatorname{Crit}(f)\) tem medida nula. Para isso fazemos \textbf{indução em $m$.} \(m=0\) trivial.

\(C_i\) vai ser o conjunto onde as derivadas parciais se anulam até $i$:
 \[C_i=\left\{ p \in U: \frac{\partial^{(\alpha)}}{\partial x^{\alpha}}f_k(p)=0 \forall \alpha, 0<| \alpha|\leq 1, \forall  k \right\}.\]
 Note que \(C_{i+1}\subset C_i \subset C_{i-1}\subset\ldots C_1\subset C:=\operatorname{Crit}f\).

 \begin{thing7}{Objetivo}\leavevmode
 \(f(C)\) tem medida nula.
 \begin{enumerate}[label=\textbf{Paso \arabic*}]
 \item \(f(C_N)\) tem medida nula para algum  \(N \gg 0\). {\color{8}Crucial}
\item \(f(C_i \setminus C_{i+1}\) tem medida nula para toda $i$.
\item \(f(C\setminus C_i\) tem medida nula.
 \end{enumerate}

\begin{enumerate}[label=\textbf{Paso \arabic*}]
\item Podemos supor sem perda de generalidade que \(U\subset\)cubo, a fórmula de Taylor diz que
	\[\|f(x)-f(y)\|_\infty \leq K \cdot \|x-y\|_\infty^{i+1}\]
para todo \(x, y \in C_i\).

Tem que botar \(C_i\) den'de um cubo \(D_j\) que se divide em \(r^m\) cubos de lado \(b/r\). Então  \(f(D_j)\) está contido num cubo em \(\mathbb{R}^n\) de lado \(K \cdot \left(\frac{b}{r}\right)^{i+1}:=R_j \). Também note que pontos den'de \(D_j\) são tq. \(\|x-y\|_\infty \leq  \frac{b}{r}\).

Agora
\[f(C_i) \subset f\left(\bigcup_{j=1}^{r^m}D_j\right) \subset \bigcup_{j=1}^{r^m}f(D_j)\subset \bigcup_{j=1}^{r^m}R_j.\]
Então
\begin{align*}
\sum_{j=1}^{r^m}\operatorname{Vol}(R_j)&=r^m\cdot K^n\cdot \left(\frac{b}{r}\right)^{(i+1)\cdot n}\\
&= \frac{K^n \cdot b^{n(N+1)}}{r^{n(N+1)-m}}
\end{align*}
\end{enumerate}
 \end{thing7}

\item 
\end{enumerate}

\end{proof}













\bibliography{bib.bib}
\end{document}
