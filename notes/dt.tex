\input{/Users/daniel/github/config/preamble.sty}%available at github.com/danimalabares/config
\input{/Users/daniel/github/config/thms-eng.sty}%available at github.com/danimalabares/config

\newcommand{\rightlooparrow}{\mathbin{
    \vbox{\openup-10.25pt\halign{\hss$##$\hss\cr\circ\cr\longrightarrow\cr}}
}}
\usepackage[style=authortitle-terse,backend=bibtex]{biblatex}
\addbibresource{~/github/config/bibliography.bib}

\setcounter{secnumdepth}{2}

\begin{document}

\begin{minipage}{\textwidth}
	\begin{minipage}{1\textwidth}
		Topologia diferencial\hfill Daniel González Casanova Azuela
		
		{\small Prof. Vinicius Ramos\hfill\href{https://github.com/danimalabares/k3}{github.com/danimalabares/dt}}
	\end{minipage}
\end{minipage}\vspace{.2cm}\hrule

\vspace{10pt}
{\huge Topologia Diferencial}

\section{Aula 1}

Função suave. Espaço tangente.

\section{Aula 2}

\subsection{Lembre}

Dada uma variedade suave \(M\). Definimos como velocidades de curvas ou como derivações: \(T_pM\) é um espaço vetorial de dimensão $n$, onde para \(p \in U\), \((U, \varphi)\) carta, \(\varphi=(x^1,\ldots,x^n\) com base \(\left\{ \frac{\partial }{\partial x_1}\Big|_{p},\ldots,\frac{\partial }{\partial x^n}\Big|_{p} \right\} \). O \textit{\textbf{espaço cotangente}} é
\[T^*_p M=(T_pM)^* =\operatorname{Hom}(T_pM,\mathbb{R}).\]
A base dual é \(\left\{ dx^1|_{p},\ldots,dx^n|_{p} \right\} \) dada por
\[ dx^i|_{p}=\left(\frac{\partial }{\partial x^j}\right)\Big|_{p}=\delta_i^j=\begin{cases}
	1\qquad &\text{se } i=j \\
	0\qquad &\text{se não} 
\end{cases}\]

e ai extendemos por linearidade a todos os demais covetores.

\begin{remark}\leavevmode
	Note que mudando de carta a gente muda de base---não tem uma base canônica do espaço cotantente.
\end{remark}

\subsection{Fórmula de mudança de bases}

\begin{thing6}{Fórmula de mudança de bases}[Exercício]\leavevmode
\((U,\varphi),(V,\psi), p \in U \cap V\), \(\varphi=(x^1,\ldots,x^n\), \(\psi(y^1,\ldots,y^n\) com bases
\[\left\{ \frac{\partial }{\partial x_1}\Big|_{p},\ldots,\frac{\partial }{\partial x^n}\Big|_{p} \right\}, \qquad \left\{ \frac{\partial }{\partial y_1}\Big|_{p},\ldots,\frac{\partial }{\partial y^n}\Big|_{p} \right\},\]
mostre que
\[\frac{\partial }{\partial x^j}=\sum_{i=1}^n \frac{\partial y^i}{\partial x^j}\frac{\partial }{\partial y^i}\]

\end{thing6}

\subsection{Fibrado tangente}
\(M\) variedade,
\[TM:=\bigsqcup_{p \in M}T_pM.\]

Note que para toda carta \((U,\varphi)\) existe uma bijeção
\begin{align*}
	\phi^{-1}:U \times \mathbb{R}^n &\longrightarrow \pi^{-1}(U) \\
	\Big(p,(v_1,\ldots,v_n) \Big) &\longmapsto \sum_{i=1}^n v_i\frac{\partial }{\partial x^i}
\end{align*}
usando essa bijeção, topologizamos \(TM\). Mas ainda, induz uma estrutura de variedade topológica com cartas dadas pelas \(\phi\). Mas exatamente, as cartas são
\begin{align*}
	\phi_{(U,\varphi)}: \pi^{-1}(U) &\longrightarrow \varphi(U)\times\mathbb{R}^n \subset \mathbb{R}^{2n} \\
	\sum v_i \frac{\partial }{\partial x^i}\Big|_{p} &\longmapsto \Big(\varphi(p),(v_i) \Big)
\end{align*}
e a mudança de coordenadas também é \(C^\infty\), i.e. esa estrutura é diferenciável.

\begin{remark}\leavevmode
	Se variedade é \(C^k\), o fibrado tangente é \(C^{k-1}\).
\end{remark}

A gente vai fazer isso mesmo com o fibrado cotangente:
\[T^*  M= \bigsqcup_{p \in M} T^*_pM.\]
O mesmo procedimento mostra que \(T^* M\) é uma \(C^\infty\)-variedade de dimensão \(2n\).

\begin{remark}\leavevmode
	Para todo \(p \in M\) existe \(U \ni p\) vizinhança tal que \(\pi_1(U) \cong U \times \mathbb{R}^n\). {\color{2}Mas \(TM \not\cong M \times \mathbb{R}^n\) em geral}; nesse caso dizemos que \(M\) é \textit{\textbf{paralelizável}}.
\end{remark}

\begin{thing6}{Casos onde \(TM \cong M \times \mathbb{R}^n\)}\leavevmode
\begin{enumerate}
\item \(M \cong \mathbb{R}^n\), \(TM \cong \mathbb{R}^n \times \mathbb{R}^n\).
\item  \(M= S^1\), \(TS^1 \cong S^1 \times \mathbb{R}\).
\item  \(M\) 3-variedade orientável, então \(TM \cong M \times \mathbb{R}^3\). (Difícil mas verdadeiro.) \textbf{Hint.} Usando quaternios não é difícil obter uma base global.
\end{enumerate}
\end{thing6}

\subsection{Imersões e mergulhos}

Até agora definimos funções suaves, mas não o que é a diferencial delas.

\begin{defn}\leavevmode
	\(M,N\) variedades suaves e \(f:M \to N\) suave. A \textit{\textbf{derivada de $f$}} é
	\[Df_p:T_pM \to T_{f(p)}N,\]
uma aplicacão linear que pode ser definida usando a definição do espaço tangente de curvas ou de derivações. Se pensamos que \(v\) é uma clase de equivalência de curvas,
\(Df_p[\gamma]=[f \circ \gamma].\)
Se \(v: C^\infty(M) \to \mathbb{R}\) é uma derivação, a definição é o pus
rward
\begin{align*}
	Df_pv: C^\infty(N) &\longrightarrow \mathbb{R} \\
	(Df_pv)g &\longmapsto v(g \circ f).
\end{align*}
Tem outra forma de definir, que usando cartas coordenadas, onde \(Df_p\) está dada como uma matriz em termos das bases locais: em cartas \((U,\varphi),(V,\psi)\) de \(p\) e \(f(p)\), \(\varphi=(x^1,\ldots,x^n)\) e \(\psi=(y^1,\ldots,y^n)\). A notação fica
\[Df_p\left(\frac{\partial }{\partial x^j}|_{p}\right) =\sum_{i=1}^n\frac{\partial f_i}{\partial x^j}|_{p}\frac{\partial }{\partial y^i}|_{f(p)}\]
onde \(\frac{\partial f_i}{\partial x^j}\) é definida como
\[D(\psi \circ f \circ \varphi^{-1})_{ij}=\frac{\partial }{\partial x^j}(\psi \circ \varphi \circ \varphi^{-1})\]
\end{defn}

\begin{defn}[Imersão]\leavevmode
Seja \(f:M \to N\) uma função suave. \(f\) é uma \textit{\textbf{imersão em $p$}} se a derivada \(Df_p\) é injetiva. \(f\) é uma \textit{\textbf{submersão em $p$}} se \(Df_p\) é sobrejetiva. \(f\) é um \textit{\textbf{mergulho}} se é uma imersão injetiva tom inversa \(g:f(M) \to M\) contínua.
\end{defn}

\begin{example}\leavevmode
	O exemplo mas fácil é o caso das incusões em variedades produto:
	\begin{align*}
		M &\longrightarrow M \times N \\
		p &\longmapsto (p,q)
	\end{align*}
	E as projecões:
	\begin{align*}
		M \times N &\longrightarrow M \\
		(p,q) &\longmapsto p
	\end{align*}
	Outros exemplos de submersões são as projeções dos fibrados tangente e cotangente.
\end{example}

Para ver por que na definição de mergulho pedimos que a inversa seja contínua, considere o seguinte contraexemplo: \(\mathbb{R} \to \mathbb{R}^2\) uma curva que tem um ponto límite demais: a topologia no domínio é uma linha, mas a topologia no contradomínio e de um outro espaço, mas $f$ é um mergulho injetivo! A inversa de $f$ não é contínua (não manda limites em limites).

\begin{remark}\leavevmode
	Se \(f:M \to N\) é um mergulho, então \(f(M)\) herda uma estrutura de variedade diferenciável e $f$ é um difeomorfismo entre \(M\) e \(f(M)\).
\end{remark}

\begin{upshot}\leavevmode
	Merhulo são as treis condições que precisamos para que a imagem de $f(M)$ tenha estrutura diferenciável e \(f\) um difeomorphismo entre \(M\) e \(f(M)\). O lance é usar o teorema da função inversa. \(f(M)\) é chamada de uma \textit{\textbf{subvariedade}} de $N$.
\end{upshot}
Uma definição alternativa de \textit{\textbf{subvariedade}} é que para cada ponto \(p \in  Q \subset M\), \(Q\) subespaço topológico, existe uma carta de $N$ tal que \(\varphi(U \cap Q)=\mathbb{R}^k\). (Misha's). Tem uma terceira definição: \(Q\) é a imagem de um mergulho; para isso pode usar a inclusão como o mergulho.
In Misha's handouts:
\begin{thing4}{Exercise 2.23}\label{exer:2.23}\leavevmode
Let \(N_1,N_2\) be two manifolds and let \(\varphi_i:N_i\to M\) be smooth embeddings. Suppose that the image of \(N_1\) coincides with that of \(N_2\). Show that \(N_1\) and \(N_2\) are isomorphic.
\end{thing4}

\begin{thing5}{Remark 2.10}\leavevmode
By the above problem, in order to define a smooth structure on $N$, it sufficies to embed $N$ into \(\mathbb{R}^n\). As it will be clear in the next handout, every manifold is embeddable into \(\mathbb{R}^n\) (assuming it admits partition of unity). Therefore, in place of a smooth manifold, we can use ``manifolds that are smoothly embedded into \(\mathbb{R}^n\)".
\end{thing5}

\begin{thing3}{Notação}\leavevmode
Se \(f:M \to N\) é uma imersão escrevemos \(M \rightlooparrow N\), se é mergulho \(M \hookrightarrow N\) e se é submersão \(f: M \twoheadrightarrow N\).
\end{thing3}
Uma \textit{\textbf{subvariedade imersa}} é a imagem de uma imersão (que pode nem ser variedade…)

\begin{remark}\leavevmode
	\(Q \subset M\) subvariedade, então existe uma inclusão natural \(T_qQ \subset T_q M\) (linear injetiva) para todo \(q \in Q\). Claro, a derivada da inclusão \(\iota:Q \to M\), i.e. \(D\iota_q:T_qQ \to T_qM\).
\end{remark}
kj
Dado \(q \in Q\), existe \((U,\varphi)\) carta de \(M\) tal que \(\varphi|_{U \cap Q}\) é uma carta de \(Q\), é só botar a base \(\left\{\frac{\partial}{\partial x_1}\Big|_{p},\ldots,\frac{\partial}{\partial x^n}\Big|_{p}\right\}\) dentro da base de \(M\).

\subsubsection{Valores regulares}
\begin{defn}\leavevmode
	Seja \(f:M \to N\) \(C^\infty\), um ponto \(y \in N \) é dito \textit{\textbf{valor regular}} se $f$ é uma submersão em $x$ para todo \(x \in f^{-1}(y)\) i.e. \(Df_x\) é sobrejetiva para todo \(x \in f^{-1}(y)\).
\end{defn}

\begin{thm}[Do valor regular]\leavevmode
Se \(y \) é um valor regular de $f$, então \(f^{-1}(y)\) é uma subvariedade de \(M\) de dimensão \(\dim M- \dim N\). (Se \(f^{-1}(y)\neq \varnothing\).)
\end{thm}

\begin{remark}\leavevmode
	Isso é só outra encarnação do teorema da função implícita.
\end{remark}

\begin{proof}\leavevmode
\(x \in f^{-1}(y):=Q\). Pega cartas \(\varphi\) de $x$ e \(\psi\) de $y$. Supondo que \(f(U) \subset V\), e que \(x,y\) tem coordenadas 0.
\[\begin{tikzcd}
	U \subset M \arrow[r,"f"]\arrow[d,swap,"\varphi"]&  V \subset N\arrow[d, "\psi"]\\
	\mathbb{R}^m \arrow[ r, swap,"\Phi:\psi \circ f \circ \varphi^{-1}"]& \mathbb{R}^n
\end{tikzcd}\]
Note que \(\Phi(0)=0\) e que \(\Phi^{-1}(0)=\varphi(f^{-1}(y) \cap U)\).
\begin{claim}\leavevmode
	\(\Phi^{-1}(0)\) é uma subvariedade.
\end{claim}
Para tudo ficar claro vamos reescrever o teorema de função implícita.  \(\Phi'(0)\) é sobrejetiva. Temos que
\begin{align*}
	: \mathbb{R}^m &\longrightarrow \mathbb{R}^n \times \mathbb{R}^{m-n} \\
	z &\longmapsto \Phi(z)
\end{align*}
A ideia é que existe uma vizinhança \(W\) de \(0 \in \mathbb{R}^m\) e um difeomorfismo \(\eta:W \to W^{\smile}\) tal que
\begin{align*}
	\phi \circ \eta: W \subset \mathbb{R}^n \times \mathbb{R}^{m-n} &\longrightarrow \mathbb{R}^n \\
	(x_1,x_2) &\longmapsto x_1
\end{align*}
\end{proof}

\subsection{Fibrados vetoriais}
Um fibrado vetorial é uma coisa que generaliza os fibrados tangente e cotangente.
\begin{defn}\leavevmode
Sejam \(E, M\) variedades e \(\pi: E \to M\) submersão sobrejetiva. Dizemos que \(\pi\) é um \textit{\textbf{fibrado vetorial}} se para todo \(p \in M\), \(\pi^{-1}(p)=E_p\) possui uma estrutura de espaço vetorial tal que para todo \(p \in M\) existe \(U \ni p\) aberto e um difeomorfismo \(\varphi: \pi^{-1}(U) \to U \times \mathbb{R}^n\) tal que o seguinte diagrama comuta
\[\begin{tikzcd}
\pi^{-1}(U)\arrow[rr,"\varphi"]\arrow[dr,swap,"\pi"]&&U \times \mathbb{R}^n\arrow[dl,"\operatorname{pr}_1"]\\
&U
\end{tikzcd}\]
e
\[\varphi|_{E_p}:E_p \to \{ p\}\times \mathbb{R}^n\]
é um isomorfismo.
\end{defn}

\begin{example}\leavevmode
	\(TM,T^* M,TM \oplus  TM, TM \otimes TM, \Lambda^{k}(TM),\Lambda^{k}(T^*M),\operatorname{Sym}^k(TM)\).
\end{example}

\subsection{Seções}

\begin{defn}\leavevmode
	Uma \textit{\textbf{seção}} de \(\pi:E \to M\) é \(s:M \to E\) suave tal que \(\pi \circ s = \operatorname{id}\)
\[\begin{tikzcd}
E\arrow[d,"\pi",swap]\\
M \arrow[u, bend right,swap,"s"]
\end{tikzcd}\]
Uma seção de TM é uma função \(X: M \to TM\) tal que \(X(p) \in T_pM\), um \textit{\textbf{campo vetorial}}.
\end{defn}

\begin{thm}[da bola cabeluda]\leavevmode
\(M = S^{n}\), \(n\) par, \(X:M \to TM\) campo vetorial, então existe \(p \in M\) tal que \(X(p)=0 \in T_pM\).
\end{thm}

\begin{thing6}{Notação}\leavevmode
\(\Gamma(E) = \{ \text{seções de \(\pi:E \to M\)} \}\), \(\Gamma(TM)=\mathfrak{X}(M)\), \(\Gamma(T^*M)=\Omega^{1}(M)\), \(\Gamma(\Lambda^{k}(T^*M))=\Omega^{k}(M)\).

Para qualquer espaço vetorial \(V\),
\[\operatorname{Sym}^2(V^*)=\{f:V \times V \to \mathbb{R},\text{ bilinear, }f(x,y)=f(y,x) \}\subset V^* \otimes V^*.\]
E para fibrado vetorial \(E\),
 \[\operatorname{Sym}^2(E)=\bigsqcup_{p \in M}\operatorname{Sym}^2(E^*_p).\]
\end{thing6}

\begin{defn}\leavevmode
	Uma \textit{\textbf{métrica Riemanniana}} em \(E\) é uma seção \(s: M \to \operatorname{Sym}^2(E)\) tal que \(s(p):E_p \times E_p \to \mathbb{R}\) é positiva definida, i.e. \(s(p)(x,x)>0\) se \(x \neq  0\).
\end{defn}

\begin{remark}[Aprox.]\leavevmode
	Todo fibrado vetorial tem uma métrica Riemanniana: usando a métrica euclidiana dada em cada carta, usamos uma partição da unidade para extender a uma seção global, somar e notar que fica positiva definida.
\end{remark}

É muito fácil construir seções do fibrado cotangente: para \(f \in C^\infty(M)\), a diferencial \(df :M \to T^*M\) é uma seção do fibrado cotangente, i.e. \(df \in \Gamma(T^*M)\) porque
\[df_p=Df_p:T_pM \to T_{f(p)}\mathbb{R}\]

\begin{exercise}\leavevmode
	Qualquer seção é um mergulho de \(M\) em \(E\).
\end{exercise}

\begin{thing6}{Extra}\leavevmode
$g$ uma métrica Riemanniana em \(TM\).
\[g_p:T_pM \times T_p M \to \mathbb{R}\]

\begin{align*}
	g_p^\sharp :T_pM &\longrightarrow  T^*_pM\\
	v &\longmapsto g(v,\cdot)
\end{align*}
Então o \textit{\textbf{gradiente}} de \(f\) é
\[(g^\sharp _p)^{-1}(df_p):=\operatorname{grad}_pf\]


\end{thing6}

\end{document}
