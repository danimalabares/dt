\input{/Users/daniel/github/config/preamble-por.sty}%available at github.com/danimalabares/config
%\input{/Users/daniel/github/config/thms-por.sty}%available at github.com/danimalabares/config

\newcommand{\rightlooparrow}{\mathbin{
    \vbox{\openup-10.25pt\halign{\hss$##$\hss\cr\circ\cr\longrightarrow\cr}}
}}

\begin{document}
\bibliographystyle{alpha}

\begin{minipage}{\textwidth}
	\begin{minipage}{1\textwidth}
		Topologia diferencial\hfill Daniel González Casanova Azuela
		
		{\small Prof. Vinicius Ramos\hfill\href{https://github.com/danimalabares/k3}{github.com/danimalabares/dt}}
	\end{minipage}
\end{minipage}\vspace{.2cm}\hrule

\vspace{10pt}
{\huge Topologia Diferencial}
\tableofcontents
\section{Aula 1}

\subsection{Plano do curso, bibliografia}

Cronograma
\begin{enumerate}
	\item[0.] Revisão de variedades.
	\item Transversalidade: Sard, top. forte, fraca, aproximação.
	\item  Teoria da interseção e indice.
	\item Teoria de Morse.
	\item Tópicos adicionais (possiveis): h-cobordismo, top. de baixa dimensão, Poincaré \(n\geq 5\).
\end{enumerate}

Bibliografía: \cite{milnordt} (intuição),  \cite{gui} (tranqui, tem muito), \cite{hirsch} (pesado, tem tudo, e importante ler, usa Análise Funcional). 

\subsection{Resumo da aula 1}

\begin{enumerate}
\item Revisão de vriedades, espaço topológico, 2-enumerável, 2-contável, Hausdorff, loc. euclidiano, dimensão é fixa nas componentes conexas, def. de carta, atlas, atlas \(C^k\), atlas maximal. \textbf{Obs.} Existem atlas que não contém sub atlas \(C^k\).
\item \textbf{Teorema.} \(k=1,\ldots, +\infty\) tuda \(C^k\)-variedade é  \(C^k\)-difeomorfa a uma \(C^\infty\)-variedade.
\item \textbf{Teorema.} \(1 \leq  \ell \leq  k \leq +\infty\), se \(M,N\) são \(C^k\)-variedades, \(C^\ell\)-difeomorfas, então \(M\) e $N$ são \(C^k\)-difeomorfas. {\color{2}No será \(\ell?\)}
\item \textbf{Partições da unidade}. Definição. \textbf{Exercício:} toda variedade topológica é paracompacta. \textbf{Teorema:} \(M\) variedade \(C^\infty\) e \(\{ U_i\}\) cobertura, então existe \(C^\infty\) partição da unidade subordinada. 
\end{enumerate}


\section{Aula 2}

\subsection{Lembre}

Dada uma variedade suave \(M\). Definimos como velocidades de curvas ou como derivações: \(T_pM\) é um espaço vetorial de dimensão $n$, onde para \(p \in U\), \((U, \varphi)\) carta, \(\varphi=(x^1,\ldots,x^n\) com base \(\left\{ \frac{\partial }{\partial x_1}\Big|_{p},\ldots,\frac{\partial }{\partial x^n}\Big|_{p} \right\} \). O \textit{\textbf{espaço cotangente}} é
\[T^*_p M=(T_pM)^* =\operatorname{Hom}(T_pM,\mathbb{R}).\]
A base dual é \(\left\{ dx^1|_{p},\ldots,dx^n|_{p} \right\} \) dada por
\[ dx^i|_{p}=\left(\frac{\partial }{\partial x^j}\right)\Big|_{p}=\delta_i^j=\begin{cases}
	1\qquad &\text{se } i=j \\
	0\qquad &\text{se não} 
\end{cases}\]

e ai extendemos por linearidade a todos os demais covetores.

\begin{remark}\leavevmode
	Note que mudando de carta a gente muda de base---não tem uma base canônica do espaço cotantente.
\end{remark}

\subsection{Fórmula de mudança de bases}

\begin{thing6}{Fórmula de mudança de bases}[Exercício]\leavevmode
\((U,\varphi),(V,\psi), p \in U \cap V\), \(\varphi=(x^1,\ldots,x^n\), \(\psi(y^1,\ldots,y^n\) com bases
\[\left\{ \frac{\partial }{\partial x_1}\Big|_{p},\ldots,\frac{\partial }{\partial x^n}\Big|_{p} \right\}, \qquad \left\{ \frac{\partial }{\partial y_1}\Big|_{p},\ldots,\frac{\partial }{\partial y^n}\Big|_{p} \right\},\]
mostre que
\[\frac{\partial }{\partial x^j}=\sum_{i=1}^n \frac{\partial y^i}{\partial x^j}\frac{\partial }{\partial y^i}\]

\end{thing6}

\subsection{Fibrado tangente}
\(M\) variedade,
\[TM:=\bigsqcup_{p \in M}T_pM.\]

Note que para toda carta \((U,\varphi)\) existe uma bijeção
\begin{align*}
	\phi^{-1}:U \times \mathbb{R}^n &\longrightarrow \pi^{-1}(U) \\
	\Big(p,(v_1,\ldots,v_n) \Big) &\longmapsto \sum_{i=1}^n v_i\frac{\partial }{\partial x^i}
\end{align*}
usando essa bijeção, topologizamos \(TM\). Mas ainda, induz uma estrutura de variedade topológica com cartas dadas pelas \(\phi\). Mas exatamente, as cartas são
\begin{align*}
	\phi_{(U,\varphi)}: \pi^{-1}(U) &\longrightarrow \varphi(U)\times\mathbb{R}^n \subset \mathbb{R}^{2n} \\
	\sum v_i \frac{\partial }{\partial x^i}\Big|_{p} &\longmapsto \Big(\varphi(p),(v_i) \Big)
\end{align*}
e a mudança de coordenadas também é \(C^\infty\), i.e. esa estrutura é diferenciável.

\begin{remark}\leavevmode
	Se variedade é \(C^k\), o fibrado tangente é \(C^{k-1}\).
\end{remark}

A gente vai fazer isso mesmo com o fibrado cotangente:
\[T^*  M= \bigsqcup_{p \in M} T^*_pM.\]
O mesmo procedimento mostra que \(T^* M\) é uma \(C^\infty\)-variedade de dimensão \(2n\).

\begin{remark}\leavevmode
	Para todo \(p \in M\) existe \(U \ni p\) vizinhança tal que \(\pi_1(U) \cong U \times \mathbb{R}^n\). {\color{2}Mas \(TM \not\cong M \times \mathbb{R}^n\) em geral}; nesse caso dizemos que \(M\) é \textit{\textbf{paralelizável}}.
\end{remark}

\begin{thing6}{Casos onde \(TM \cong M \times \mathbb{R}^n\)}\leavevmode
\begin{enumerate}
\item \(M \cong \mathbb{R}^n\), \(TM \cong \mathbb{R}^n \times \mathbb{R}^n\).
\item  \(M= S^1\), \(TS^1 \cong S^1 \times \mathbb{R}\).
\item  \(M\) 3-variedade orientável, então \(TM \cong M \times \mathbb{R}^3\). (Difícil mas verdadeiro.) \textbf{Hint.} Usando quaternios não é difícil obter uma base global.
\end{enumerate}
\end{thing6}

\subsection{Imersões e mergulhos}

Até agora definimos funções suaves, mas não o que é a diferencial delas.

\begin{defn}\leavevmode
	\(M,N\) variedades suaves e \(f:M \to N\) suave. A \textit{\textbf{derivada de $f$}} é
	\[Df_p:T_pM \to T_{f(p)}N,\]
uma aplicacão linear que pode ser definida usando a definição do espaço tangente de curvas ou de derivações. Se pensamos que \(v\) é uma clase de equivalência de curvas,
\(Df_p[\gamma]=[f \circ \gamma].\)
Se \(v: C^\infty(M) \to \mathbb{R}\) é uma derivação, a definição é o pus
rward
\begin{align*}
	Df_pv: C^\infty(N) &\longrightarrow \mathbb{R} \\
	(Df_pv)g &\longmapsto v(g \circ f).
\end{align*}
Tem outra forma de definir, que usando cartas coordenadas, onde \(Df_p\) está dada como uma matriz em termos das bases locais: em cartas \((U,\varphi),(V,\psi)\) de \(p\) e \(f(p)\), \(\varphi=(x^1,\ldots,x^n)\) e \(\psi=(y^1,\ldots,y^n)\). A notação fica
\[Df_p\left(\frac{\partial }{\partial x^j}|_{p}\right) =\sum_{i=1}^n\frac{\partial f_i}{\partial x^j}|_{p}\frac{\partial }{\partial y^i}|_{f(p)}\]
onde \(\frac{\partial f_i}{\partial x^j}\) é definida como
\[D(\psi \circ f \circ \varphi^{-1})_{ij}=\frac{\partial }{\partial x^j}(\psi \circ \varphi \circ \varphi^{-1})\]
\end{defn}

\begin{defn}\leavevmode
Seja \(f:M \to N\) uma função suave. \(f\) é uma \textit{\textbf{imersão em $p$}} se a derivada \(Df_p\) é injetiva. \(f\) é uma \textit{\textbf{submersão em $p$}} se \(Df_p\) é sobrejetiva. \(f\) é um \textit{\textbf{mergulho}} se é uma imersão injetiva tom inversa \(g:f(M) \to M\) contínua.
\end{defn}

\begin{example}\leavevmode
	O exemplo mas fácil é o caso das incusões em variedades produto:
	\begin{align*}
		M &\longrightarrow M \times N \\
		p &\longmapsto (p,q)
	\end{align*}
	E as projecões:
	\begin{align*}
		M \times N &\longrightarrow M \\
		(p,q) &\longmapsto p
	\end{align*}
	Outros exemplos de submersões são as projeções dos fibrados tangente e cotangente.
\end{example}

Para ver por que na definição de mergulho pedimos que a inversa seja contínua, considere o seguinte contraexemplo: \(\mathbb{R} \to \mathbb{R}^2\) uma curva que tem um ponto límite demais: a topologia no domínio é uma linha, mas a topologia no contradomínio e de um outro espaço, mas $f$ é um mergulho injetivo! A inversa de $f$ não é contínua (não manda limites em limites).

\begin{remark}\leavevmode
	Se \(f:M \to N\) é um mergulho, então \(f(M)\) herda uma estrutura de variedade diferenciável e $f$ é um difeomorfismo entre \(M\) e \(f(M)\).
\end{remark}

\begin{upshot}\leavevmode
	Merhulo são as treis condições que precisamos para que a imagem de $f(M)$ tenha estrutura diferenciável e \(f\) um difeomorphismo entre \(M\) e \(f(M)\). O lance é usar o teorema da função inversa. \(f(M)\) é chamada de uma \textit{\textbf{subvariedade}} de $N$.
\end{upshot}
Uma definição alternativa de \textit{\textbf{subvariedade}} é que para cada ponto \(p \in  Q \subset M\), \(Q\) subespaço topológico, existe uma carta de $N$ tal que \(\varphi(U \cap Q)=\mathbb{R}^k\). (Misha's) In Misha's handouts:
\begin{thing4}{Exercise 2.23}\label{exer:2.23}\leavevmode
Let \(N_1,N_2\) be two manifolds and let \(\varphi_i:N_i\to M\) be smooth embeddings. Suppose that the image of \(N_1\) coincides with that of \(N_2\). Show that \(N_1\) and \(N_2\) are isomorphic.
\end{thing4}

\begin{thing5}{Remark 2.10}\leavevmode
By the above problem, in order to define a smooth structure on $N$, it sufficies to embed $N$ into \(\mathbb{R}^n\). As it will be clear in the next handout, every manifold is embeddable into \(\mathbb{R}^n\) (assuming it admits partition of unity). Therefore, in place of a smooth manifold, we can use ``manifolds that are smoothly embedded into \(\mathbb{R}^n\)".
\end{thing5}

\begin{thing3}{Notação}\leavevmode
Se \(f:M \to N\) é uma imersão escrevemos \(M \rightlooparrow N\), se é mergulho \(M \hookrightarrow N\) e se é submersão \(f: M \twoheadrightarrow N\).
\end{thing3}
Uma \textit{\textbf{subvariedade imersa}} é a imagem de uma imersão (que pode nem ser variedade…)

\begin{remark}\leavevmode
	\(Q \subset M\) subvariedade, então existe uma inclusão natural \(T_qQ \subset T_q M\) (linear injetiva) para todo \(q \in Q\). Claro, a derivada da inclusão \(\iota:Q \to M\), i.e. \(D\iota_q:T_qQ \to T_qM\).
\end{remark}
kj
Dado \(q \in Q\), existe \((U,\varphi)\) carta de \(M\) tal que \(\varphi|_{U \cap Q}\) é uma carta de \(Q\), é só botar a base \(\left\{\frac{\partial}{\partial x_1}\Big|_{p},\ldots,\frac{\partial}{\partial x^n}\Big|_{p}\right\}\) dentro da base de \(M\).

\subsubsection{Valores regulares}
\begin{defn}\leavevmode
	Seja \(f:M \to N\) \(C^\infty\), um ponto \(y \in N \) é dito \textit{\textbf{valor regular}} se $f$ é uma submersão em $x$ para todo \(x \in f^{-1}(y)\) i.e. \(Df_x\) é sobrejetiva para todo \(x \in f^{-1}(y)\).
\end{defn}

\begin{thm}[Do valor regular]\leavevmode
Se \(y \) é um valor regular de $f$, então \(f^{-1}(y)\) é uma subvariedade de \(M\) de dimensão \(\dim M- \dim N\). (Se \(f^{-1}(y)\neq \varnothing\).)
\end{thm}

\begin{remark}\leavevmode
	Isso é só outra encarnação do teorema da função implícita.
\end{remark}

\begin{proof}\leavevmode
\(x \in f^{-1}(y):=Q\). Pega cartas \(\varphi\) de $x$ e \(\psi\) de $y$. Supondo que \(f(U) \subset V\), e que \(x,y\) tem coordenadas 0.
\[\begin{tikzcd}
	U \subset M \arrow[r,"f"]\arrow[d,swap,"\varphi"]&  V \subset N\arrow[d, "\psi"]\\
	\mathbb{R}^m \arrow[ r, swap,"\Phi:\psi \circ f \circ \varphi^{-1}"]& \mathbb{R}^n
\end{tikzcd}\]
Note que \(\Phi(0)=0\) e que \(\Phi^{-1}(0)=\varphi(f^{-1}(y) \cap U)\).
\begin{claim}\leavevmode
	\(\Phi^{-1}(0)\) é uma subvariedade.
\end{claim}
Para tudo ficar claro vamos reescrever o teorema de função implícita.  \(\Phi'(0)\) é sobrejetiva. Temos que
\begin{align*}
	: \mathbb{R}^m &\longrightarrow \mathbb{R}^n \times \mathbb{R}^{m-n} \\
	z &\longmapsto \Phi(z)
\end{align*}
A ideia é que existe uma vizinhança \(W\) de \(0 \in \mathbb{R}^m\) e um difeomorfismo \(\eta:W \to W^{\smile}\) tal que
\begin{align*}
	\phi \circ \eta: W \subset \mathbb{R}^n \times \mathbb{R}^{m-n} &\longrightarrow \mathbb{R}^n \\
	(x_1,x_2) &\longmapsto x_1
\end{align*}
\end{proof}

\subsection{Fibrados vetoriais}
Um fibrado vetorial é uma coisa que generaliza os fibrados tangente e cotangente.
\begin{defn}\leavevmode
Sejam \(E, M\) variedades e \(\pi: E \to M\) submersão sobrejetiva. Dizemos que \(\pi\) é um \textit{\textbf{fibrado vetorial}} se para todo \(p \in M\), \(\pi^{-1}(p)=E_p\) possui uma estrutura de espaço vetorial tal que para todo \(p \in M\) existe \(U \ni p\) aberto e um difeomorfismo \(\varphi: \pi^{-1}(U) \to U \times \mathbb{R}^n\) tal que o seguinte diagrama comuta
\[\begin{tikzcd}
\pi^{-1}(U)\arrow[rr,"\varphi"]\arrow[dr,swap,"\pi"]&&U \times \mathbb{R}^n\arrow[dl,"\operatorname{pr}_1"]\\
&U
\end{tikzcd}\]
e
\[\varphi|_{E_p}:E_p \to \{ p\}\times \mathbb{R}^n\]
é um isomorfismo.
\end{defn}

\begin{example}\leavevmode
	\(TM,T^* M,TM \oplus  TM, TM \otimes TM, \Lambda^{k}(TM),\Lambda^{k}(T^*M),\operatorname{Sym}^k(TM)\).
\end{example}

\subsection{Seções}

\begin{defn}\leavevmode
	Uma \textit{\textbf{seção}} de \(\pi:E \to M\) é \(s:M \to E\) suave tal que \(\pi \circ s = \operatorname{id}\)
\[\begin{tikzcd}
E\arrow[d,"\pi",swap]\\
M \arrow[u, bend right,swap,"s"]
\end{tikzcd}\]
Uma seção de TM é uma função \(X: M \to TM\) tal que \(X(p) \in T_pM\), um \textit{\textbf{campo vetorial}}.
\end{defn}

\begin{thm}[da bola cabeluda]\leavevmode
\(M = S^{n}\), \(n\) par, \(X:M \to TM\) campo vetorial, então existe \(p \in M\) tal que \(X(p)=0 \in T_pM\).
\end{thm}

\begin{thing6}{Notação}\leavevmode
\(\Gamma(E) = \{ \text{seções de \(\pi:E \to M\)} \}\), \(\Gamma(TM)=\mathfrak{X}(M)\), \(\Gamma(T^*M)=\Omega^{1}(M)\), \(\Gamma(\Lambda^{k}(T^*M))=\Omega^{k}(M)\).

Para qualquer espaço vetorial \(V\),
\[\operatorname{Sym}^2(V^*)=\{f:V \times V \to \mathbb{R},\text{ bilinear, }f(x,y)=f(y,x) \}\subset V^* \otimes V^*.\]
E para fibrado vetorial \(E\),
 \[\operatorname{Sym}^2(E)=\bigsqcup_{p \in M}\operatorname{Sym}^2(E^*_p).\]
\end{thing6}

\begin{defn}\leavevmode
	Uma \textit{\textbf{métrica Riemanniana}} em \(E\) é uma seção \(s: M \to \operatorname{Sym}^2(E)\) tal que \(s(p):E_p \times E_p \to \mathbb{R}\) é positiva definida, i.e. \(s(p)(x,x)>0\) se \(x \neq  0\).
\end{defn}

\begin{remark}[Aprox.]\leavevmode
	Todo fibrado vetorial tem uma métrica Riemanniana: usando a métrica euclidiana dada em cada carta, usamos uma partição da unidade para extender a uma seção global, somar e notar que fica positiva definida.
\end{remark}

É muito fácil construir seções do fibrado cotangente: para \(f \in C^\infty(M)\), a diferencial \(df :M \to T^*M\) é uma seção do fibrado cotangente, i.e. \(df \in \Gamma(T^*M)\) porque
\[df_p=Df_p:T_pM \to T_{f(p)}\mathbb{R}\]

\begin{exercise}\leavevmode
	Qualquer seção é um mergulho de \(M\) em \(E\).
\end{exercise}

\begin{thing6}{Mais uma}\leavevmode
$g$ uma métrica Riemanniana em \(TM\).
\[g_p:T_pM \times T_p M \to \mathbb{R}\]

\begin{align*}
	g_p^\sharp :T_pM &\longrightarrow  T^*_pM\\
	v &\longmapsto g(v,\cdot)
\end{align*}
Então o \textit{\textbf{gradiente}} de \(f\) é
\[(g^\sharp _p)^{-1}(df_p):=\operatorname{grad}_pf\]
\end{thing6}

\section{Aula 3: Teorema de Sard}

\subsection*{Teorema da função implícita (aula pasada)}

\begin{upshot}\leavevmode
Think of the circle as the zero-set of \(f(x,y)=x^2+y^2-1\). If the Jacobian matrix of $f$ has enough rank at a given point \((x,y)\), we can find a function \(g: \mathbb{R} \to \mathbb{R}\) that parametrizes the circle locally. That is: \(\Big(x,g(x)\Big)\) is in the circle, i.e. \(f\Big(x,g(x) \Big)=0\). (So the zero-set is the graph of \(g\)).

So you have the zero-set of a function at a point, which is a subset of the domain. It is a geometric object. You want to parametrize it. If the derivative has rank  \(k\), you can put \(k\) of the variables of the domain in terms of the remaining ones.

So the real upshot is that this geometric thing is truly determined by \(k\) variables, so we don't need the other variables. That's it. The theorem says how to get rid of the excess of info locally: \(g\) needs less variables and produces the geometric thing locally.
\end{upshot}

Uma função \(f:\mathbb{R}^n \to \mathbb{R}^m\) suave tal que \(f(0)=0\) e  \(f'(0)\) é sobrejetiva (\(\implies n \geq m\)). Então existe uma vizinhança de \(0 \in \mathbb{R}^n\) \(U\) e \(\tilde{U}\) \textbf{e um difeomorfismo} \(\varphi: U \to \tilde{U}\) tal que
\begin{align*}
	f \circ \varphi: \mathbb{R}^m \times \mathbb{R}^{n-m} &\longrightarrow \mathbb{R}^m \\
	(x,y) &\longmapsto x
\end{align*}

\begin{proof}\leavevmode
Parecido à prova de \cite{tus} no teorema do valor regular, usando uma matrix com um \(*\), a identidade, e uma matriz invertível.
\end{proof}

\subsection{Transversalidade: Teorema de Sard}

A prova do teorema de Sard é muito técnica. Porém, a parte difícil é só análise em \(\mathbb{R}^n\).

Pegue \(a=(a_1,\ldots,a_n),b=(b_1,\ldots,b_n) \in \mathbb{R}^n\), defina um \textit{\textbf{cubo}} como sendo
\[c(a,b)=\prod_{i=1}^n ]a_i,b_i[ \subset \mathbb{R}^n.\]
Note que \(\operatorname{Vol}(a,b)=\prod_{i=1}^n(b_i-a_i).\)

\begin{thing5}{What \cite{lee} does}\leavevmode
\begin{enumerate}
\item A compact subset whose intersection with every hyperplane has measure zero has measure zero.
\item Graph of continuous function has measure zero.
\item Affine subspaces of \(\mathbb{R}^n\) have measure zero.
\item Smooth map from manifold to \(\mathbb{R}^n\) maps measure zero to measure zero.
\item A set in a manifold has \textit{\textbf{measure zero}} if it(s intersection with the respective domain) is mapped to a set of measure zero by any chart.
\item Confusing lemma.
\item Complement of zero measure is dense (in manifolds).
\item Smooth map \textit{of manifolds} maps measure zero to measure zero.
\item Sard's theorem (heavy proof): critical value set of smooth map has measure zero.
\item Corollary \textbf{(minisard)}: image of smaller dimension manifolds under smooth map has measure zero {\color{6}(Esto fue aclarado en clase: ¿por qué nosotros probamos minisard antes de sard? nuestra definición de punto crítico no permite aplicar este corolario; lo probamos de otra forma)}. Corollary 2: smaller dimension immersed submanifolds have measure zero.
\item Up next: Whitney embedding theorem.
\end{enumerate}
\end{thing5}

\begin{defn}\leavevmode
	\begin{figure}[H]
		\centering
		\includegraphics[width=0.7\textwidth]{fig1}
	\end{figure}
\(S \subset \mathbb{R}^n\) possui \textit{\textbf{medida nula}} se \(\forall  \varepsilon >0\) existe \(\{c_i\}_{i=1}^\infty\) cubos (ou bolas) tais que \[S \subset \bigcup_{i=1}^\infty \operatorname{Vol}(c_i)<\varepsilon\]
\end{defn}

\begin{prop}\leavevmode
	\begin{enumerate}
	\item Uma união enumerável de conjuntos de medida nula tem medida nula.
	\item \(f : \mathbb{R}^n \to \mathbb{R}^n\) \(C^1\) e \(S \subset \mathbb{R}^n\) tem medida nula, então \(f(S)\) tem medida nula.
	\end{enumerate}
\end{prop}

\begin{proof}\leavevmode
\begin{enumerate}
\item \(\{S_i\}\) enumerável de medida nula, para cada $i$ você pode escolher cubos \(C^i_1,C^i_2,\ldots\) que cobren \(S_i\) e tal que a soma dos volumes deles é menor do que \(\sum_j \operatorname{Vol}(C^i_j)<\frac{\varepsilon}{2^i}\). Vai ver que a soma dos volumeis variando tanto \(i\) como \(j\) da \(\varepsilon\).

\item (Foto)
\end{enumerate}
\end{proof}

\begin{defn}\leavevmode
	\(X\) variedade diferenciável. \(S \subset X\). Dizemos que \(S\) tem \textit{\textbf{medida nula}} se \(\exists \{U_i\}_{i=1}^\infty\) cobertura aberta de \(S\), i.e. \(\bigcup_{i=1}^\infty S_i \supset S\), e cartas \(\varphi_i:U_i\to \mathbb{R}\) e \(S_i \subset U_i\) e \(\varphi(S)\) tem medida nula.

	O más bien: sólo el chiste es que cada conjunto tiene medida en \(\mathbb{R}^n\) cuando proyectas con cualquier carta.
\end{defn}

\begin{coro}\leavevmode
	\begin{enumerate}
	\item \(\{S_i\}_{i=1}^\infty\).  \(S_i \subset X\) medida nula, entao \(\bigcup_{i \in \mathbb{N}}S_i\) tem medida nula.
	\item \(X^n, Y^n\) variedades, \(f:X \to Y\) suave, \(S \subset X\) medida nula. Então \(f(S)\) tem medida nula.
	\end{enumerate}
\end{coro}

\begin{prop}\leavevmode
\(Y^n\) variedade, \(X^m \subset Y^n\) subvariedade de dimensão \(m <n\). Então \(X\) tem medida nula.
\end{prop}

\begin{proof}\leavevmode
É simplesmente levar para \(\mathbb{R}^n\): considera \(X_i\) como a parte de \(X\) que está den'de cada  \(U_i\) no atlas de \(Y\) e vai ver que ele tem dimensão menor. Daí é só provar que subespaços (acho que lineares) de dimensão menor em \(\mathbb{R}^n\) tem dimensão menor.
\end{proof}

\begin{coro}[Minisard]\leavevmode
\(X^m,Y^n\) variedades \(m<n\)  e \(f:X \to Y\) suave. Então \(f(X)\) tem medida nula.
\end{coro}

\begin{proof}\leavevmode
Aqui se usa o corolário: usar a inclusão \(\iota:X \to X \times \mathbb{R}^{n-m},x \mapsto (x,0)\), compor com \(\tilde{f}:X \times R^{n-m}\to Y\), \((x,y) \mapsto  f(x)\). Então \(\tilde{f}(i(X))=f(X)\). O lance é que \(\iota(X)\) é uma subvariedade de codimensão positiva, então pela prop anterior tem medida nula. Daí $f(X)$ também.
\end{proof}

\begin{coro}[Versão fácil do teorema de mergulho de Whitney]\leavevmode
	Se \(X^n\) variedade diferenciável compacta, então existem 
	\[X \hookrightarrow \mathbb{R}^{2n+1},\qquad  X\rightlooparrow \mathbb{R}^{2n}\]
\end{coro}

\begin{thm}[Difícil de Sard]\leavevmode
\[X \hookrightarrow \mathbb{R}^{2n},\qquad  X\rightlooparrow \mathbb{R}^{2n-1}\]
\end{thm}

\begin{proof}[Prova do corolário]\leavevmode
\begin{enumerate}[label=\textbf{Step \arabic*}]
\item Mergulhar a variedade num espaço euclidiano \textit{grande}. Pegue um atlas finito \(\{(U_i,\varphi_i)_{i=1}^k\}\), note que \(\varphi_i:U_i \to \mathbb{R}^n\) são mergulhos.

\begin{thing6}{Ideia}\leavevmode
\begin{align*}
	\Phi: X &\longrightarrow \mathbb{R}^n\times \mathbb{R}^n \times\ldots\times \mathbb{R}^n \subset \mathbb{R}^{nk} \\
	p &\longmapsto (\varphi_1(p),\varphi_2(p),\ldots
\end{align*}
Isso não da. Para fazer bem precisamos de uma partição da unidade \(\{\rho_i\}_{i=1}^k\) subordinada a \(\{U_i\}_{i=1}^k\) sobertura. Defina \(\rho_i\varphi_i:X \to \mathbb{R}^n\) como sendo zero fora do conjunto bom; note que essa função não é mais um mergulho, mas tudo bem. Agora faça \(X \to (\mathbb{R}^n)^k\times \mathbb{R}^k=\mathbb{R}^{nk+k}\)
\begin{align*}
	\Phi: X &\longrightarrow (\mathbb{R}^n)^k \times\mathbb{R}^k = \mathbb{R}^{nk+k} \\
	p &\longmapsto \Big((\rho_1 \varphi_1)(p),\ldots,\Big(\rho_k \varphi_k)(p)\Big)
\end{align*}
\begin{exercise}[Importante]\leavevmode
	Mostre que \(\Phi\) é uma imersão injetiva.
\end{exercise}
\end{thing6}

\item \textbf{Afirmação:}
	\[X \hookrightarrow  \mathbb{R}^n \implies \begin{cases}
		X \hookrightarrow \mathbb{R}^{N-1}\qquad &\text{ se $N>2n+1$}  \\
		X \rightlooparrow \mathbb{R}^{N-1} \qquad &\text{se $N>2n$.} 
	\end{cases}\]
\begin{proof}[Prova da afirmação]\leavevmode
Vamos projetar a variedade mergulhada em \(\mathbb{R}^n\) no plano ortogonal a algum vetor \(a \in \mathbb{R}^n\). Resulta que
\begin{exercise}\leavevmode
	\begin{align*}
	g: X \times X \times \mathbb{R} &\longrightarrow \mathbb{R}^N \\
	(x,y,t) &\longmapsto \operatorname{pr}_a \circ f
\end{align*}
é injetiva.
\end{exercise}
\end{proof}

\item \textbf{Ideia:} ver que em quase todo ponto podemos projetar.

Considere agora o mapa pusforward que pega um vetor tangente e manda mediante $f$:
\begin{align*}
	h: TX &\longrightarrow \mathbb{R}^N \\
	(x,v) &\longmapsto (Df)_xv
\end{align*}
Agora note que
\begin{thing4}{Afirmação}\leavevmode
\(a \not\in \operatorname{Im}(h) \iff \operatorname{pr}_a \circ f\) é uma imersão \(\iff\) \(D(\operatorname{pr}_a \circ f)_a\) é injetiva para toda $x$.
\end{thing4}

\item A prova termina usando minisard: as imagens de $g$ e de \(h\) tem medida nula. Mesmo a união delas. Então existe um ponto fora dessa união.
\end{enumerate}
\end{proof}

\begin{defn}\leavevmode
Sejam \(X^m, Y^k\) variedades, \(f:X \to Y\) suave, dizemos que
\begin{enumerate}[label=(\alph*)]
	\item \(x \in X\) é \textit{\textbf{ponto crítico}} se o posto de \(Df_x\) é menor do que \(\operatorname{min}(m,n)\). (\(\iff\)não é surjetiva I think) {\color{6}Aula 7: essa definição é que a derivada não é de posto máximo. Isso permete que o domínio tenha pontos regulares, así fez Sard e  \cite{gui2}, mas não \cite{lee}, \cite{gui}.}
\item \(x \in X\) é \textit{\textbf{ponto regular}} se posto \(Df_x=\operatorname{min}(m,n)\).
\item \(y \in Y\) é \textit{\textbf{valor crítico}} se existe um ponto crítico tal que \(f(x) = y\).
\item \(y \in Y\) é \textit{\textbf{valor regular}} se \(\forall x \in f^{-1}(y)\), \(x\) é valor regular.
\end{enumerate}
\end{defn}

\begin{thm}[Sard]\leavevmode
\(f:X \to Y\) suave. Então \(\{\text{valores críticos} \}\) tem medida nula.
\end{thm}

\begin{remark}\leavevmode
\begin{enumerate}
\item Teorema vale se \(f\) é \(C^\ell\), onde \(\ell>\operatorname{max}(m-n,0)\). 

	. \end{enumerate}
\end{remark}

\begin{proof}\leavevmode
\begin{enumerate}[label=\textbf{Step \arabic*}]
\item \textbf{Redução para a versão local.} Supomos que \(X= \mathbb{R}^{m}, Y=\mathbb{R}^n\). \(f:U \subset \mathbb{R}^m \to \mathbb{R}^n\), \(U\) aberto.
	\[\operatorname{Crit}f= \{x \in 0:\text{posto } f'(x) < \operatorname{min}(m,n)\}\]
Então \(f(\operatorname{Crit}(f)\) tem medida nula. Para isso fazemos \textbf{indução em $m$.} \(m=0\) trivial.

\(C_i\) vai ser o conjunto onde as derivadas parciais se anulam até $i$:
 \[C_i=\left\{ p \in U: \frac{\partial^{(\alpha)}}{\partial x^{\alpha}}f_k(p)=0 \forall \alpha, 0<| \alpha|\leq 1, \forall  k \right\}.\]
 Note que \(C_{i+1}\subset C_i \subset C_{i-1}\subset\ldots C_1\subset C:=\operatorname{Crit}f\).

 \begin{thing7}{Objetivo}\leavevmode
 \(f(C)\) tem medida nula.
 \begin{enumerate}[label=\textbf{Paso \arabic*}]
 \item \(f(C_N)\) tem medida nula para algum  \(N \gg 0\). {\color{8}Crucial}
\item \(f(C_i \setminus C_{i+1}\) tem medida nula para toda $i$.
\item \(f(C\setminus C_i\) tem medida nula.
 \end{enumerate}

\begin{enumerate}[label=\textbf{Paso \arabic*}]
\item Podemos supor sem perda de generalidade que \(U\subset\)cubo, a fórmula de Taylor diz que
	\[\|f(x)-f(y)\|_\infty \leq K \cdot \|x-y\|_\infty^{i+1}\]
para todo \(x, y \in C_i\).

Tem que botar \(C_i\) den'de um cubo \(D_j\) que se divide em \(r^m\) cubos de lado \(b/r\). Então  \(f(D_j)\) está contido num cubo em \(\mathbb{R}^n\) de lado \(K \cdot \left(\frac{b}{r}\right)^{i+1}:=R_j \). Também note que pontos den'de \(D_j\) são tq. \(\|x-y\|_\infty \leq  \frac{b}{r}\).

Agora
\[f(C_i) \subset f\left(\bigcup_{j=1}^{r^m}D_j\right) \subset \bigcup_{j=1}^{r^m}f(D_j)\subset \bigcup_{j=1}^{r^m}R_j.\]
Então
\begin{align*}
\sum_{j=1}^{r^m}\operatorname{Vol}(R_j)&=r^m\cdot K^n\cdot \left(\frac{b}{r}\right)^{(i+1)\cdot n}\\
&= \frac{K^n \cdot b^{n(N+1)}}{r^{n(N+1)-m}}
\end{align*}
\end{enumerate}
 \end{thing7}

\item 
\end{enumerate}

\end{proof}

\section{Aula 4}

\subsection{Teorema de Sard}

\begin{thm}[Sard]\leavevmode
\(f: \mathbb{R}^m \to \mathbb{R}^n\) \(C^\ell\), \(\ell >  \operatorname{max}(m-n,0)\). Então \(\{\text{valores críticos} \}\) tem medida nula.
\end{thm}

\begin{proof}\leavevmode
\textbf{Note que} \(\{\text{valores críticos} \}=f(\{\text{ptos críticos} \})\).

Seja \(C=\{\text{ptos críticos de $f$} \}\). Então aproximamos a conjunto onde todas as derivadas parcias são zero com o conjunto \(C_i\) onde as derivadas parciais até \(i\) se anulam. 
\begin{enumerate}[label=\textbf{Passo \arabic*}]
\item \(f(C_N)\) tem medida nula se \(N>\operatorname{max}(m-n,0)\). {\color{6}(Feito na aula pasada.)}
\item \(f(C_i\setminus C_{i+1}\) tem medida nula
\item \(f(C\setminus C_1\) tem medida nula.
\end{enumerate}
Concluimos porque \(f(C)\) é a união de treis conjuntos de medida nula: um por cada passo. Segundo e terceiro passos são com indução em $m$.

Prova:
\begin{enumerate}[label=\textbf{Passo \arabic*}]
\item Feito ontem.
\item A ideia é que podemos dar coordenadas de dimensão 1 menos usando que a derivada \(i+1\) não se anula. (Acho.)
\item É parecido só que um pouco mas dificil. No caso anterior os valores da função \(h\) são zero, aqui não (ver foto). Aqui usamos
	\begin{lemma}\leavevmode
A compact subset whose intersection with every hyperplane has measure zero has measure zero:

\(A \subset\mathbb{R}^n\) compacto tal que \(X \cap \{ x \}\times \mathbb{R}^{n-1}\) tem medida nula em \(\mathbb{R}^{n-1}\) para toda \(x \in \mathbb{R}\). Então \(A\) tem medida nula.
	\end{lemma}
	\begin{proof}[Prova do lema]\leavevmode
	A ideia es pegar uma faixinha de altura $x$ e cobrir esse pedaço de \(A\) com quadradinhos naquele plano \(C^x_j\). Dai, ``como \(A\) é compacto" podemos pegar um \(I_x \subset \mathbb{R}\) intervalo tal que para todo \(y \in I_x\) ($y$ perto de \(x\)), a faixinha de altura $y$ fique contida em \(\bigcup I_x \times C_j^x\)

	{\color{4}\bfseries Ideia.}\hspace{.5em} Como \(A\) é compacto podemos pegar um mini intervalo tal que todas as faixinhas muito pertinho (bom, a parte de \(A\) em cada faixinha) fica den'dos quadrados \(C^x_j\) multiplicados por esse mini-intervalo.

	Agora calculamos os vulmeis. Lembre de análise na reta (ver \cite{lee} lem 6.2, tem que shrink os intervalos) que a soma dos comprimentos dos intervalos \(I_{x_i}\) que conformam uma cobertura esencial (não pode tirar nenhum dos abertos da coberta) de um intervalo \(L\) \textbf{é menor do que duas vezes o tamanho do intervalo}:  \(\sum \operatorname{compr}(I_{x_i})< 2(2L)=4L\).

	Em fim, a soma dos comprimentos é um número finito. Então fica que
	\begin{align*}
	\sum_{i,j}\operatorname{Vol}(I_{x_i} \times C_j^{x_i}&= \sum_i \sum_j \operatorname{Vol}_1(I_x) \operatorname{Vol}_{n-1}(C_j^{x_i})<\varepsilon \sum \operatorname{Vol}_1(I_{x_i}) < 4L\varepsilon.
	\end{align*}
	\end{proof}
\end{enumerate}
\end{proof}

\subsection{Espaço de jatos}

Son como vectores de orden de diferenciabilidad más grande: a ideia é generalizar o espaço tangente e o espaço cotangente \textbf{para derivadas de ordem maior}.

\begin{defn}\leavevmode

	Dos funciones \(f,g:X\to Y\) suaves que mandan \(p\) al mismo punto son equivalentes si existem cartas tales que las derivadas parciales de sus representaciones en coordenadas coinciden hasta orden \(k\).
\end{defn}

Sejam \(X, Y\) variedades diferenciaveis suaves e \(f,g:X \to Y\) suaves. Dizmos que \(f \sim_k g\) em \(p \in X\) se, intuitivamente, as derivadas parciais de \(f \) e \(g\) coincidem até ordem $k$. Isso é inuitivo porque precisamos pegar cartas para isso ficar bem definido: precisamos que existam cartas \((U, \varphi),(V,\psi)\) aoredor de \(p\) e \(f(p)\) tais que
\[\frac{\partial ^{|\alpha|}}{\partial x^\alpha}(\psi \circ f \circ \varphi^{-1}(\varphi(p))=\frac{\partial ^{| \alpha|}}{\partial x^\alpha}(\]
Los \textit{\textbf{jatos}} son gérmenes:
\[J^k(X,Y)_{p,q}=\{f:X \to Y: f(p)=q\}\Big/ \sim_k.\]
Isso generaliza o espaço tangente do seguinte jeito:
\[J^1(\mathbb{R},Y)_{0,q} \cong T_qY.\]
\begin{exercise}\leavevmode
\[J^1(X,\mathbb{R})_{p,0}\cong T_p^*Y.\]
\end{exercise}

\begin{proof}[Solution]\leavevmode
Vamos definir uma correspondência que a cada jato associa um funcional em \(T^*_pM\) pensando que os vetores são classes de equivalência de curvas \([\gamma]\). Pegue um jato \(\sigma\). Esse jato \(\sigma\) tem um representante \(f:X \to \mathbb{R}\) (que manda $p$ a zero). Pegue \(d_pf:T_pX \to \mathbb{R}\), um elemento de \(T_p^* X\). É claro que esta correspondência está bem definida: se \(g : X \to \mathbb{R}\) está relacionada com \( f\), as derivadas parciais delas coincidem em $p$, de forma que \(d_pf=d_pg\). A contenção oposta é evidente: toda forma em \(T_p^*X\) é a diferencial de algum jato já que a base canônica de \(T^*_pX\) está dada por diferenciais de funções em $p$.
\end{proof}
Daí definimos o \textit{\textbf{espaço de $k$-jatos}}:
\[J^k(X,Y):= \bigsqcup_{\substack{p \in X \\ q \in Y}}J^k(X,Y)_{p,q}\]

Então pega um jato \(\sigma \in J^k(X,Y)\). Isso cuspe um \(p\) e um  \(q\) tais que  \(\sigma \in J^k(X,Y)_{p,q}\). Definamos as funções
\[\begin{aligned}
	\alpha: J^k(X,Y) &\longrightarrow X \\
	\sigma &\longmapsto p
\end{aligned}\qquad \qquad \begin{aligned}
	\beta: J^k(X,Y) &\longrightarrow Y \\
	0 &\longmapsto q
\end{aligned}\]

\begin{example}\leavevmode
\(X=U \subset \mathbb{R}^n\), \(Y= V \subset \mathbb{R}^m\) abertos. O que é o espaço de jatos neste caso?

Tem uma bijeção 
\begin{align*}
	: J^k(U,V)_{x,y} &\xrightarrow{\cong}B_{n,m}^k \\
	f &\longmapsto (f_1^k,\ldots,f_m^k)
\end{align*}
Lance: pode pensar que esas funções são polinomias de grau maximo $k$.
\[B^k_{n,m}=\{p: \mathbb{R}^n\to \mathbb{R}^m: p \text{ polinomial de grau \(\leq k\) tal que \(p(0)=0\)} \}.\]

\begin{exercise}\leavevmode
Calcule a dimensão de \(B^k_{n,m}\).
\end{exercise}

\begin{proof}[Solution]\leavevmode
Enquanto eu achei que a dimensão era \(k^m\), porque cada espaço de polinomios (\textit{em uma variável}, sem termo constante) está generado por \(\left<x,x^2,\ldots,x^k\right>\) e temos $m$ deles, errei. Porque são polinomios em $n$ variaveis. Então a base do espaço de polinomios em $n$ variáveis de grau máximo \(k\) e sem termo constante e… o conjunto de monomios linearmente independentes de grau máximo \(k\)

Façamos um polinomio de $n$ variáveis de grau máximo \(k\). Tendo $n$ variáveis, pode pegar só uma delas, ai tem $n$ opções. Pode pegar duas delas, ai tem mais \(n^2\) opções. Tres, quatro, até \(k\) delas. Tem \(\sum_{i=1}^k n^i\) diferentes monomios. {\color{2}Mas…} o que acontece com a ordem?

De acordo com ChatGPT, o problema fica melhor quando pensamos que a cada variável (=indeterminada \(x_i \in k[x_1,\ldots,x_n]\)) de um monómio associamos um número inteiro maior o igual de que 0, o exponente da variável. Daí a quantidade de monomios de grau \(i\) é simplesmente a quantidade de vetores inteiros não negativos \(a_j\) tais que \(\sum a_j=i\). Isso é o mesmo que a quantidade de formas de acomodar \(i\) bolinhas em $n$ caixinhas. Isso é o mesmo que a quantidade de formas de acomodar \(n-1\) divisionsinhas numa lista de \(i\) bolinhas (porque todas as bolinhas são iguais). Então são \(i + n-1\) coisas (divisionsinhas ou bolinhas) numa linha, e nesses \(i + n-1 \) lugares escolhemos \(n-1\) para botar as divisionsinhas. O resultado é \(\binom{i+n-1}{n-1}\). Somar sobre \(i\). Elevar à $m$.
\end{proof}
\[\begin{tikzcd}
J^k(U,V)\arrow[rr,"\cong"]\arrow[dr,"\alpha",swap]&&U \times V \times B^k_{n,m}\arrow[dl,"\operatorname{pr}_1"]\\
&U
\end{tikzcd}\]


Creo que: definimos \(f^k_i\) como as "partes sem constante dos polinómios de Taylor de ordem $k$ das coordenadas de $f$",
{\color{7}CREO QUE la idea es que la clase de equivalencia \([f]\)  está determinada por los principios de los polinomios de Taylor de sus funciones coordenadas.}

No entendí esto pero va:
\begin{align*}
	p: \mathbb{R}^n &\longrightarrow \mathbb{R}^m \qquad \text{polinomial}
\end{align*}
\(x_0 \in U, y_0 \in V\), entre aspas:
\["f(x-x_0) = y_0+p(x-x_0),\]
\(f(U) \subset V.\)
En fim, temos que
\begin{enumerate}
	\item[2.] \(J^1(M,\mathbb{R}) \cong \mathbb{R} \times T^*M\).
	\item[3.] \(J^1(\mathbb{R},M) \cong \mathbb{R} \times TM.\)
\end{enumerate}
\end{example}
Agora o pushforward e o pullback, que basicamente é precompor e poscompor:
\begin{defn}\leavevmode
\begin{enumerate}
\item \(\varphi:Y \to Z\) suave, \(X \) variedade suave. O \textit{\textbf{pushforward}} é
	\begin{align*}
		\varphi_*: J^k(X,Y) &\longrightarrow J^k(X,Z) \\
		[f]_x &\longmapsto [\varphi \circ f]_x
	\end{align*}
\item O  \textit{\textbf{pullback}} é… mas aqui \textbf{precisamos que  \(\psi\) seja difeomorfismo} 
	\begin{align*}
		\psi^*: J^k(X,Y) &\longrightarrow J^k(Z,Y) \\
		[f]_x &\longmapsto [f \circ \psi]_{\psi(x)}
	\end{align*}
\end{enumerate}
\end{defn}

\begin{remark}\leavevmode
\begin{enumerate}
\item \(\sigma \in J^k(X,Y)_{x,y}\), \(\varphi_* \sigma \in J^k(X,Z)_{x,\varphi(y)}\)
\item \(\sigma \in J^k(X,Y)\), \(\psi ^*\sigma \in J^k(Z,Y)_{\psi^{-1}(x),y}\)
\end{enumerate}
\end{remark}

\subsubsection{Estrutura diferenciável no espaço de jatos}

\subsubsection{When dani finally understood this}

Here's how the charts (and the topology) of \(J^k(X,Y)\) is constructed.
\begin{enumerate}
\item Choose charts \((U,\varphi)\) of \(X\) and \((V,\psi)\) of \(Y\).

 \item Take the jet space \(J^{k}(U,V)\) to an euclidean jet space via the induced maps of the charts:
	 \[(\psi^{-1})^*\varphi*:J^{k}(U,V)\to J^{k}(U',V')\]
	where \(U'=\varphi(U)\) and \(V'=\psi(V)\).

\item Now that you are in euclidean space you can compute Taylor polynomials: a  \(k\)-jet \([f]\in J^{k}(U',V')\) has coordinate functions, the Taylor polynomials of which are well-defined up to order \(k\):
	\begin{align*}
		f: U' \subset \mathbb{R}^n &\longrightarrow V' \subset \mathbb{R}^m \\
		x_0 &\longmapsto \Big(f_1(x_0),\ldots,f_m(x_0) \Big)
	\end{align*}
Define
\[T_kf_i(x_0):=\text{Taylor polynomial of \(f_i\) at \(x_0\) up to degree \(k\)}. \]
\item Realise: for every \(k\)-jet \([f] \in J^{k}(U',V')\) its coordinates are:
	\begin{align*}
	\Big(\alpha[f],\beta[f],T_kf_1(\alpha[f]),\ldots,T_kf_m(\alpha[f]) \Big)&\in U' \times V'\times P^k_{n}\times\ldots\times P^k_n\\
	&=U' \times V' \times B^k_{n,m}.
	\end{align*}
	where \(\alpha[f]\) is the source, \(\beta[f]\) its target.

	 
\end{enumerate}

\subsubsection{Lecture notes}

Pegue \(\sigma \in J^k(X,Y)_{p,q}\) e cartas \((U,\varphi)\) de \(p\) e \((V,\psi)\) de $q$. {\color{4}Ideia:} usar o pushforward e o pullback das cartas para levar o problema no \(\mathbb{R}^n\).

\begin{exercise}\leavevmode
Considere
\[J^k(U,V)= \bigsqcup_{\substack{p \in U \\ q \in V}}J^k(X,Y)_{p,q}.\]
Então
\begin{align*}
	J^k(U,V) &\longrightarrow J^k(\varphi(U),\psi(V)) \\
	\sigma &\longmapsto \psi_*(\varphi^{-1})^*\sigma
\end{align*}
é uma bijeção.
\end{exercise}
Então para dar uma estrutura de variedade topológica no espaço de jatos note que também
\[J^k(\varphi(0),\varphi(V))\cong \varphi(0) \times \varphi(V) \times B^k_{n,m}\subset \mathbb{R}^{n+m+ \dim B^k_{n,m}}\]
(lo bueno es que ya sabes cual es la dimension de \(B^k_{n,m}\). Mas não interessa qual é a dimensão: o importante é que o \(B^k_{n,m}\) tem uma base, é um espaço vetorial.)
)
Em fim, tudo isso da uma estrutura de variedade topologica. Para terminhar só temos que ver o que acontece com as mudanças de coordenadas.

\begin{align*}
	\varphi(U) \times \psi(V) \times B^k_{n,m}  &\longrightarrow \tilde{\varphi}(\tilde{U})\times \tilde{\psi}(\tilde{V}) \times B^k_{n,m} \\
	(p,q,f) &\longmapsto \Big(\tilde{\varphi} \circ \varphi^{-1}(p),\tilde{\psi} \circ \psi^{-1}(q) \Big)
\end{align*}
\textbf{Isso é suave!} E isso implica que \(J^k(X,Y)\) é uma \(C^\infty\) variedade de dimensão \(n+m+ \dim B^k_{n,m}\). 


\section{Aula 5: topologia de Whitney}

\begin{enumerate}
\item Terminhamos a estrutura de variedade diferenciavel do espaço de jatos. As cartas são
	\[J^k(\varphi(U),\psi(V)) \xrightarrow{\text{ bijeção} }\varphi(U) \times \psi(U) \times B^k_{n,m}\subset \mathbb{R}^N\]
	lembramos como ver que as funções de transição são suaves {\color{2}(exercício: fazer detalhes)}
\item
	\begin{defn}\leavevmode
	Sejam \(E\), \(B\), \(F\) varoedades dif. \(\pi: E \to B\) suave é um \textit{\textbf{fibrado}} com fibra \(F\) se
	\begin{itemize}
	\item \(\pi\) é uma submersão sobrejetiva.
	\item  para todo \(b \in B\) existe \(U \subset B \) aberto \( U \ni b\) e um difeomorfismo
		\[\begin{tikzcd}
		\pi^{-1}(U)\arrow[rr,"\text{ dif} "]\arrow[dr,"\pi",swap]&&U \times F\arrow[dl,"\operatorname{pr}_1"]\\
		&U
		\end{tikzcd}\]
	\end{itemize}
	\end{defn}

\item  \(\alpha \) e \(\beta\) são fibrados. Também \(\alpha \times \beta: J^k(X,Y)\to X \times Y\). As fibras do último são difeomorfas a um espaço vetorial (polinomios B) mas esse difeomorfismo não induzem estrutura de espaço vetorial. Caso \(Y=\mathbb{R}^n\) pode sim pq os jatos são funções em \(m\mathbb{R}^n\), assim pode botar estrutura de espaço vetorial nas fibras.
\item Para cada função  \(f \in C^\infty(X,Y)\) temos uma \textbf{seção} de \(J^k(X,Y)\) chamada \(j^k f:X \to J^k(X,Y)\) dada por \(x \mapsto  [f]_x\).

\item  Topologia de Whitney \(C^k\) defn.
\item  Lema: \(k \leq  \ell \implies  W_k \subset W_\ell\).
\item Def: a topologia \(C^ \infty\) \textit{\textbf{de Whitney}} é a topologia gerada por \( \bigcup_{k=1}^\infty W_k\) onde \(W_k\) e a \(k\)-ésima. ``\(U\) é aberto em \(C^\infty\)" se para todo \(x\in U\) existe \(V\) aberto em algum \(C^k\) (para todo \(x\) existe \(V\) e existe \(k\)) tais que \(x \in V \subset U\).
\item  Def: Seja \(\delta : X \to \mathbb{R}_+=(0,\infty)\) e \(f \in C^\infty(X,Y)\) definimos a \textit{\textbf{\(\delta\)-bola}} como
	\[D_\delta(f)=\{f \in C^\infty(X,Y)|d(j^kf(x),j^kg(x))<\delta(x)\forall x \in X\}\]
	onde $d$ é qualquer métrica em \(J^k(X,Y)\). {\color{6}(A distancia entre as derivadas é muito pequena.)}
\item Prop: \(\{B_\delta(f): \delta:X \to \mathbb{R}_+\}\) é uma base local centrada em \(f\) da topologia \(C^k\), i.e.
	\begin{itemize}
	\item \(B_\delta(f)\) é aberto.
	\item \(\forall  \mathcal{W} \ni f\) aberto, existe \(\delta\) tal que \(B_\delta(f) \subset \mathcal{W}\).
	\end{itemize}
\item Fix: a locally finite atlas \(\Phi\) of \(X\) with a \(\Subset\) cover (every open set of the cover contains an open subset whose closure is compact and contained in the original open set), an atlas \(\Psi\) of \(Y\), e uns números positivos \(\mathcal{E}=\{ \varepsilon_i\}\), e uma função \(f \in C^\infty(X,Y)\). Então a topologia gereada por (foto) coincide com a topologia \(C^k\).
\item Obs: quando \(X\) é compacto, \(\delta_n=\frac{1}{n}\), \(\{B_{\delta_n}(f)\}\) é uma base local da topologia \(C^k\).
\item Prop: seja \(\{ f_n\}_{n \in \mathbb{N}} \subset C^\infty(X,Y)\), \(f_n \xrightarrow{C^k}f\). Então existe \(K \subset X\) compacto tal que \(f_n \equiv f\) em \(X \setminus K\) e \(j^k f_n \xrightarrow{u}j^kf\) em \(K\).
\end{enumerate}


\section{Aula 6: topologia de Whitney (cont.). Teoremas de transversalidade.}

\subsection{Topologica de Whitney (cont.)}

\begin{enumerate}
\item Lembrança das topologias \(C^k\) e \(C^\infty\).
\item \(X\) variedade não compacta existe \(\{ K_n\}\) compactos tais que \(K_n \subset \int (K_{n+1}\) e \(\bigcup_{n \in N} K_n=X\). Também existe \(\rho: X \to \mathbb{R}_+\),  \(\rho|_{K_{2n+1}\setminus K_{2n}}=2\). \(\rho\) é própria.
\item  Em toda variedade existe uma métrica completa (análise em variedades…).
\item Definição de espaço de Baire.
\item Teo (Prop 3.3). Se \(X, Y\) são variedades suaves, então \(C^\infty(X,Y)\) é um espaço de Baire na topologia \(k=1,2,\ldots,+\infty\).
\item Prop 3.\(X, Y\) variedades diferenciaveis \( j^k : C^\infty(X,Y)\to C^\infty(X, J^k(X,Y))\) é contínua em \(C^\infty\).
\item  \(\phi: Y \to Z \) suave então \(\phi_*: C^\infty(X,Y)\to C^\infty(X,Z)\), \(f \mapsto \phi \circ f\) é contínua em \(C^\infty\). Também \(C^\infty(X,Y) \times C^\infty(X,Z)\to C^\infty(X, Y \times Z)\), \(f, g) \mapsto f \times g\).
\end{enumerate}

\subsection{Teoremas de transversalidade}

A ideia é que a transervsalidade é uma condição generica.

\begin{enumerate}
\item Def: \(X,Y\),  \(f \in C^\infty(X,Y)\), \(W \subset Y\) subvariedade. \(f\) é \textit{\textbf{transversal}} a \(W\) se para todo  \(x \in f^{-1}(W)\),
	\[df_x(T_xW)+T_{f(x)}W=T_{f(x)}Y.\]
\item Obs. Isso impoe restrições sobre as dimensoes: se \(f \pitchfork W\) e \(f^{-1}(W)\neq \varnothing.\) Então \(\dim X + \dim W \geq \dim Y\).

	\begin{remark}[\cite{gui}, p. 35]\leavevmode
	Why can't two curves in \(\mathbb{R}^3\) never intersect transversally (except if they do not intersect at all)? Doesn't make sense because of course they can intersect transversally, right? what happens is that \textit{by a small deformation of either curve, one can abruptly pull the two enteirly apart; their intersection is not stable}:
	\begin{figure}[H]
		\centering
		\includegraphics[width=0.7\textwidth]{fig2}
		\caption*{}
	\end{figure}
	\end{remark}
\item Teorema muito importante. \(X,Y\) variedades suaves \(f \in C^\infty(X,Y)\), \(W \subset Y\) subvariedade, \(f \pitchfork W\) e \(f^{-1}(W) \neq  \varnothing\). Então \(f^{-1}(W)\subset X\) é uma subvariedade e \(\operatorname{codim}f^{-1}(W)=\operatorname{codim}(W)\).

	Outra encarnação do teroema da função inversa.

	\begin{remark}[\cite{gui}, p. 29]\leavevmode
	When \(W\) is just a single point, its tangent space is the zero subspace of \(T_y(Y)\). Thus \(f\) is transversal to \(y\) if \(df_x[T_x(X)]=T_y(Y)\) for all \(x \in f^{-1}(y)\) \textbf{which is to say that \(y\) is a regular value of \(f\)}. So transversality includes the notion of regularity as a special case. 
\end{remark}
\end{enumerate}

\section{Aula 7}



\begin{enumerate}
\item Prop: Sejam \(X,Y\) variedades e \(W \subset Y\) subvariedade fechada como subconjunto. \[T_W:= \{f \in C^\infty(X,Y): f \pitchfork W\}\]
é aberto em \(C^\infty\).

{\color{7}A ideia é que podemos perturbar variedades que se intersectam transversalmente, e isso fica transversal que podemos perturbar variedades que se intersectam transversalmente, e isso fica transversal.}

\item Prop: Sejam \(X,Y,B\) variedades, \(W \subset Y\) subvariedade, \(j:B \to C^\infty(X,Y)\) (não podemos supor que \(j\) é contínua. Considere também
\begin{align*}
	\Phi: X \times B &\longrightarrow Y \\
	(x,b) &\longmapsto j(b)(x).
\end{align*}
Suponha que \(\Phi \pitchfork W\).

Então \(\{ B \in B : j(b) \pitchfork W\}\) é denso em \(B\). (Também é verdade que esse conjunto tem medida total.)
	
\item Corolário 4.7.
\item Teorema de transversalidade de Thom. (Formulação.)
\item Coro 4.11. (Formulação.)

\end{enumerate}

\section{Aula 8: teorema de transversalidade de Thom; multijatos}

\subsection{Teorema de transversalidade de Thom}

\begin{thm}[de transversalidade de Thom]\leavevmode
Sejam \(X, Y\) variedades diferenciáveis, \(W \subset J^k(X,Y)\) subvariedade. Então
\[\{ d \in C^\infty (X,Y): j^kf \pitchfork W\}\]
é residual (=interseção de abertos densos).

(Se \(W\) é fechado, então o conjunto também é aberto.)
\end{thm}

\begin{proof}\leavevmode
Primeiro, para o último comentário ali entre paréntese, note que se \(W\) é fechado,
\[U=\{g \in C^\infty (X,J^k(X,Y)):g \pitchfork W\}\]
é aberto. Note que \(T_W= (j^k)^{-1}(U)\).

Como \(j^k: C^\infty (X,Y) \to C^\infty (X,J^k(X,Y))\) é contínua, \(T_W\) é aberto.

\textbf{Começa a prova.} $\mathsf{OK}$ então pegue \(\sigma \in W\). Seja \(W_\sigma \subset W\)  uma vizinhança, e cartaso \(\varphi_\sigma: U_\sigma \to \mathbb{R}^n\), \(\psi_\sigma:V_\sigma \to \mathbb{R}^m\) tais que
\begin{enumerate}
\item \(\alpha(\sigma) \in U_\sigma\) \(\beta(\sigma) \in V_\sigma\).
\item \(\overline{W}_\sigma\), \(\overline{U}_\sigma\) compactos.
\item \(\alpha(\overline{W}_\sigma) \subset U_\alpha\), \(\beta(\overline{W}_\sigma) \subset V_\sigma\).
\item \(\psi_\sigma(V_\sigma) = \mathbb{R}^m\).
\end{enumerate}
Então beleza. Agora pegue
\[T_\sigma=\{f \in C^\infty (X,Y): j^kf \pitchfork W \text{ em } \overline{W}_\sigma\}\]
Então
\[\bigcap_{\sigma \in W}T_\sigma=T_W.\]
Como \(W\) é 2-enumerável, podemos escolher \(\{\sigma_n\}_{n \in \mathbb{N}}\) tal que \(\bigcup_{n \in \mathbb{N}}W_{\sigma_n}=W\). Logo
\[ \bigcap_{n \in \mathbb{N}}=T_W.\]
\begin{claim}\leavevmode
\(T_\sigma\) é aberto e denso.
\end{claim}

\begin{proof}[Prova da afirmação]\leavevmode
\begin{prop}[da aula passada modificada]\leavevmode
\(X,Y\) variedades dif. \(W\subset Y\) subvariedade, \(W' \subset W\) fechado em \(Y\). Então
 \[\{f \in C^\infty (X,Y): j^kf \pitchfork W \text{ em } W'\}\]
é aberto.
\end{prop}
Ussando essa proposição é o fato de que \(j^k\) é contínua, mostramos que \(T_\sigma\) é aberto.

{\color{6}
Para ver que \(T_\sigma\) é denso considere \(f \in C^\infty (X,Y)\). Vamos construir \(\{ g_n\}\subset C^\infty (X,Y)\) tal que \(g_n \xrightarrow{C^\infty }f\) e \(g_n \in T_\sigma\).}

Vamos escolher
\[\rho_1:\mathbb{R}^n\to[0,1]\qquad \qquad \rho_2:\mathbb{R}^m\to [0,1]\]
suaves, \(\rho_1 \equiv 1\) numa vizinhança de \(\varphi_\sigma(\alpha(\overline{W}_\sigma))\), \(\rho_2 \equiv 1\) numa viz. de \(\psi_\sigma(\beta(\overline{W}_\sigma))\), \(\operatorname{supp}\rho_1, \operatorname{supp}\rho_2\) compactos (partição da unidade).

Seja \(B=\{\text{funções polinomiais de \(\mathbb{R}^n \to \mathbb{R}^n\) de grau \(\leq k\)} \} \cong \mathbb{R}^m \oplus  B^k_{n,m}\).

Para \(b \in B\), definimos \( g_b: X \to Y\), 


\[g_b=\begin{cases}
	\psi^{-1}_\sigma(\psi_\sigma(f(x))+\rho_2(\psi_\sigma(f(x))\rho_1(\varphi_\sigma(x))\cdot b(\varphi_\sigma(x))),\qquad &\text{se }x \in U_\sigma\text{ e } f(x) \in V_\sigma  \\
	f(x)\qquad &\text{ se } x \not \in U_\sigma \text{ e } f(x) \not \in V_\sigma
\end{cases}\]
\(g_0=f\).

\paragraph{Intuition for \(g_b\) by ChatGPT.} 
The function \(g_b\) is a smooth perturbation of \(f\) designed to ensure \(j^k g_b \pitchfork W\) in \(\overline{W}_\sigma\). Locally, near \(\overline{W}_\sigma\), \(g_b\) behaves like \(f\) plus a polynomial perturbation \(b(\varphi_\sigma(x))\), where \(b \in B\) spans the \(k\)-jet space. The partition of unity functions \(\rho_1, \rho_2\) localize the perturbation to a compact region, ensuring \(g_b\) is globally smooth and matches \(f\) outside this neighborhood.

Agora:
\begin{align*}
	\Phi: X \times B &\longrightarrow J^k(X,Y) \\
	(x,b) &\longmapsto j^kg_b(x)
\end{align*}

\paragraph{Dependence of \(g_b\) on \(b\).} 
For each \(b \in B\), where \(B\) is the space of polynomials of degree \(\leq k\) in the local coordinates \(\varphi_\sigma(x)\), there is a corresponding perturbation \(g_b\). The choice of \(b\) determines how \(g_b\) modifies \(f\) locally near \(\overline{W}_\sigma\). The perturbation:
\begin{itemize}
    \item Is localized to \(\overline{W}_\sigma\) by the partition of unity functions \(\rho_1\) and \(\rho_2\), ensuring \(g_b = f\) outside \(U_\sigma\).
    \item Explores all possible modifications in the \(k\)-jet space \(B^k_{n,m}\), making the construction rich enough to enforce transversality.
\end{itemize}
While \(b\) can be chosen freely from \(B\), the perturbation is controlled, smooth, and compactly supported, ensuring \(g_b\) remains globally well-behaved.


\textbf{Ideia:} \(\Phi \not{\pitchfork} W\).

Seja \(\varepsilon=\frac{1}{2}\operatorname{dist}(\psi_\sigma(\beta(\overline{W}_\sigma)),\rho^{-1}_2([0,1)))>0\).

\[\tilde{B}=\{b \in B: \|b(x)\|<\varepsilon \forall  x \in \operatorname{sup p} \rho_1\}\]
aberto, \(0 \in \tilde{B}.\)


\paragraph{The subset \(\tilde{B}\) (GPT).} 
The set \(\tilde{B}\) is a subset of the polynomial space \(B\), defined as:
\[
\tilde{B} = \{b \in B : \|b(x)\| < \varepsilon \, \forall x \in \operatorname{supp} \rho_1\},
\]
where \(\varepsilon = \frac{1}{2} \operatorname{dist}(\psi_\sigma(\beta(\overline{W}_\sigma)), \rho_2^{-1}([0,1))) > 0\). This means that \(\tilde{B}\) contains only those polynomials \(b(x)\) whose values are bounded by \(\varepsilon\) within the support of \(\rho_1\). The set \(\tilde{B}\) is open in \(B\), and \(0 \in \tilde{B}\).

\(\tilde{\Phi}:\Phi\Big|_{X \times \tilde{B}}X \times \tilde{B} \to J^k(X,TY)\) é um difeomorfismo local numa viz. de \((x,0)\), onde  \(\Phi(x,b) \in \overline{W}_\sigma.\)

\((x,b)\) t.q. \(\Phi(x,b) \in \overline{ W}_\sigma\). \(j^k g_b(x) \implies x \in \alpha (\overline{W}_\sigma)\), \(g_b(x)=\beta (\Phi(x,b))\).

\[\Psi_b(g_b(x))=\Psi_\sigma(f(x))+\rho_2(\psi_\sigma(f(x))\rho_1(\varphi_\sigma(x))b(\varphi_\sigma(x)).\]
\[\|\Psi_\sigma(g_b(x))-\psi_\sigma(f(x))\| \leq  \|b(\varphi(x))\|<\varepsilon.\]
Então
\[\rho_2(b(\varphi_\sigma(x)))=1.\]

\[g_b(x)=\psi^{-1}_\sigma(\psi_\sigma(f(x))+b(\varphi_\sigma(x)))\]
\[g_b'(x)=\psi^{-1}_\sigma(\psi_\sigma(f(x))+b'(\varphi_\sigma(x))\]
para \(b'\) suficientement próxima de  $b$. \textbf{Isso implica que \(\Phi\)} é um difeomorfismo local para todo \((x,b) \in \Phi^{-1}(\overline{W}_\sigma\).

\paragraph{Summary of \(\tilde{B}\) and its Role (GPT).}
The set \(\tilde{B} = \{b \in B : \|b(x)\| < \varepsilon \, \forall x \in \operatorname{supp} \rho_1\}\), where 
\(\varepsilon = \frac{1}{2} \operatorname{dist}(\psi_\sigma(\beta(\overline{W}_\sigma)), \rho_2^{-1}([0,1))) > 0\), defines a controlled subset of perturbations. Perturbations \(b \in \tilde{B}\) ensure that:
\begin{itemize}
    \item The map \(\tilde{\Phi} = \Phi\big|_{X \times \tilde{B}} : X \times \tilde{B} \to J^k(X, Y)\) is a local diffeomorphism near \((x, 0)\), allowing smooth control of \(j^k g_b\).
    \item For \((x, b)\) such that \(\Phi(x, b) \in \overline{W}_\sigma\), we have \(x \in \alpha(\overline{W}_\sigma)\) and \(g_b(x) = \beta(\Phi(x, b))\), ensuring the perturbation is relevant to \(\overline{W}_\sigma\).
    \item The bound \(\|\Psi_\sigma(g_b(x)) - \psi_\sigma(f(x))\| < \varepsilon\) ensures the perturbation remains localized, with \(\rho_2 = 1\) in the active region, maintaining smoothness and compact support.
\end{itemize}
This construction ensures \(g_b\) perturbs \(f\) only where needed, while staying smooth and controlled globally.

Logo \(\tilde{\Phi}\) é submersão para todo \(x \in \tilde{\Phi}^{-1}(\overline{W}_\sigma)\).

Logo \(d \tilde{\Phi}_{(x,b)}(T_{(a,b)}(X \times \Big)+ T_{\tilde{\Phi}(x,b)}W=T_{\tilde{\Phi}(x,b)}(J^k(X,Y)) \implies \tilde{\Phi} \pitchfork W\) em \(X \times \tilde{B}\).

Pelo lema da aula passada
\[j^kg_b=\Phi_b \pitchfork W\]
para um conjunto denso de \(b \in \tilde{ B}\). Então existe \(b_n \to 0\) tal que \(j^k g_{b_n} \pitchfork W\) em \(\overline{W}_\sigma\), \(g_{b_n}\xrightarrow{C^\infty }f\). (Tem uma conta aqui para ser feita.)

\paragraph{Final Step of the Proof.}
The map \(\tilde{\Phi}: X \times \tilde{B} \to J^k(X, Y)\) is a submersion on \(\tilde{\Phi}^{-1}(\overline{W}_\sigma)\), meaning:
\[
d\tilde{\Phi}_{(x,b)}(T_{(x,b)}(X \times \tilde{B})) + T_{\tilde{\Phi}(x,b)}W = T_{\tilde{\Phi}(x,b)}(J^k(X,Y)).
\]
This implies \(\tilde{\Phi} \pitchfork W\) on \(X \times \tilde{B}\).

By a previous lemma, for a dense subset of \(b \in \tilde{B}\), we have:
\[
j^k g_b = \tilde{\Phi}_b \pitchfork W \text{ on } \overline{W}_\sigma.
\]
Thus, there exists a sequence \(b_n \to 0\) in \(\tilde{B}\) such that:
\[
j^k g_{b_n} \pitchfork W \text{ on } \overline{W}_\sigma, \quad \text{and} \quad g_{b_n} \xrightarrow{C^\infty} f.
\]
This concludes the proof by showing that \(f\) can be approximated arbitrarily closely by maps achieving transversality.


\end{proof}

\end{proof}

\begin{coro}[da demostração]\leavevmode
Sejam \(X, Y\) variedades, \(W \subset J^k(X,Y)\) subvariedade, agora já sei que \(\alpha(\overline{W}) \subset U\), \(U\) aberto de \(X\). {\color{6}``So preciso modificar a função num conjunto aberto, não preciso modificar a função globalmente."}

Então existe \(g \in \mathcal{V}\) tal que \(j^kg \pitchfork W\) \textbf{e \(g \equiv f\) em \(X\setminus U\)}.
\end{coro}

\begin{coro}[Versão simples da transversalidade]\leavevmode
\begin{enumerate}[label=(\alph*)]
\item \(X,Y\) variedades, \(W \subset Y\) subvariedade. Então
\[\{ f \in C^\infty (X,Y): f \pitchfork W\}\]
é denso.

Se \(W\) é fechado, então esse conjunto é aberto.
\item \(U_1,U_2\subset X\)abertos, \(\overline{U_1}\subset U_2\). Então existe \(g \in C^\infty (X,Y)\) tal que \(f \equiv g\) em \(U_1\), \(g \pitchfork W\) em \(X \setminus U_2\). (É só aplicar o corolário para caso de 0-jatos.
\end{enumerate}
\end{coro}

\subsection{Multijatos}
Um pouco mais de teoria para obter corolários bonitos.

\(X,Y\) variedades, \(s \in \mathbb{N}\).
\[X^{(s)}=\{(x_1,x_s)\in X^s: x_1 \neq  x_j, i\neq j\}\]
(são os pontos que não tem duas coordenadas iguais). Isso é claramente aberto em \(X^s\).

\begin{align*}
	\alpha^s: J^k(X,Y)^s &\longrightarrow X^s \\
	(\sigma_1,\ldots,\sigma_s) &\longmapsto (\alpha(\sigma_1),\ldots,\alpha(\sigma_s))
\end{align*}
O espaço de \textit{\textbf{multijatos}} é o espaço de $s$-jatos tal que cada jato está num ponto diferente:
\[J^k_s(X,Y):=(\alpha^s)^{-1}(X^{(s)})\]
i.e. \((\sigma_1,\ldots,\sigma_s) \in J^k_s(X,Y)\) sse \(\sigma_1=j^kf_i(x_i)\), \(x_i \neq  x_j, \qquad  i \neq  j\).

Note que
\[J^k_s(X,Y) \subset J^k(X,Y)^s\]
é aberto.

E agora
\[\alpha^s: J^k_s(X,Y) \to X^{(s)}\]
fibrado.

Para \(f \in C^\infty (X,Y)\),
\begin{align*}
	j^k_sf: X^{(s)} &\longrightarrow J^k_s(X,Y) \\
	(x_1,\ldots,x_s) &\longmapsto (j^kf(x_1),\ldots,j^kf(x_s))
\end{align*}

\begin{thm}[transversalidade para multijatos]\leavevmode
Sejam \(X,Y\) variedades e \(W \subset J^k_s(X,Y)\) subvariedade.
\[T_w=\{ f \in C^\infty (X,Y):j^k_sf \pitchfork W\}\]
é residual. Além disso, se \(W\) é \textbf{compacto}, então \(T_W\) é aberto.
\end{thm}

\subsection{Imersão e mergulho de Whitney}

Sejam \(X, Y\) variedades, \(\sigma = j^1f(x) \in J^1(X,Y)\), \(\operatorname{rk}\sigma=\operatorname{rk}(df_x:T_xX \to T_{f(x)}Y)\). Definimos o \textit{\textbf{coposto}} como sendo
\[\operatorname{corank}=\operatorname{min}(n,m)-\operatorname{rk}\sigma\]
\begin{lemma}\leavevmode
Seja \(S_r=\{\sigma \in J^1(X,Y): \operatorname{corank}\sigma=r\}\). $f$ é imersão (\(n\leq n\)) ou submersão ( \(n \geq m\)) see \(j^1f(X) \cap \bigcup_{r \geq  1}S_r= \varnothing\).
\end{lemma}

\begin{thm}[Imersão de Whitney]\leavevmode
Sejam \(X,Y\) variedades tais que \(m \geq  2n\). Então
\[\operatorname{Im}(X,Y)= \{ f: X \to Y: f \text{é imersão} \}\]
é aberto e denso.
\end{thm}

\begin{prop}\leavevmode
\(S_r\) é uma subvariedade de codimensão \((n-q+r)(m-q+r)\), onde \(q=\operatorname{min}(m,n)\) (o posto máximo).
\end{prop}

\begin{proof}\leavevmode
\(S_r\) é um fibrado sobre \(X \times Y\) cuja fibra é
\[\{A \in \mathcal{L}(\mathbb{R}^n,\mathbb{R}^m): \operatorname{corank}k=r\}\]
Pega uma transformação linear de posto $r$, \(M \in \mathcal{L}^r( \mathbb{R}^n , \mathbb{R}^m)\), \(q = \operatorname{min} (n,m)\), \(k = \operatorname{rk} M= q-r\). Então
\[[M]=\begin{bmatrix} A& B\\C & D \end{bmatrix} \]
Seja \(U\) uma viz. de \(M\) em \(\mathcal{L} (\mathbb{R}^m,\mathbb{R}^m)\). Pegue \(M' \in U\). Então

\(\begin{bmatrix} M' \end{bmatrix} = \begin{bmatrix} A' & B'\\ C' & D' \end{bmatrix}_{n \times m}\) 

Vamos multiplicar \(M'\) por uma matriz que preserva o rank:
\[\operatorname{rk}M'=\operatorname{rk} \begin{bmatrix} I_k& 0\\-C'(A')^{-1} &  I_{m-k} \end{bmatrix}_{m \times m}M'\]
\[=\operatorname{rk} \begin{bmatrix} A' & B'\\0 & D'-C'(A')^{-1} B' \end{bmatrix} \]
{\color{7}Como é que o posto dessa matriz é \(k\)?} Isso acontece see  \(D'-C'(A')^{-1}B'=0\).

Ou seja, \(M' \in \mathcal{L}^r(\mathbb{R}^n,\mathbb{R}^m) \iff D'-C'(A')^{-1}B'=0\).

\begin{align*}
	\varphi: U \subset \mathcal{L}(\mathbb{R}^n,\mathbb{R}^m) &\longrightarrow \mathcal{L}(\mathbb{R}^{n-k},\mathbb{R}^{m-k}) \\
	[M']=\begin{bmatrix} A' & B'\\ C' & D' \end{bmatrix}  &\longmapsto [D' - C'(A')^{-1}B']
\end{align*}

\[\mathcal{L}^r(\mathbb{R}^n,\mathbb{R}^m) \cap U = \varphi^{-1}(0).\]
Submersão, pois
\[\varphi \begin{bmatrix} A' & B' \\ C' & * \end{bmatrix} =\operatorname{Id}- C' (A')^{-1}B'\]
\[\operatorname{codim} \mathcal{L}^r(\mathbb{R}^n,\mathbb{R}^m)=(n-k)(m-k)=(m-q+r)(n-q+r)\]
\end{proof}

\begin{proof}[Do teorema de imersão de Whitney]\leavevmode
\begin{align*}
\operatorname{Im}(X,Y)&=\{f : j^1f(X) \subset S^0\}\\
&=\{ f : j^1f(X) \cap \bigcup_{r \geq  1}S_r = \varnothing\}.
\end{align*}
\end{proof}

\section{Aula 9}

\subsection{Teorema das imersões injetivas}

\begin{thm}[das imersões injetivas]\leavevmode
\(X^n, Y^m\) variedades dif. \(m \geq  2n+1\),
\[\{f \in C^\infty(X,Y): f \text{ imersão injetiva} \}\]
é residual.
\end{thm}
\begin{proof}\leavevmode
Note que $f$ não é injetiva pode ser dito na linguagem de multijatos assim: \(\exists  (x_1,x_2) \in X^{(2)}\) tal que \(j^0_2f(x_1,x_2) \in X^{(2)}\times \Delta_Y\).

Logramos argumentar que $f$ é injetiva see \(j^0_2f(X^{(2)}) \pitchfork W= \varnothing\) para \(W=X^{(2)}\times \Delta_Y \subset J^0_2(X,Y)\), note que \(\operatorname{codim} W=\dim Y=m\). Dai \(\dim X^{(2)}=2n<m=\operatorname{codim} W\). Dai \(j^0_2 f \pitchfork W \iff j^0_2f(X^{(2)}) \pitchfork W= \varnothing\).

aplicamos teorema de transversalidade de multijatos, obtendo que \(\operatorname{In j}(X,Y)\) é residual, e o resultado segue.
\end{proof}

\begin{lemma}\leavevmode
Seja \(X\) variedade.
\[\operatorname{Pr op}(X, \mathbb{R}^m)=\{f: X \to \mathbb{R}^m: f \text{ é própria} \}\]
é não-vazio e aberto.
\end{lemma}

\begin{proof}\leavevmode
\begin{itemize}
\item \textbf{(Não vazio.)} Pegue \(\rho:X \to \mathbb{R}\) própria. Dai \(i:\mathbb{R} \to \mathbb{R}^m\), \(x \mapsto (x,0,\ldots,0)\) é injetiva e linear, asi \(i \circ \rho\) é própria.

\item \textbf{(Aberto.)} Pegue \(f \in \operatorname{Pr o p}(X,\mathbb{R}^m)\). Pegue \(V \subset J^0(X,\mathbb{R}^m) \cong X \times \mathbb{R}^m\) aberto. Queremos ver que \(f \in M(V) \subset \operatorname{ Pr o p}(X,\mathbb{R}^m)\).

	Defina \(Vx=\{y \in \mathbb{R}^m: d(y,f(x))<1\}.\) O truque e esse!

	\[g \in M(V) \iff d (g(x),f(x))<1 \forall  x \in X \iff j^0 g(X) \subset V\]
	então para \(g \in M(V)\) com \(d(g(x),g(x))<1\)  \(\forall  x \in X\), e asi
	\[g^{-1}(\overline{B_r(0)}\subset f^{-1}(\overline{B_r(0))}\]
	e isso implica que \(g\) é própria.
\end{itemize}
\end{proof}

\subsection{Teorema do mergulho de Whitney}

\begin{coro}[Teorema do mergulho de Whitney]\leavevmode
\(X^m\) variedade, \(X \hookrightarrow  \mathbb{R}^{2n+1}\).
\end{coro}

\begin{proof}\leavevmode
\(\operatorname{Im}(X,Y) \cap \operatorname{Inj}(X,Y)\) é residual, logo denso. Então \(\operatorname{Im}(X,Y) \cap \operatorname{Inj}(X,Y) \cap \operatorname{Pr o p}(X,Y) \neq  \varnothing\), which implies that \(X \hookrightarrow  Y=\mathbb{R}^{2n+1}\).
\end{proof}

\subsubsection{Funções de Morse}

\begin{defn}\leavevmode
\( f \in C^\infty(X)=C^\infty(X,\mathbb{R})\). Um ponto crítico de $f$ é quando a derivada não tem posto máximo, neste caso isso implica que a derivada tem posto 0, ou seja, \(d_pf=0\).
 \[D^2_pf:T_pM \times T_pM \to \mathbb{R}\]
 é uma forma bilinear simétrica. Para mostrar que isso está bem definido quando mudamos de coordanadas necesitamos usar que \(d_pf=0\). Também pode ver isso usando  ``conexões": quando a derivada é zero, temos uma eleção canônica de horizontal bundle e podemos identificar \(T_{(p,\underbrace{v}_{=0})}(TM) \cong T_pM \times T_pM\)
\end{defn}

\begin{defn}\leavevmode
\(p \in \operatorname{Crit}(f)\) é não degenerado se \(D^2f_p\) é não degenerada, i.r. para todo \(v \in T_pM \setminus\{0\}\) existe \(w\) tal que \(D^2f_p(v,w)\neq 0 \iff \left(\frac{\partial^2(f \circ \varphi^{-1}}{\partial x^i \partial x^j}\right) \) é residual.
\end{defn}

\[S^1:= \{ j^1 g(x) \in J^1(X,\mathbb{R}):dg_x=0 \iff \operatorname{corank}\sigma=1\}\]

\begin{prop}\leavevmode
\(p \in \operatorname{Crit}(f)\) é não degenerada \(\iff\) \(j^1f \pitchfork S^1\) em $p$. \(p \in (j^1f)^{-1}(S^1)\)
\end{prop}

\begin{proof}\leavevmode
Se usa que a dimensão de \(X\) é a mesma do que a dimensão de \(\mathcal{L}(\mathbb{R}^{2n},\mathbb{R})\).
\end{proof}

\begin{defn}\leavevmode
\(f \in C^\infty(X)\) é uma \textit{\textbf{função de Morse}} se todo ponto crítico de $f$ é não  degenerado.
\end{defn}

\begin{coro}[da proposição]\leavevmode
$f$ é de Morse \(\iff\) \(j^1 f \pitchfork S^1\).
\end{coro}

\begin{thm}\leavevmode
\(\{f \in C^\infty(X): f \text{ é de Morse} \}\) é aberto denso de \(C^\infty(X,\mathbb{R})\).
\end{thm}

\begin{proof}\leavevmode
\(S^1\) é fechado porque é subvariedade, logo o conjunto das funções de Morse é igual a \(\{f : j^1 f \pitchfork S^1\}\) é aberto e denso pelo teorema de Thom.
\end{proof}

\subsection{Teoria de interseção}

\subsubsection{Preliminares: variedades com bordo e orientação}

\begin{defn}\leavevmode
Uma \textit{\textbf{variedade topológica}} \(X\) \textit{\textbf{com bordo}} é um espaço topológico Hausdorff 2-enumerável tal que todo ponto possui uma vizinhança aberta homeomorfa a um aberto de \(\mathbb{H}^n=\{(x_1,\ldots,x_n) \in \mathbb{R}^n :x_1 \geq 0\}\). \(\partial \mathbb{H}^n=\{(x_1,\ldots,x_n):x_1=0\}\).
\end{defn}

\begin{prop}\leavevmode
\(X\) variedade topológica com bordo, \(p \in X\) tal que existe cart \((U,\varphi)\), \(p \in U\) tal que \(\varphi(p) \in \partial \mathbb{H}^n\). Seja \((V,\psi)\) outra carta, \(p \in V\). Então \(\psi(p) \in \partial \mathbb{H}^n\).
\end{prop}

\begin{proof}\leavevmode
Suponha que existe \((V,\psi)\) tal que \(\psi(p) \in \operatorname{in t}\mathbb{H}^n\). Pega a bolinha pequenenina que está contida no interior do \(\mathbb{H}^n\) e a preimagem dela está dentro de \(U\). Então obtemos um difeomorfismo entre uma bola cortada e uma bola. Tira o ponto: de um lado obtemos um conjunto contrátil, do outro lado não. Usar topologia algébrica.
\end{proof}

\begin{defn}\leavevmode
O \textit{\textbf{bordo}} de uma variedade com bordo é
\[\partial X=\{ p \in X : \exists (U,\varphi)\text{ cata } , U \ni p, \varphi(p) \in \partial \mathbb{H}^n\}\]
O \textit{\textbf{interior}} é
\[\operatorname{ in t }X \setminus \partial X=\{ p \in X: \exists  (U, \varphi) \text{ carta,} U \ni p, \varphi(p) \in \operatorname{in t}\mathbb{H}^n\}.\]
\end{defn}

\begin{remark}\leavevmode
\(\operatorname{ in t}X, \partial X\) são variedades topológicas sem bordo.
\end{remark}

\begin{defn}\leavevmode
\(f: U \overset{\text{op} }{\subset} \mathbb{H}^n \to \mathbb{H}^n\)  é \textit{\textbf{suave}} se existe uma extensão suave \(\tilde{f}: \tilde{U} \subset \mathbb{R}^n \to \mathbb{R}^n\), \(\tilde{U} \supset U\), \(\tilde{U}\) aberto.
\end{defn}

\begin{thing4}{Intuição}\leavevmode
É uma função que era suave e vc trabou. Não que ela acaba aí.
\end{thing4}

\begin{defn}\leavevmode
\(X\) é uma \textit{\textbf{variedade diferenciável com bordo}} se está munida de um atlas suave maximal. (Tem que definir compatibilidade de cartas em \(\mathbb{H}^n\).)
\end{defn}

\begin{example}[Look!]\leavevmode
	\(B^n\) the unit ball is \(f^{-1}((-\infty,1])\) where \(f\) is the norm function.
\end{example}

\begin{prop}\leavevmode
	\(f \in C^\infty(X)\), \(a \in \mathbb{R}\) valor regular de $f$. Então \(f^{-1}(-\infty,a]\) e \(f^{-1}([a,+\infty))\)são variedades com bordo.
\end{prop}

\begin{proof}\leavevmode
	\(f^{-1}(-\infty,a))\) é aberto porque $f$ é contínua, então é uma subvariedade do domínio. Seja \(p \in f^{-1}(a)\). Usando TFI podemos upor que \(\varphi^{-1}(x_1,\ldots,x_n)=-x_1+a\). Pegue \(y \in f^{-1}(-\infty,a])\)…
\end{proof}

\begin{defn}\leavevmode
\(X\) variedade dif. com bordo. O \textit{\textbf{espaço tangente}} em pontos interiores é o mesmo que em variedades sem bordo, e no bordo fica um espaço vetorial da mesma dimensão: porque com curvas que moram no interior e podem ser extendidas podemos obter muitos vetores.
\end{defn}

Se \(X\) é uma variedade com bordo, também é podemos definir:

\begin{itemize}
\item \textit{\textbf{submersão}} em \(x \in X\) se a derivada for sobreyetiva.
\end{itemize}

\begin{prop}[level set for manifolds with boundary]\leavevmode
	Sejam \(X,Y\) variedades com bordo, \(\dim X > \dim Y\), pega um ponto \(y \in \operatorname{ in t}Y\), \(f \in C^\infty(X,Y)\) tal que \(y\) valor regular de $f$ e \(y\) \textbf{valor regular de \(\partial f:=f|_{\partial X}\) } (isso é mais forte). Então \(f^{-1}(y)\) é uma variedade com bordo e \(\partial f^{-1}(y)=(\partial f)^{-1}(y)\).
\end{prop}

\section{Aula 10}

\subsection{Mais variedades com bordo}

Lembre que o espaço tangente em pontos do bordo não tem mistério: \(T_xX=\operatorname{span}\frac{\partial }{\partial x^i}\), porque é generado pelas derivadas de curvas que realmente continuam sendo suaves em \( \mathbb{R}^n\).

Como o bordo de \(X\) é uma subvariedade, note que, olhando essas curvas tangentes no bordo como curvas em \(M\), obtemos uma inclusão natural
\[T_x(\partial X)\subset T_x X\]

\begin{prop}\leavevmode
\(X\), \(Y\) variedades \(X\) tem bordo, \(\dim X> \dim Y\) e \(\partial Y \neq  0\). \(y \in Y\) valor regular de \(f\), e valor regular de \(\partial f:= f|_{ \partial X}\).

Então \(f^{-1}(y)\) é uma variedade com bordo de codimensão $m$ e \(\partial f^{-1}(y)=(\partial f)^{-1}(y)=f^{-1}(y) \cap \partial X\).
\end{prop}

``Não é só que a imagem inversa é uma subvariedade; é que a fronteira dela é \textit{exatamente a parte que esta tocando a fronteira} "

\begin{prop}\leavevmode
	\(X,Y\) variedades, \(\partial Y= \varnothing\). \(W \subset Y\) subvaiedade sem bordo. \(f : X \to Y\), \(f \pitchfork W\),  \(\partial f \pitchfork W\).

	Então \(f^{-1}(W)\) é uma subvariedade com bordo \(\partial  (f^{-1}(W)) = (\partial  f)^{-1}(W)= f^{-1}(W) \cap \partial  X\) e \(\operatorname{codim} f^{-1}(W)=\operatorname{codim}W\).
\end{prop}

\subsection{Teorema de Sard com bordo}

\begin{thm}[de Sard com bordo]\leavevmode
\(X,Y\) variedades, \(\partial Y=\varnothing\), \(f:X \to Y\), \(\{y \in Y: y \text{ valor crítico de \(f\) em \(\partial f\)} \}\) tem medida nula.
\end{thm}

\begin{proof}\leavevmode
 \begin{align*}
 f(\operatorname{ C ri t}f \cup \partial f(\operatorname{C r i t}(\partial f)&=f(\operatorname{C r it }(f) \cup  \operatorname{ C r i t}(\partial f))
 \end{align*}
\end{proof}

\subsection{Teorema de transversalidade de Thom com bordo}

\begin{thm}[de transversalidade de Thom com bordo]\leavevmode
\(X,Y\) variedades \(W \subset J^k(X,Y)\), \(\partial W \subset \alpha^{-1}(\partial X)\). Então
\[\{f \in C^\infty(X,Y): j^kf \pitchfork W\text{ e } j^k(\partial f) \pitchfork W\}\]
é residual.
\end{thm}

\begin{proof}\leavevmode
Tá no interior não tem problema. Tá no bordo olha pro bordo é uma variedade sem bordo tudo bem. Interseção de residuais é residual.
\end{proof}

\begin{coro}\leavevmode
\begin{enumerate}
\item \(X,Y\) variedades, \(W \subset Y\) subvariedade, \(\partial Y=\partial W=\varnothing\). Então
	\[\{f \in C^\infty(X,Y): f \pitchfork W \text{ e } \partial f \pitchfork W\}\]
	é residual.
\item \(f \in C^\infty(X,Y)\), \(\partial f \pitchfork W\), \(\forall U \ni f\) aberto \(\exists g \in U\) tal que \(g=f\) em uma vizinhançade \(\partial X\).
\end{enumerate}
\end{coro}

\begin{remark}\leavevmode
Sobre b.: mais forte que \(\partial f\) seja transversal a \(W\) do que \(f\) transversal a \(W\) nos pontos do bordo.
\end{remark}

\subsection{Orientação}

\begin{defn}\leavevmode
\(V\) espaço vetorial. Duas bases \(\{x^i\}, \{y^i\}\) são \textbf{equivalentes}  se a transformação linear \(T:V \to V\), \(x^i \mapsto  y^i\) tem determinatne positivo.
\end{defn}

\begin{remark}\leavevmode
Existem exatamente duas classes de equivalência.
\end{remark}

\begin{defn}\leavevmode
Uma \textit{\textbf{orientação}} em uma variedade \(X\) é uma escolha de orientação de cada \(T_pX\) para todo \(p \in X\) tal que \(\forall (U,\varphi)\) carta \(\varphi=(x_1,\ldots,x_n)\) é sempre positive o sempre negativa (\(\forall p \in U\)).
\end{defn}

\begin{remark}[Extra]\leavevmode
Se \(X\) é simplesmente conexa, então existe uma orientação.
\end{remark}

\begin{remark}\leavevmode
\(X\) conexa e orientável, então \(X\) possui exatamente 2 orientações.
\end{remark}

\begin{remark}\leavevmode
Existem variedades não orientáveis.
\end{remark}

\(X\) variedade orientada com bordo induiz uma orientação em \(\partial X\). É porque existe um fibrado vetorial \(N \) de dimenão 1 ao longo do \(\partial  X\) tal que \(N_x \pitchfork T_x (\partial X)\) \(\forall  x \in \partial  X\). Pode pegar uma métrica Riemanniana o pode fazer assim: pega em cada carta (atlas localmente finito, usamos partição da unidade) ai pega o campo vetorial \(\frac{\partial }{\partial x_1}\) que com a nossa definição de \( \mathbb{H}^{n}\) é a direção que ``sae" da variedade, multiplica com a partição da unidade e some. Ai você tem um campo vetorial que não está no fibrado tangenta ao bordo.

Então em cada \(x \in \partial X\) temos uma base \(\{x_1,\ldots,x^{n-1}\}\) de \(T_x(\partial X)\). Completamos a uma base de \(T_xX\), \(\{n_x,x_1,\ldots,x^{n-1}\}\).

\begin{thing2}{Definição}\leavevmode
\(p \in \partial X\). \(T_p(\partial X) \subset T_pX\). \(N_p \oplus  T_p(\partial X) = T_pX\). uma base é orientada se, {\color{2}e aqui tem que ter alguém que decide}, a base obtida declarando que a primeria entrada vai ter um sinal \(-\frac{\partial }{\partial x_1}\), e completando para uma base, i.e. \(\left(- \frac{\partial }{\partial x_1},\ldots,\right) \), esa base tá na orientação de \(X\).
\end{thing2}

\begin{defn}\leavevmode
\(\{v_1,\ldots,v_n\}\) é \textit{\textbf{orientada}} se \(\{n_x,v_1,\ldots,v_n\}\) é positivamente orientada.
\end{defn}

\begin{example}\leavevmode
O intervalo. Pode decidir que um ponto da borda tenha orientação positiva, ai o outro ponto fica com orientação negativa.

\(X\) orientada sem bordo. \(\partial  (I\times X)=\{0\}\times X \sqcup \{1\}\times X\). \(T_{(t,x)}(I \times X) \cong I_t I \oplus  T_xX\). Então o bordo fica \(\partial (I \times X)= \{1\}\times X \sqcup (- \{ 0\} \times X\).
\end{example}

\begin{prop}\leavevmode
\(X,Y\) variedades, \(W \subset Y\) subvariedade, \(\partial W=\partial Y=\varnothing\), \(f: X \to Y\), \(f \pitchfork W\), \((\partial f)\pitchfork W\). Suponhamos que \(X, Y\) e \(W\) são orientadas.

Então  \(Q= f^{-1}(W)\) possui uma orientação induzida.
\end{prop}






































\bibliography{bib.bib}
\end{document}
