\input{/Users/daniel/github/config/preamble.sty}%This is available at github.com/danimalabares/config
\input{/Users/daniel/github/config/thms-eng.sty}%This is available at github.com/danimalabares/config
\usepackage{multicol}
%\usepackage[style=authortitle-terse,backend=bibtex]{biblatex}
%\addbibresource{/Users/daniel/github/config/bibliography.bib}

\begin{document}

\begin{minipage}{\textwidth}
	\begin{minipage}{1\textwidth}
	Handouts by Misha Verbitsky	 \hfill Daniel González Casanova Azuela
		
		{\small \href{http://verbit.ru/IMPA/METRIC-2023/}{verbit.ru/IMPA/METRIC-2023/} \hfill\href{https://github.com/danimalabares/dt}{github.com/danimalabares/dt}

		 \href{http://verbit.ru/MATH/GEOM-2013/}{verbit.ru/MATH/GEOM-2013}}
	\end{minipage}
\end{minipage}\vspace{.2cm}\hrule

\vspace{10pt}
{\huge Practice exercises on smooth manifolds}

{\large Second meeting, 20 of December}
\vspace{1em}

Plan for today: selection of exercises (I skipped those with *) from sections

\begin{itemize}
\item Hausdorff spaces (3)
\item Compact spaces (8)
\item Topological manifolds (2)
\item Smooth manifolds (4)
\end{itemize}

\section{Hausdorff spaces}

\begin{thing4}{Exercise 1.12}[Points are closed in Hausdorff]\label{exer:1.12}\leavevmode
Let $M$ be a Hausdorff topological space. Prove that all points in $M$ are closed subsets.
\end{thing4}

\begin{proof}[Solution]\leavevmode
Let $x \in M$ and let's see that $M\setminus \{x\}$ is open. Choose a point $y \in M\setminus \{ x\}$. Then there are open sets $U \ni x$ and $V \ni y$ such that $ U \cap V= \varnothing$. Then $V \subset M\setminus \{ x\}$.
\end{proof}

\begin{thing3}{Definition 1.10}\leavevmode
	Let $M,N$ be topological spaces. \textit{\textbf{Product topology}} is a topology on $M \times N$, with open sets obtained as unions $\bigcup_{\alpha} U_\alpha \times V_\alpha$, where $U_\alpha$ is open in $M$ and $V_\alpha$ is open in $N$.
\end{thing3}

\begin{thing4}{Exercise 1.16}\leavevmode
	Prove that a topology on $X$ is Hausdorff if and only if the diagonal $\Delta:=\{(x,y) \in X \times X|x=y\}$ is closed in the product topology.
\end{thing4}

\begin{proof}[Solution]\leavevmode
	$(\implies )$ Suppose that $X$ is Hausdorff. To check that $\Delta$ is closed suppose that $(x,y) \in X \times X$ is a limit point of $\Delta$. We need to show that $(x,y) \in \Delta$, i.e. that $x=y$. If $x \neq y$ we can separate $x$ and  $y$ by disjoint open subsets $U \ni x$ and  $V \ni y$. Then the open set $U \times V$ contains $(x,y)$,  and since $(x,y)$ is a limit point of  $\Delta$ there must be a point $(z,z) \in U \times V$. Then $z \in U$ and $z \in V$, which is a contradiction.

$(\impliedby)$ Suppose $\Delta$ is closed in the product topology and choose two different points $x \neq y$ in $X$. Then $(x,y) \in (X \times X)\setminus \Delta$, which is an open set by hypothesis. Then by definition of product topology there must be two open sets in $X$,  $U \ni x$ and $V \ni y$. Suppose there is a point in the intersection $z \in U \cap V$. Then $(z,z) \in (U \times V) \cap \Delta$, a contradiction.
\end{proof}

\begin{thing4}{Exercise 1.18*}\leavevmode
	Let $\sim$ be an equivalence relation on a topological space $M$, and $\Gamma \subset M \times M$ its \textit{\textbf{graph}}, that is, the set $\{(x,y) \in M \times M|x \sim y\}$. Suppose that the map $M \longrightarrow M/\sim$ is open, and that $\Gamma$ is closed in $M \times M$. Show that $M/\sim$ is Hausdorff.

	\textbf{Hint} . Prove that diagonal is closed in $M \times M$.
\end{thing4}

\begin{proof}[Solution]\leavevmode
Notice that any open surjective map is closed: let $f:X \twoheadrightarrow Y$ be an open surjective map and $F\subset X$ closed, then $f(X\setminus F)=f(X)\setminus f(F)=Y\setminus f(F)$.

Our objective is to show that the diagonal $\tilde{\Delta}$ in $(M/\sim)\times(M/\sim)$ is closed. The projection of the graph $\Gamma$ is $\tilde{\Delta}$. Since $\Gamma$ is closed, by the remark above it follows that $\tilde{\Delta}$ is closed in $(M/\sim)\times(M/\sim)$ as we needed.
\end{proof}

\section{Compact spaces}

\begin{thing3}{Definition 1.12}\leavevmode
	A \textit{\textbf{cover}} of a topological space $M$ is a collection of open subsets $\{U_\alpha \in 2^M\}$ such that $\bigcup U_\alpha=M $. A \textit{\textbf{subcover}} of a cover $\{U_\alpha\}$ is a subset $\{U_\beta\}\subset \{U_\alpha\}$. A topological space is called \textit{\textbf{compact}} if any cover of this space has a finite subcover.
\end{thing3}

\begin{thing4}{Exercise 1.22}[Closed subset of compact is compact]\label{exer:1.22}\leavevmode
	Let $M$ be a compact topological space, and $Z \subset M$ a closed subset. Show that $Z$ is also compact.
\end{thing4}

\begin{proof}[Solution]\leavevmode
Choose a cover $\{U_\alpha\}$ of $Z$. Complete to a cover $\{U_\alpha\}\cup (M\setminus Z)$ of $M$ since $M\setminus Z$ is open by hypothesis. Since $M$ is compact then there is a finite subcover $\{U_\beta\}$ of $M$. This is also a finite subcover of $Z$.
\end{proof}

\begin{thing4}{Exercise 1.23}[Countable metrizable $\implies $ contains convergent subseq. or is discrete]\label{exer:1.23}\leavevmode
	Let $M$ be a countable, metrizable topological space. Show that either $M$ contains a converging sequence of pairwise different elements, or $M$ is discrete.
\end{thing4}

\begin{proof}[Solution]\leavevmode
Suppose $M$ is not discrete. Then there is a point $z_0$ such that $\{z_0\}$ is not an open set. Then every open set containing $z_0$ contains another point. Choose for every $n \in \mathbb{N}$ a point $z_n$ different from $z_0$ inside the a ball $B_{1/n}(z_0)$. Taking a subsequence if necessary, we obtain a sequence of pairwise different elements $\{z_i\}$ converging to $z_0$.

If $M$ is discrete, it's clear that it cannot have a convergent sequence of pairwise disjoint elements: if the limit point $\{z_0\}$ was open,  $M\setminus \{z_0\}$ would be closed and thus it would contain all its limit points!
\end{proof}

\begin{thing3}{Definition 1.13}\leavevmode
	A topological space is called \textit{\textbf{sequentially compact}} if any sequence $\{z_i\}$ of points of $M$ has a converging subsequence.
\end{thing3}

\begin{thing4}{Exercise 1.24}[Metrizable compact $\implies $ sequentially compact]\label{exer:1.24}\leavevmode
	Let $M$ be a metrizable compact topological space. Show that $M$ is sequentially compact.
\end{thing4}

\begin{proof}[Solution]\leavevmode
Let $\{z_i\}$ be a sequence. Since the restricition of a metric to a subset is also a metric, we may use Exercise \hyperref[exer:1.23]{1.23} on the countable metric subspace $\{z_i\}$. Suppose by contradiction that $\{z_i\}$ has no limit point in $M$. In particular it has no limit point in $\{z_i\}$, so by Exercise \hyperref[exer:1.23]{1.23} it is discrete. Then there are neighbourhoods $U_i \ni z_i$ such that $U_i \cap\{z_j\}_{j \neq  i}=\varnothing$. Then $\{U_i\} \cup  (M\setminus \{U_i\})$ is an open cover of $M$, which has a finite subcover. By the pigeon principle, at least one of the $U_i$ contains an infinite number of points in $\{z_i\}$, which is not possible.
\end{proof}
\iffalse
\begin{thing4}{Exercise 1.25*}\label{exer:1.25}\leavevmode
Construct an example of a Hausdorff topological space which is sequentially compact, but not compact.
\end{thing4}

\begin{thing4}{Exercise 1.26*}\label{exer:1.26}\leavevmode
Construct an example of a Hausdorff topological space which is compact, but not sequentially compact.
\end{thing4}

\begin{thing3}{Definition 1.14}\leavevmode
	A \textit{\textbf{topological group}} is a topological space with group operations $G \times G \longrightarrow G$, $x,t \mapsto xy$ and $G\longrightarrow G$, $x \mapsto  x^{-1}$ which are continuous. In a similar way, one defines \textit{\textbf{topolocial vector spaces}}, \textit{\textbf{topological rings}} and so on.
\end{thing3}

\begin{thing4}{Exercise 1.27*}\label{exer:1.27}\leavevmode
Let $G$ be a compact topological group, acting on a topological space $M$ in such a way that the map $M \times G \longrightarrow M$ is continuous. Prove that the quotient space is Hausdorff.
\end{thing4}

\begin{proof}[Solution]\leavevmode
\end{proof}\fi


\begin{thing4}{Exercise 1.28}[Continuous function maps compact to compact]\label{exer:1.28}\label{exer:1.28}\leavevmode
Let $f:X\to Y$ be a continuous map of topological spaces with $X$ compact. Prove that $f(X)$ is also compact.
\end{thing4}

\begin{proof}[Solution]\leavevmode
	Choose an open cover $\{U_\alpha\}$ of $f(X)$. Then  $\left\{f^{-1}\left( U_\alpha \right) \right\}$ is an open cover of $X$ since $f$ is continuous, and thus it has an open subcover $\{f^{-1}(U_\beta)\}$. I claim that $\{U_\beta\}$ is a cover of $f(X)$: if there was a point $f(x) \not \in \bigcup U_\beta $, then $x$ couldn't be in any of the  $f^{-1}(U_\beta)$, which cover $X$. 
\end{proof}

\begin{thing4}{Exercise 1.29}[Compact subset of Hausdorff is closed]\label{exer:1.29}\label{exer:1.29}\leavevmode
Let $Z \subset Y$ be a compact subset of a Hausdorff topological space. Prove that it is closed.
\end{thing4}

\begin{proof}[Solution]\leavevmode
Recall that a set is closed if it contains all its limit points (any point that is not a limit point has a neighbourhood not intersecting the set, making the complement open).

Let $z_0$ be a limit point of $Z$. Choose for every point $z \in Z$ neighbourhoods $U_z \ni z$ and $V_z \ni z_0$ such that $U_z \cap V_z=\varnothing$. If $z_0 \not\in Z$, then $\{U_z\}$ is an open cover of $Z$, so there exists a finite subcover $U_{z_1},\ldots,U_{z_n}$. The set $\bigcup_{i=1}^n V_{z_i}$ is an open neighbourhood of $z_0$ that does not intersect $Z$, a contradiction.
\end{proof}

\begin{thing4}{Exercise 1.30}\label{exer:1.30}\leavevmode
Let $f:X \to Y$ be a continuous, bijective map of topological spaces, with $X$ compact and $Y$ Hausdorff. Prove that it is a homeomorphism.
\end{thing4}

\begin{proof}[Solution]\leavevmode
We need to see that $f^{-1}$ is continuous, i.e. that $(f^{-1})^{-1}(U)$ is open for any $U \subset Y$ open. Since $f$ is bijective,  $(f^{-1})^{-1}(U)=f(U)$; so we must check $f$ is open. Equivalently, we can check $f$ is closed:  if $f(F)$ is closed for any closed $F \subset X$, then for any open set $U\subset X$, we see $f(X\setminus U)=Y\setminus f(U)$ is closed.

To see $f$ is closed note that since $X$ is compact and  $f$ is bijective, $f(X)=Y$ is also compact by  Exercise \hyperref[exer:1.28]{1.28}. By Exercise \hyperref[exer:1.22]{1.22} a closed subset $F$ of $X$ is compact. Again by continuity, $f(F)$ is compact in  $Y$. Finally by Exercise \hyperref[exer:1.29]{1.29}, since $Y$ is Hausdorff and $f(F)$ is compact, it must be closed.
\end{proof}


\begin{thing3}{Definition 1.15}\leavevmode
	A topological space $M$ is called \textit{\textbf{pseudocompact}} if any continuous function $f:M\to \mathbb{R}$ is bounded.
\end{thing3}

\begin{thing4}{Exercise 1.31}\label{exer:1.31}\leavevmode
Prove that any compact topological space is pseudocompact.
\end{thing4}

\begin{proof}[Solution]\leavevmode
We must show that any continuous function $f:M \to \mathbb{R}$ is bounded, in the sense that its image contained in a ball of finite radius (c.f. Exercise \hyperref[exer:1.33]{1.33}). The image of any such function is compact by Exercise \hyperref[exer:1.28]{1.28}. But compact sets of $\mathbb{R}$ are bounded: if for every $r>0$, the image  $f(X)$ is not contained in the ball of radius $r$ centered at zero, $B_r(0)$, then $\{B_r(0)\}$ is an open cover of $f(X)$ (since its union is all of  $\mathbb{R}$) without a finite subcover.
\end{proof}

\begin{thing4}{Exercise 1.32}\label{exer:1.32}\leavevmode
Show that for any continuous function $f:M\longrightarrow \mathbb{R}$ on a compact space there exists $x \in M$ such that  $f(z)=\operatorname{sup}_{z \in  M} f(z)$.
\end{thing4}

\begin{proof}[Solution]\leavevmode
As in Exercise \hyperref[exer:1.31]{1.31}, the image of $f$ is a bounded set of $ \mathbb{R}$, which means the supremum is well-defined. To see it is attained at a point in $M$ notice that $f(M)$ is compact and thus closed. This means that all its limit points belong to  $f(M)$: if a limit point $z_0$ of the closed set $f(M)$ is not in $f(M)$, then $z_0$ has no neighbourhood contained in $\mathbb{R}\setminus f(M)$, but $\mathbb{R}\setminus f(M)$ is open. In particular  $\operatorname{sup}_{z \in M}f(z)$ is in $f(M)$.
\end{proof}

\begin{thing4}{Exercise 1.33}\label{exer:1.33}\leavevmode
Consider $\mathbb{R}^n$ as a metric space, with the standard (Euclidean) metric. Let $Z \subset \mathbb{R}^n$ be a closed, bounded set (\textit{\textbf{bounded}} means contained in a ball of finite radius). Prove that $Z$ is sequentially compact.
\end{thing4}

\begin{proof}[Solution]\leavevmode
First consider the case $n=1$. If  $\{z_n\}$ is a sequence contained in a closed and bounded set $Z$, then  $\operatorname{sup}_{n \in N}z_n$ is a well-defined element in $Z$. Taking balls of radius $1/m$ we construct a subsequence of  $\{z_n\}$ converging to the $\operatorname{sup}$.

Now let's show that $[0,1]^n \subset\mathbb{R}^n$ is also sequentially compact. Chose a sequence $\{z_n\}\subset [0,1]^n$. This gives a sequence in every coordinate, each of which must have a convergent subsequence, {\color{2}but this doesn't give a convergent subsequence in the product…}
\end{proof}

\section{Topological manifolds}

\begin{thing4}{Exercise 2.3}\label{exer:2.3}\leavevmode
Consider the quotient of  $\mathbb{R}^2$ by the action of $\{\pm  1\}$ that maps $x$ to  $-x$. Is the quotient space a topological manifold?
\end{thing4}

\begin{proof}[Solution]\leavevmode
	Yes (it's not \textit{smooth} but it is topological). A homeomorphism between $\mathbb{R}^2/\{\pm 1\}$ and $\mathbb{R}^2$ is $re^{i\theta}\mapsto re^{i2\theta}$ where the angle $\theta$ in the domain is in the interval  $[0,\pi)$ and  $r\in [0,\infty)$. This map is clearly bijective (any equivalence class has a unique representative of the given form) and its continuous inverse is $re^{i\varphi}\mapsto re^{i\varphi/2}$ for $\varphi \in [0,2\pi)$.
\end{proof}



\begin{thing4}{Exercise 2.7}\label{exer:2.7}\leavevmode
Show that $\mathbb{R}^2/\mathbb{Z}^2$ is a manifold.
\end{thing4}

\begin{proof}[Solution]\leavevmode
Let $\bar{z}$ be a point in $\mathbb{R}^2/\mathbb{Z}^2$. Its preimage is the lattice $\{z+(a,b):a,b \in \mathbb{Z}\}$. A ball of radius  $\frac{1}{2}$ centered at any representative of $\bar{z} $ contains only one representative of any other class, so that the restriction of the projection is bijective (and continuous by definition of quotient topology). The inverse map is also continuous by definition of quotient topology: an open set in the ball on $\mathbb{R}^2$ is mapped to an open set in the quotient because its preimage is open.
\end{proof}

\section{Smooth manifolds}

\begin{thing3}{Definition 2.2}\leavevmode
	A \textit{\textbf{cover}} of a topological space $X$ is a family of open sets $\{U_i\}$ such that $\bigcup_{i} U_i=X$. A cover $\{V_i\}$ is a \textit{\textbf{refinement}} of a cover $\{U_i\}$ if every $V_i$ is contained in some $U_i$.
\end{thing3}

\begin{thing4}{Exercise 2.11}\label{exer:2.11}\leavevmode
Show that any two covers of a topological space admit a common refinement.
\end{thing4}

\begin{proof}[Solution]\leavevmode
Let $\{U_i\}$ and $\{U_i'\}$ be covers of a topological space $X$. Then  $\{V_{ij}:=U_i \cap U_j'\}$ is a common refinement. It is obvious that $V_{ij}$ is contained in $U_i$ and $U_j$, so it is a subcover of both covers. And it is also a cover: it $x \in X$ then $x$ must be in some  $U_i$ and some $U_j'$, so that it is in $V_{ij}$.
\end{proof}

\begin{thing4}{Definition 2.3}\leavevmode
	A cover $\{U_i\}$ is an \textit{\textbf{atlas}} if for every $U_i$ we have a map $\varphi_i:U_i\to \mathbb{R}^n$ giving a homeomorphism of $U_i$ with an open subset in  $\mathbb{R}^n$. The \textit{\textbf{transition maps}} 
	\[\phi_{ij}:\varphi_i(U_i\cap U_j)\to \varphi_j(U_i \cap U_j)\]
	are induced by the above homeomorphisms. An atlas is \textit{\textbf{smooth}} if all transition maps are smooth (of class $C^\infty$, i.e., infinitely differentiable), \textit{\textbf{smooth of class}} $C^i$ if all transition functions are of differentiability class $C^i$ and \textit{\textbf{real analytic}} if all transition maps admita a Taylor expansion at each point.
\end{thing4}

\begin{thing3}{Definition 2.4}\leavevmode
A \textit{\textbf{refinement of an atlas}} is a refinement of the corresponding cover $V_i \subset U_i$ equipped with the maps $\varphi_i:V_i\to \mathbb{R}^n$ that are the restricitions of $\varphi_i:U_i \to \mathbb{R}^n$. Two atlases $(U_i,\varphi_i)$ and $(U_i, \psi_i)$ of class $C^\infty$ of $C^i$ (with the same cover) are \textit{\textbf{equivalent}} in this class if, for all $i$, the map $\psi_i \circ \varphi_i^{-1}$ defined on the corresponding open subset in $\mathbb{R}^n$ belongs to the mantioned class. Two arbitratry atlases are \textit{\textbf{equivalent}} if the corresponding covers possess a common refinement giving equivalent atlases.
\end{thing3}

\begin{thing3}{Definition 2.5}\leavevmode
	A \textit{\textbf{smooth structure}} on a manifold (of class $C^\infty$ of $C^i$) is an atlas of class $C^\infty$ or $C^i$ considered up to the above equivalence. A \textit{\textbf{smooth manifold}} is a topological manifold equipped with a smooth structure.
\end{thing3}

\begin{thing5}{Remark 2.3}\leavevmode
	Terrible, isn't is?
\end{thing5}

\begin{thing3}{Definition 2.6}\leavevmode
	A \textit{\textbf{smooth function}} on a manifold $M$ is a function $f$ whose restriction to the chart $(U_i,\varphi_i)$ gives a smooth function $f \circ \varphi_i^{-1}:\varphi_i(U_i)\to\mathbb{R}$ for each open subset $\varphi_i(U_i) \subset \mathbb{R}^n$.
\end{thing3}

\begin{thing5}{Remark 2.4}\leavevmode
	There are several ways to define a smooth manifold. The above way is most standard. It is not the most convenient one but you should know it. Two other ways (via sheaves of functions and via Whitney's theorem) are presented further in these handouts.
\end{thing5}

\begin{thing3}{Definition 2.7}\leavevmode
	A \textit{\textbf{presheaf of functions}} on a topological space $M$ is a collection of subrings $\mathcal{F}(U) \subset C(U)$ in the ring $C(U)$ of all functions on $U$, for each open subset $U \subset M$, such that the restriction of every $\gamma \in \mathcal{F}(U)$ to an open subset $U_1 \subset U$ belongs to $\mathcal{F}(U_1$.
\end{thing3}

\begin{thing3}{Definition 2.8}\leavevmode
	A presheaf of functions $\mathcal{F}$ is called a \textit{\textbf{sheaf of functions}} if these subrings satisfy the following condition. Let $\{U_i\}$ be a cover of an open subset $U\subset M$ (possibly infinite) and $f_i \in \mathcal{F}(U_i)$ a family of functions defined on the open sets of the cover and compatible on the pairwise intersections:
	\[f_i|_{U_i\cap U_j}=f_j|_{U_i \cap U_j}\]
	for every pair of memebers of the cover. Then there exists $f \in \mathcal{F}(U)$ such that $f_i$ is the restriction of $f$ to $U_i$ for all $i$.
\end{thing3}

\begin{thing5}{Remark 2.5}\leavevmode
	A \textit{presheaf of functions} is a collection of subrings of functions on open subsets, compatible with restrictions. A \textit{sheaf of functions} is a presheaf allowing "gluing" of a function on a bigger open set if its restriction to smaller open sets lies in the presheaf.
\end{thing5}

\begin{thing3}{Definition 2.9}\leavevmode
	A sequence $A_1 \longrightarrow A_2 \longrightarrow A_3 \longrightarrow \ldots$ of homomorphisms of abelian grous or vector spaces is called \textit{\textbf{exact}} if the image of each map is the kernel of the next one.
\end{thing3}

\begin{thing4}{Exercise 2.13}\label{exer:2.13}\leavevmode
Let $ \mathcal{F}$ be a presheaf of functions. Show that $\mathcal{F}$ is a sheaf if and only if for every open cover $\{ U_i\}$ of an oen subset $U\subset M$ the sequence of restriction maps
\[\begin{tikzcd}0\arrow[r]&\mathcal{F}(U)\arrow[r,"\varphi_1"]&\prod_{i}\mathcal{F}(U_i)\arrow[r,"\varphi_2"]&\prod_{i \neq  j}\mathcal{F}(U_i \cap U_j)\end{tikzcd}\]
 is exact, with $\eta \in \mathcal{F}(U_i)$ mapped to $\eta|_{U_i \cap U_j}$ and $-\eta|_{U_j \cap U_i}$.
\end{thing4}

\begin{proof}[Solution]\leavevmode
The key observation is that elements of $\ker \varphi_2$ are collections of functions $f_i \in \mathcal{F}(U_i)$ satisfying compatibility in pairwise intesections, i.e.,
\[f_i|_{U_i \cap U_j}=f_j|_{U_i \cap U_j}.\]
{\color{2}To achieve this} I think we must define $\varphi_2$ by
\[(\ldots,\quad f_i,\quad \ldots,\quad \quad f_j,\quad \ldots)\longmapsto (\ldots,\quad f_i|_{U_i \cap U_j}-f_j|_{U_i \cap U_j},\quad \ldots).\]
Then elements in $\ker \varphi_2$ satisfy the desired compatibility condition.
\iffalse
{\color{2}Indeed:} if $f_i \in \mathcal{F}(U_i)$ and $f_j\in \mathcal{F}(U_j)$ are coordinates of some $F \in \prod_{i}\mathcal{F}(U_i)$, the image of $F$ under  $\varphi_2$ is of the form
\[(\ldots,\;\;\;f_i|_{U_i \cap U_j},\;\;\;\ldots,\;\;\;-f_i|_{U_j \cap U_i},\;\;\;\ldots,\;\;\;f_j|_{U_i \cap U_j},\;\;\;\ldots,\;\;\;-f_j|_{U_j \cap U_i},\;\;\;\ldots)\]
which we assume to be the zero element of $\prod_{i \neq j}\mathcal{F}(U_i \cap U_j)$.\fi

$(\implies )$ Suppose the the sequence above is exact. To show $\mathcal{F}$ is a sheaf fix $f_i \in \mathcal{F}(U_i)$ for every $i$ satisfying $f_i|_{U_i \cap U_j} = f_j |_{U_i \cap U_j}$ for every $i,j$. This is equivalent to choosing an element in $\ker \varphi_2$. By exactness this element is in $\operatorname{img} \varphi_1$. This means that $f_i$ is the restriction of some $f \in \mathcal{F}(U)$ for every $i$ as desired.

$(\impliedby)$ Suppose $\mathcal{F}$ is a sheaf. Injectivity of $\varphi_1$ is immediate: if $f \in \mathcal{F}(U)$ is mapped to zero under $\varphi_1$, meaning $f_i|_{U_i}=0$ for all $i$, it must be zero since $\{U_i\}$ is a cover. Exactness in the second ring is equivalent to the definition of sheaf by the remarks above.
\end{proof} 

\begin{thing4}{Exercise 2.14}\label{exer:2.14}\leavevmode
Show that the following spaces of functions on $\mathbb{R}^n$ define sheaves of functions.
\begin{enumerate}[label=(\alph*)]
\item Space of continuous functions.
\item Space of  smooth functions.
\item Space of functions of differentiability class $C^i$.
\item (*) Space of functions which are pointwise limits of sequences of continuous functions.
\item Space of functions vanishing outside a set of measure 0.
\end{enumerate}
\end{thing4}

\begin{proof}[Solution]\leavevmode
Injectivity of $\varphi_1$ is immediate in all cases: a function that vanishes on every subset of an open cover vanishes identically.
\begin{enumerate}[label=(\alph*)]
\item Define a global function $f$ on $U$ by  $x \mapsto f_i(x)$ for any $f_i \in \mathcal{F}(U_i)$ such that $x \in U_i$. Continuity follows from continuity of $f_i$, and the fact that $f$ is well-defined follows from the gluing condition of $\mathcal{F}$.

\item Like above: smoothness follows from smoothness of $f_i$.

\item Like above.

\item 

\item {\color{2}Not sure} (uncountable union of measure-zero sets may have positive measure).
\end{enumerate}
\end{proof}

\begin{thing4}{Exercise 2.15}\label{exer:2.15}\leavevmode
Show that the following spaces of functions on $\mathbb{R}^n$ are presheaves, but not sheaves
\begin{enumerate}[label=(\alph*)]
\item Space of constant functions.
\item Space of bounded functions.
\item Space of functions vanishing outside of a bounded set.
\item Space of continuous functions with finite  $\int |f|$.
\end{enumerate}
\end{thing4}

\begin{proof}[Solution]\leavevmode
The presheaf condition, that the restriction of a function to 
	\begin{enumerate}[label=(\alph*)]
\item Open sets with two connected components may not glue to a global constant function.

\item Unbounded functions may be bounded in open subsets! Take the open set $(0,\infty)\subset \mathbb{R}$ and the cover $U_i=(1/i,\infty)$. Define the bounded function $f_i(x)=1/x$ in every $U_i$.

	In $\mathbb{R}^n$ we may do the same trick using the half-spaces with bounded last coordinate $U_i=\{(x_1,\ldots,x_n):x_n \geq 1/i\}$ and taking  $f_i(x)=1/\|x\|$.

\item Let the open set $U$ be all of $\mathbb{R}^n$. An open cover is given by balls $B_i$ of radius $2/3$ with center in  $i \in\mathbb{Z}^n$.  For every $i \in \mathbb{Z}^n$ define functions $f_i$ that vanish only outside a ball of very small radius, say $1/6$, with center in $i$. These functions coincide (they vanish) in the intersections of the cover, but the function obtained by gluing cannot vanish outside any bounded set: it is non-zero in the union of balls of radius $1/6$ with centers in  $\mathbb{Z}^n$.

\item Item (c) works if we manage to make the functions continuous. This can be done using a {\color{2}partition of unity}. Also we must require that the values of the integrals in the smaller balls of radius $1/6$ do not tend to zero (this way the global integral is an infinite sum of numbers that do not tend to zero, so it cannot be finite).
\end{enumerate}
\end{proof}

\end{document}
