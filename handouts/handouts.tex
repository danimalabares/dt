\input{/Users/daniel/github/config/preamble.sty}%This is available at github.com/danimalabares/config
\input{/Users/daniel/github/config/thms-eng.sty}%This is available at github.com/danimalabares/config
\usepackage{multicol}
%\usepackage[style=authortitle-terse,backend=bibtex]{biblatex}
%\addbibresource{/Users/daniel/github/config/bibliography.bib}

\begin{document}

\begin{minipage}{\textwidth}
	\begin{minipage}{1\textwidth}
	Handouts by Misha Verbitsky	 \hfill Daniel González Casanova Azuela
		
		{\small \href{http://verbit.ru/IMPA/METRIC-2023/}{verbit.ru/IMPA/METRIC-2023/} \hfill\href{https://github.com/danimalabares/dt}{github.com/danimalabares/dt}

		 \href{http://verbit.ru/MATH/GEOM-2013/}{verbit.ru/MATH/GEOM-2013}}
	\end{minipage}
\end{minipage}\vspace{.2cm}\hrule

\vspace{10pt}
{\huge Practice exercises on smooth manifolds}

A pdf file with the questions may also be found \href{https://github.com/danimalabares/dt/blob/main/handouts/questions.pdf}{here}.
\tableofcontents

\section{Remedial topology}

\subsection{Topological spaces}

\begin{thing3}{Definition 1.1}\leavevmode
	A set of all subsets of $M$ is denoted $2^M$. \textit{\textbf{Topology}} on $M$ is a collection of subsets $S \subset 2^M$ called \textit{\textbf{open subsets}}, and satifsying the following conditions:
	\begin{enumerate}
	\item Empty set and $M$ are open
	\item A union of any number of open sets is open
	\item An intersection of a finite number of open subsets is open.
	\end{enumerate}
	A complement of an open set is called \textit{\textbf{closed}}. A set with topology on it is called a \textit{\textbf{toplogical space}}. An \textit{\textbf{open neighbourhood}} of a point is an open set containig this point.
\end{thing3}

\begin{thing3}{Definition 1.2}\leavevmode
	A map $\phi:M\to M'$ of topological spaces is called \textit{\textbf{continuous}} if a preimage of each open set $U \subset M'$ is open in $M$. A bijective cintunuous map is called a \textit{\textbf{homeomorphism}} if its inverse is also continuous.
\end{thing3}

\begin{thing4}{Exercise 1.1}\leavevmode
	Let $M$ be a set, and $S$ a set of all subsets of $M$. Prove that $S$ defines a topology on $M$. This topology is called \textit{\textbf{discrete}}. Describe the set of all continuous maps form $M$ to a given topological space.
\end{thing4}

\begin{proof}[Solution]\leavevmode
Since all sets are open, topology axioms are satisfied by $S$. All maps from $M$ to a given topological space are continuous.
\end{proof}

\begin{thing4}{Exercise 1.2}\leavevmode
	Let $M$ be a set, and $S \subset 2^M$ a set of two subsets: empty set and $M$. Prove that $S$ defines a topology on $M$. This topology is called \textit{\textbf{codiscrete}}. Describe the set of all continuous maps from $M$ to a space with discrete topology.
\end{thing4}

\begin{proof}[Solution]\leavevmode
It's trivial that $S$ satisfies the axioms of topology. Continuous maps from $M$ to a space with discrete topology are constant maps. Such maps are continuous since the preimage of any (open) set is open: either it contains the value of the map at all points, in which case the preimage is all of $M$, or it doesn't, in which case the preimage is empty. Conversely, if a map has more than one value, the preimage of such a value cannot be neither $M$ nor the empty set.
\end{proof}

\begin{thing3}{Definition 1.3}\leavevmode
	Let $M$ be a topological space, and $Z \subset M$ its subset. \textit{\textbf{Open subsets}} of $Z$ are subsets obtained as $Z \cap U$, where $U$ is open in $M$. This topology is called \textit{\textbf{induced topology}}.
\end{thing3}

\begin{thing3}{Definition 1.4}\leavevmode
	A \textit{\textbf{metric space}} is a set $M$ equipped with a \textit{\textbf{distance function}} $d: M \times M \longrightarrow \mathbb{R}^{\geq 0}$ satisfying the following acioms.
	\begin{enumerate}
	\item $d(x,y)=0$ iff $x=y$.
	\item $d(x,y)=d(y,x)$.
	\item (triangle inequality) $d(x,y)+d(y,z) \geq d(x,z)$.
	\end{enumerate}
An \textit{\textbf{open ball}} of raduis $r$ with center in xx is $\{y \in M: d(x,y)<r\}$.
\end{thing3}

\begin{thing3}{Definition 1.5}\leavevmode
	Let $M$ be a metric space. A subset $U \subset M$ is called \textit{\textbf{open }} if it is obtained as a union of open balls. This topology is called \textit{\textbf{induced by the metric}}.
\end{thing3}

\begin{thing3}{Definition 1.6}\leavevmode
A topological space is called \textit{\textbf{metrizable}} if its topology can be induced by a metric.	
\end{thing3}

\begin{thing4}{Exercise 1.3}\leavevmode
Show that discrete topology can be induced by a metric, and codiscrete cannot.	
\end{thing4}

\begin{proof}[Solution]\leavevmode
To induce the discrete metric define the distance between any two distinct points to be 1. This clearly satisfies the three axioms of metric, and the ball of radius $1/2$ is an open set that contains only its center, making any point and thus any subset an open set.

If a metric space contains at least two points at distance $d$, the ball with radius $d/2$ at any of these points is an open set distinct from the empty set and the total, so the topology induced by the metric cannot be discrete.
\end{proof}

\begin{thing4}{Exercise 1.4}\leavevmode
	Prove that an intersection of any collection of closed subsets of a topological space is closed.
\end{thing4}

\begin{proof}[Solution]\leavevmode
As I recall this is due to de Morgan laws stating that for any collection $F_\alpha$ of subsets
\begin{equation}\label{eq:1}\left( \bigcap_\alpha F_\alpha  \right)^c=\bigcup_{\alpha} F^c\end{equation}
where superscript $c$ means set complement. If this is true then we are done because if $F_\alpha$ are closed, we see that the intersection is also closed as its complement is open.

\end{proof}

\begin{thing3}{Definition 1.7}\leavevmode
	An intersection of all closed supersets $Z \subset M$ is called \textit{\textbf{closure}} of $Z$
\end{thing3}

\begin{thing3}{Definition 1.8}\leavevmode
	A \textit{\textbf{limit point}} of a set  $Z \subset M$ is a point $x \in M$ such that any neighbourhood of $M$ contains a point of $Z$ other than $x$. A \textit{\textbf{limit}} of a sequence $\{ x_i\}$ of points in $M$ is a point $x \in M$ such that any neighbourhood of  $x \in M$ contains all $x_i$ for ll $i$ except a finite number. A sequence which has a limit is called \textit{\textbf{convergent}}.
\end{thing3}

\begin{thing4}{Exercise 1.5}\label{exer:1.5}\leavevmode
	Show that a closure of a set $Z \subset M$ is a union of $Z$ and all its limit points.
\end{thing4}

\begin{proof}[Solution]\leavevmode
	It's enough to show that the union of $Z$ and all its limits points $W$ is a closed set and that it is contained in any closed set containing $Z$.

	To see $W$ is closed chose a point in its complement $p \in W^c$. Since $p$ is not a limit point of $Z$ nor a point of $Z$, there is a neighbourhood of $p$ not intersecting $Z$. This means that such neighbourhood is contatined in $W^c$. We can do this for all points in $W^c$, thus obtaining a $W^c$ as a union of open sets, which is open, and then $W$ is closed.

	To see $W$ is contained in any closed set containing $Z$, suppose $F$ contains $Z$ but not $W$. Then there must be a limit point of $Z$ that is not in $F$. But then $F$ cannot be closed because there is no neighbourhood of such a limit point contained in $F^c$, which should be open. Indeed, if $F^c$ is open then every point contains a neighbourhood contained in $F^c$: at least $F^c$ itself!
\end{proof}

\begin{thing4}{Exercise 1.6}\label{exer:1.6}\leavevmode
Let \(f:M \to M'\) be a continuous map of topological spaces. Prove that \(f\left( \lim_{i} x_i \right) =\lim_{i} f(x_i)\) for any convergent sequence \(\{x_i \in M\}\).
\end{thing4}

\begin{proof}[Solution]\leavevmode
Let \(U\) be any neighbourhood of the point \(f\left( \lim_{i} x_i \right) \). Then \(f^{-1}(U)\) is a neighbourhood of \(\lim_{i} x_i\), so it must intersect all but a finite number of the \(x_i\). Then all but a finite number of the \(f(x_i)\) must intersect \(U\), meaning \(f\left( \lim_{i} x_i \right) =\lim_{i} f(x_i)\).
\end{proof}

\begin{thing4}{Exercise 1.7}\label{exer:1.7}\leavevmode
Let $f:M\to M'$ be a map of metrizable topological spaces, such that $f \left( \lim_{i} x_i \right) =\lim_{i} f(x_i)$ for any convergent sequence $\{x_i \in M\}$. Prove that $f$ is continuous.	
\end{thing4}

\begin{proof}[Solution]\leavevmode
It is equivalent that the preimage of every open set is open (definition of $f$ being continuous) with the preimage of every closed subset is closed: for any closed set $M'\setminus U$ with $U$ open, $f^{-1}(M\setminus U)=f^{-1}(M')\setminus f^{-1}(U)$ is closed.

Consider the closed set $F\subset M'$ and let's check that its preimage is also closed. By the same reasoning as in Exercise \hyperref[exer:1.5]{1.5}, to show closedeness it's enough to show the set contains all its limit points. Take a limit point $p\in f^{-1}(F)$. We construct a convergent sequence $\{x_n\}$ taking balls of radius $\frac{1}{n}$ around $p$, each of which must contain a point in $f^{-1}(F)$. This gives a sequence in $F$, which by hypothesis must converge to a limit point $\lim_{i} f(x_i)=f\left(\lim_i x_i\right) \in F$. This means $p=\lim_{i} x_i$ is in the inverse image of $F$.
\end{proof}

\begin{thing4}{Exercise 1.8*}\leavevmode
	Find a counterexample to the previous problem for non-metrizable, Hausdorff topological spaces (see the next subsection of a definition of Hausdorff).
\end{thing4}

\begin{proof}[{\color{2}Sketch of solution}]\leavevmode
Probably Sørgenfrey line is a counter-example? I should look for its definition to make sure it is Hausdorff (and how is it defined exactly---I think open sets are positive rays).
\end{proof}

\begin{thing4}{Exercise 1.9**}\leavevmode
	Let $f: M\longrightarrow M'$ be a map of countable topological spaces, such that $f(\lim_{i} x_i)=\lim_{i} f(x_i)$ for any convergent sequence $\{ x_i \in M\}$. Prove that $f$ is continuous, or find a counterexample.
\end{thing4}

\begin{proof}[{\color{2}Sketch of solution}]\leavevmode
	Is a \textit{\textbf{countable space}} a space whose cardinality is $\mathbb{N}$? What are the possible topologies on $\mathbb{N}$? Discrete topology gives that every map is continuous. Other topologies are maybe, again, rays.
\end{proof}

\begin{thing4}{Exercise 1.10*}\leavevmode
	Let $f:M\longrightarrow N$ be a bijective map inducing homeomorphisms on all countable subsets of $M$. Show that it is a homeomorphism, or find a counterexample.
\end{thing4}

\begin{proof}[{\color{2}Sketch of solution}]\leavevmode
	If we suppose that $M$ and $M'$ are metrizable, we can use Exercise \hyperref[exer:1.7]{1.7} as follows. Choose any convergent sequence $\{x_i \in M\}$. Then the countable set  $\{x_i\}\cup \{\lim_{i} f(x_i)\}$ is mapped homeomorphically to $\{f(x_i)\}\cup  \{f(\lim_{i} x_i)\}$. This implies that $f\left( \lim_{i} f(x_i) \right) =f \left( \lim_{i} x_i \right)$, so $f$ is continuous. The same holds for $f^{-1}$, so $f$ is a homeomorphism.

	Probably the statement isn't true in general, so let's look for a counter-example.
\end{proof}

\subsection{Hausdorff spaces}

\begin{thing3}{Definition 1.9}\leavevmode
	Let $M$ be a topological space. It is called \textit{\textbf{Hausdorff}} or \textit{\textbf{separable}}, if any two distinct points $x \neq  y \in M$ can be \textit{\textbf{separated}} by open subsets, that is, there exist open neighbourhoods $U \ni x$ and $V \ni y$ wuch that $U\cap V= \varnothing$.
\end{thing3}

\begin{thing5}{Remark 1.1}\leavevmode
	In topology, the Hausdorff axiom is usually assumed by default. In subsequent handouts, it will be always assumed (unless stated otherwise).
\end{thing5}

\begin{thing4}{Exercise 1.11}\label{exer:1.11}\leavevmode
	Let $M$ be a Hausdorff topological space. Prove that all points in $M$ are closed subsets.
\end{thing4}

\begin{proof}[Solution]\leavevmode
Fix a point $x \in M$. For every $y \in M$ distinct from $x$ we have the neighbourhoods $U_y \ni x$ and $V_y \ni y$ with $U_y \cap V_y = \varnothing$. Then $X\setminus\{x\}=\bigcup_{y \neq x} V_y$, which is open.
\end{proof}

\begin{thing4}{Exercise 1.12}[Points are closed in Hausdorff]\label{exer:1.12}\leavevmode
Let $M$ be a Hausdorff topological space. Prove that all points in $M$ are closed subsets.
\end{thing4}

\begin{proof}[Solution]\leavevmode
Let $x \in M$ and let's see that $M\setminus \{x\}$ is open. Choose a point $y \in M\setminus \{ x\}$. Then there are open sets $U \ni x$ and $V \ni y$ such that $ U \cap V= \varnothing$. Then $V \subset M\setminus \{ x\}$.
\end{proof}

\begin{thing4}{Exercise 1.13}\leavevmode
	Let $M$ be a topological space, with all points of $M$ closed. Prove that $M$ is Hausdorff, or find a counterexample.
\end{thing4}

\begin{proof}[Solution]\leavevmode
No solution yet…
\end{proof}

\begin{thing4}{Exercise 1.14}\leavevmode
	Count the number of non-isomorphic topologies on a finite set of 4 elements. How many of these topologies are Hausdorff.
\end{thing4}

\begin{proof}[Solution]\leavevmode
For any set $S$ of subsets of $\{1,2,3,4\}$ we can consider the \textit{\textbf{topology generated by $S$}}, which consists of all unions and intersections of elements in $S$, along with the total space and the empty set.

For the following choices of $S$ we get non-isomorphic topologies:
\begin{multicols}{2}
\begin{enumerate}
\item  $S=\varnothing$ (codiscrete topology).
\item $S=\Big\{\{1\}\Big\}$
\item $S=\Big\{\{1\},\{2\}\Big\}$.
\item $S=\Big\{\{1\},\{2\},\{3\}\Big\}$.
\item $S=\Big\{\{1\},\{2\},\{3\},\{4\}\Big\}$

	(discrete topology).
\columnbreak\item $S=\Big\{\{1,2\}\Big\}$.
\item $S=\Big\{\{1,2\},\{3\}\Big\}$
\item $S=\Big\{\{1,2\},\{3,4\}\Big\}$.
\item $S=\Big\{\{1,2,3\}\Big\}$.
\item $S=\Big\{\{1,2,3\},\{4\}\Big\}$.
\end{enumerate}
\end{multicols}
{\color{2}There are some topologies missing…}
\end{proof}

\begin{thing4}{Exercise 1.5}[!]\label{exer:1.5}\leavevmode
	Let $Z_1,Z_2$ be nonintersecting closed subsets of a metrizable space $M$. Find open subsets $U \supset Z_1, V \supset Z_2$ which do not intersect.
\end{thing4}

\begin{proof}[Solution]\leavevmode
{\color{2}Consider the distance between $Z_1$ and $Z_2$:
\[d(Z_1,Z_2):=\operatorname{inf}\{d(z_1,z_2):z_1 \in Z_1, z_2 \in Z_2\}.\]
We must argue that $d(Z_1,Z_2) \neq 0$… but that may not hold!}
%Suppose by contradiction that $d(Z_1,Z_2) = 0$ Then for every  $n \in \mathbb{N}$ there is a pair of points $z_1^n$ and $z_2^n$ such that $d(z_1^n,z_2^n)<1/n$. {\color{2}That distance need not be zero! There must be another way to do this…}

So consider for \(z_1\in Z_1\) the number
\[d(z_1,Z_2):=\operatorname{inf}_{z_2\in Z_2}d(z_1,z_2).\]
This distance \textit{is} positive since otherwise \(z_1\) would be a limit point of \(Z_2\), which is closed, implying that \(z_1 \in Z_2\), but \(Z_1 \cap Z_2=\varnothing\).

Set
\[r_{z_1}:=\frac{d(z_1,Z_2)}{2}\]
and
\[U:=\bigcup_{z_1 \in Z_1} B_{r_{z_1}}(z_1)\]
where \(B_r(z)\) denotes the ball of radius  \(r\) with center in  \(z\).  \(V\) is defined analgously for  \(Z_2\).

We have defined two open sets \(U \supset Z_1\) and \(V \supset Z_2\). Now let's check they do not intersect. Looking for a contradiction suppose that \(z \in U \cap V\). This gives \(z_1 \in Z_1\) and \(z_2 \in Z_2\) so that
\[z \in B_{r_{z_1}}(z_1),\qquad z \in B_{r_{z_2}}(z_2),\]
which means that
\[d(z,z_1)<r_{z_1}\qquad \text{ and } \qquad d(z,z_2)<r_{z_2}.\]
By triangle inequality
\begin{align*}
d(z_1,z_2)&\leq d(z_1,z)+d(z,z_2)\\
&<r_{z_1}+r_{z_2}\\
&=\frac{d(z_1,Z_2)}{2}+\frac{d(z_2,Z_1}{2}\\
&=\dfrac{\operatorname{inf}_{z_2 \in Z_2}d(z_1,z_2')}{2}+\dfrac{\operatorname{inf}_{z_1 \in Z_1}d(z_1',z_2)}{2}\\
&\leq \frac{d(z_1,z_2)}{2}+\frac{d(z_1,z_2)}{2}=d(z_1,z_2).
\end{align*}
\end{proof}

\begin{thing3}{Definition 1.10}\leavevmode
	Let $M,N$ be topological spaces. \textit{\textbf{Product topology}} is a topology on $M \times N$, with open sets obtained as unions $\bigcup_{\alpha} U_\alpha \times V_\alpha$, where $U_\alpha$ is open in $M$ and $V_\alpha$ is open in $N$.
\end{thing3}

\iffalse\begin{remark}\leavevmode
	A closed set in product topology is a union of products of closed sets, i.e. closed sets in $M \times N$ are obtained as unions $\bigcup_{\alpha} F_\alpha \times F'_\alpha$ where $F_\alpha$ is closed in $M$ and $F'_\alpha$ is closed in $N$.

	Indeed, the complement of a closed set $\mathcal{F} \subset M\times N$ is open in the product, so by definition it is a union of products of open sets $\mathcal{U}=\bigcup_{\alpha} U_\alpha \times V_\alpha$ with $U_\alpha$ in $M$ and $V_\alpha$ in $N$. The complement of $\mathcal{U}$ is 
\begin{align*}
	\mathcal{F}&=(M\times N)\setminus \mathcal{U}\\
&=(M\times N)\setminus \left( \bigcup_{\alpha} U_\alpha\times V_\alpha \right) \\
& =\bigcap_\alpha (M\setminus U_\alpha)\times (N\setminus V_\alpha),
\end{align*}
which is a union of products of closed sets as desired.
\end{remark}\fi

\begin{thing4}{Exercise 1.16}\leavevmode
	Prove that a topology on $X$ is Hausdorff if and only if the diagonal $\Delta:=\{(x,y) \in X \times X|x=y\}$ is closed in the product topology.
\end{thing4}

\begin{proof}[Solution]\leavevmode
	$(\implies )$ Suppose that $X$ is Hausdorff. To check that $\Delta$ is closed suppose that $(x,y) \in X \times X$ is a limit point of $\Delta$. We need to show that $(x,y) \in \Delta$, i.e. that $x=y$. If $x \neq y$ we can separate $x$ and  $y$ by disjoint open subsets $U \ni x$ and  $V \ni y$. Then the open set $U \times V$ contains $(x,y)$,  and since $(x,y)$ is a limit point of  $\Delta$ there must be a point $(z,z) \in U \times V$. Then $z \in U$ and $z \in V$, which is a contradiction.

$(\impliedby)$ Suppose $\Delta$ is closed in the product topology and choose two different points $x \neq y$ in $X$. Then $(x,y) \in (X \times X)\setminus \Delta$, which is an open set by hypothesis. Then by definition of product topology there must be two open sets in $X$,  $U \ni x$ and $V \ni y$. Suppose there is a point in the intersection $z \in U \cap V$. Then $(z,z) \in (U \times V) \cap \Delta$, a contradiction.
\end{proof}

\begin{thing3}{Definition 1.11}\leavevmode
	Let $\sim$ be an equivalence relation on a topological space $M$. \textit{\textbf{Factor-topology}} (or \textit{\textbf{quotient topology}}) is a topology on the set $M/\sim$ of equivalence classes such that a subset $U \subset M/\sim$ is open whenever its preimage in $M$ is open.
\end{thing3}

\begin{thing4}{Exercise 1.17}\label{exer:1.17}\leavevmode
	Let $G$ be a finite group acting (continuously) on a Hausdorff topological space $M$.Prove that the quotient map is closed (i.e. puts closed subsets to closed subsets).
\end{thing4}

\begin{proof}[Solution]\leavevmode
The quotient map is $\pi:M\to M/\sim$ where $x \sim y$ if $y=gx$ for some  $g \in G$. To show $\pi$ is closed pick $F \subset M$ closed. We need to show that $\pi(F)$ is closed, so we may show its complement is open. According to the definition of factor topology we want to show that
\[\pi^{-1}\Big((M/\sim) \setminus \pi(F)\Big)=M\setminus\pi^{-1}(\pi(F))\]
is open. Now $\pi^{-1}(\pi(F))$ is the set of points that are $G$-related to points in  $F$, namely $\bigcup_{g \in G} gF$. Since $G$ is finite and  acts by homeomorphisms, this set is a finite union of closed sets, which is closed. {\color{7}Looks like the Hausdorff hypothesis is not necessary.}
\end{proof}

\begin{thing4}{Exercise 1.18*}\leavevmode
	Let $\sim$ be an equivalence relation on a topological space $M$, and $\Gamma \subset M \times M$ its \textit{\textbf{graph}}, that is, the set $\{(x,y) \in M \times M|x \sim y\}$. Suppose that the map $M \longrightarrow M/\sim$ is open, and that $\Gamma$ is closed in $M \times M$. Show that $M/\sim$ is Hausdorff.

	\textbf{Hint} . Prove that diagonal is closed in $M \times M$.
\end{thing4}

\begin{proof}[{\color{2}Solution here should be corrected}]\leavevmode
Notice that any open surjective map is closed: let $f:X \twoheadrightarrow Y$ be an open surjective map and $F\subset X$ closed, then $f(X\setminus F)=f(X)\setminus f(F)=Y\setminus f(F)$.

Our objective is to show that the diagonal $\tilde{\Delta}$ in $(M/\sim)\times(M/\sim)$ is closed. The projection of the graph $\Gamma$ is $\tilde{\Delta}$. Since $\Gamma$ is closed, by the remark above it follows that $\tilde{\Delta}$ is closed in $(M/\sim)\times(M/\sim)$ as we needed.
\end{proof}

\begin{thing4}{Exercise 1.19}\label{exer:1.19}\leavevmode
	Let $G$ be a finit group acting on a Hausdorff topological space $M$. Prove that $M/G$ with the quotient topology is Hausdorff,
	\begin{enumerate}[label=(\alph*)]
	\item (!) when $M$ is compact.
	\item  (*) for abitrary $M$.
	\end{enumerate}
	\textbf{Hint.} Use the previous exercise.
\end{thing4}

\begin{proof}[{\color{2}Sketch of solution}]\leavevmode
To use the previous exercise first notice that the action of $G$ induces an equivalence relation on $M$; this follows from group axioms. Then it's enough to show that the projection is closed and that the graph $\Gamma$ of the equivalence relation is closed in $M\times M$. But by Exercise \hyperref[exer:1.17]{1.17} we already know that the projection is closed, so it's enough to show that $\Gamma$ is closed.

Notice that $\Gamma=\bigcup_{x \in X} (Gx)\times (Gx)$, that is, the union of cartesian products of every orbit with itself. Each of these cartesian products is a finite set because $G$ is finite. If $M$ is compact, then {\color{2}…}
\end{proof}

\begin{thing4}{Exercise 1.20**}\label{exer:1.20}\leavevmode
Let $M=\mathbb{R}$, and $\sim$ an equivalence relation with at most two elements in each equivalence class. Prove that $\mathbb{R}/\sim$ is Hausdorff, or find a counterexample.
\end{thing4}

\begin{proof}[Solution]\leavevmode
By Exercise \hyperref[exer:1.19]{1.19}, if this equivalence relation is induced by a finite-group action, we know the quotient space is Hausdorff. Let's try to show that there always exists a group inducing this equivalence relation. Since every orbit has at most two elements, we can produce a function
\begin{align*}
	g: \mathbb{R} &\longrightarrow \mathbb{R} \\
	x &\longmapsto \begin{cases}
		y\qquad &\exists y\sim x,y \neq x \\
		x\qquad & \text{else} 
	\end{cases} 
\end{align*}
This function satisfies $g^2=\operatorname{id}$. So the group $G=\{\operatorname{id},g\}$ acts on $\mathbb{R}$ producing the equivalence relation we began with. {\color{2}But is $g$ continuous?}
\end{proof}

\begin{thing4}{Exercise 1.21*}[Gluing of closed subsets]\label{exer:1.21}\leavevmode
Let $M$ be a metrizable topological space, and $Z_i \subset M$ a finite number of closed subsets which do not intersect, grouped into pairs of homeomorphic $Z_i \sim Z_i'$. Let $\sim$ be an equivalence relation generated by these homeomorphisms. Show that $M/\sim$ is Hausdorff.
\end{thing4}

\begin{proof}[Solution]\leavevmode
{\color{2}?}
\end{proof}

\subsection{Compact spaces}

\begin{thing3}{Definition 1.12}\leavevmode
	A \textit{\textbf{cover}} of a topological space $M$ is a collection of open subsets $\{U_\alpha \in 2^M\}$ such that $\bigcup U_\alpha=M $. A \textit{\textbf{subcover}} of a cover $\{U_\alpha\}$ is a subset $\{U_\beta\}\subset \{U_\alpha\}$. A topological space is called \textit{\textbf{compact}} if any cover of this space has a finite subcover.
\end{thing3}

\begin{thing4}{Exercise 1.22}[Closed subset of compact is compact]\label{exer:1.22}\leavevmode
	Let $M$ be a compact topological space, and $Z \subset M$ a closed subset. Show that $Z$ is also compact.
\end{thing4}

\begin{proof}[Solution]\leavevmode
Choose a cover $\{U_\alpha\}$ of $Z$. Complete to a cover $\{U_\alpha\}\cup (M\setminus Z)$ of $M$ since $M\setminus Z$ is open by hypothesis. Since $M$ is compact then there is a finite subcover $\{U_\beta\}$ of $M$. This is also a finite subcover of $Z$.
\end{proof}

\begin{thing4}{Exercise 1.23}[Countable metrizable $\implies $ contains convergent subseq. or is discrete]\label{exer:1.23}\leavevmode
	Let $M$ be a countable, metrizable topological space. Show that either $M$ contains a converging sequence of pairwise different elements, or $M$ is discrete.
\end{thing4}

\begin{proof}[Solution]\leavevmode
Suppose $M$ is not discrete. Then there is a point $z_0$ such that $\{z_0\}$ is not an open set. Then every open set containing $z_0$ contains another point. Choose for every $n \in \mathbb{N}$ a point $z_n$ different from $z_0$ inside the a ball $B_{1/n}(z_0)$. Taking a subsequence if necessary, we obtain a sequence of pairwise different elements $\{z_i\}$ converging to $z_0$.

If $M$ is discrete, it's clear that it cannot have a convergent sequence of pairwise disjoint elements: if the limit point $\{z_0\}$ was open,  $M\setminus \{z_0\}$ would be closed and thus it would contain all its limit points!
\end{proof}

\begin{thing3}{Definition 1.13}\leavevmode
	A topological space is called \textit{\textbf{sequentially compact}} if any sequence $\{z_i\}$ of points of $M$ has a converging subsequence.
\end{thing3}

\begin{thing4}{Exercise 1.24}[Metrizable compact $\implies $ sequentially compact]\label{exer:1.24}\leavevmode
	Let $M$ be a metrizable compact topological space. Show that $M$ is sequentially compact.
\end{thing4}

\begin{proof}[Solution]\leavevmode
Let $\{z_i\}$ be a sequence. Since the restricition of a metric to a subset is also a metric, we may use Exercise \hyperref[exer:1.23]{1.23} on the countable metric subspace $\{z_i\}$. Suppose by contradiction that $\{z_i\}$ has no limit point in $M$. In particular it has no limit point in $\{z_i\}$, so by Exercise \hyperref[exer:1.23]{1.23} it is discrete. Then there are neighbourhoods $U_i \ni z_i$ such that $U_i \cap\{z_j\}_{j \neq  i}=\varnothing$. Then $\{U_i\} \cup  (M\setminus \{U_i\})$ is an open cover of $M$, which has a finite subcover. By the pigeon principle, at least one of the $U_i$ contains an infinite number of points in $\{z_i\}$, which is not possible.
\end{proof}

\begin{thing3}{Definition}[Folland, \textit{Real Analysis}, p. 14-15] \leavevmode
	A sequence $\{x_n\}$ in a metric space $(X,\rho)$ is called \textit{\textbf{Cauchy}} if $\rho(x_n,x_m)\to 0$ as $n,m \to \infty$. A subset $E$ of $X$ is called \textit{\textbf{complete}} if every Cauchy sequence in $E$ converges and its limit is in $E$. $E$ is called \textit{\textbf{totally bounded}} if for every $\varepsilon>0$, $E$ can be covered by finitely many balls of radius $\varepsilon$.
\end{thing3}

\begin{thing1}{Theorem 1.6.5}[Burago-Burago-Ivanov, \textit{A course in metric geometry} ]\leavevmode
	Let $X$ be a metric space. Then the following statements are equivalent:
	\begin{enumerate}
	\item $X$ is compact.
	\item Any sequence in $X$ has a converging subsequence.
	\item Any infinite subset of $X$ has an accumulation point.
	\item $X$ is complete and totally bounded.
	\end{enumerate}
	[No proof]
\end{thing1}

\begin{thing1}{Theorem 0.25}[Folland, \textit{Real Analysis}]\leavevmode
	If $E$ is a subset of the metric space $(X,\rho)$, the following are equivalent:
	\begin{enumerate}[label=(\alph*)]
	\item $E$ is complete and totally bounded.
	\item (\textbf{Bolzano-Weierstrass property}) Every sequence in $E$ has a subsequence that converges to a point of $E$.
	\item (\textbf{The Heine-Borel Property}) If $\{V_\alpha\}_{\alpha \in A}$ is a cover of $E$ by open sets, there is a finite set $F \subset A $ such that $\{ V_\alpha\}_{\alpha \in F}$ covers $E$.
	\end{enumerate}
\end{thing1}

\begin{proof}[Plan of proof]\leavevmode
	(a) and (b) are equivalent, and  (a) and (b) together imply (c).
\end{proof}

\begin{thing4}{Exercise 1.25*}\label{exer:1.25}\leavevmode
Construct an example of a Hausdorff topological space which is sequentially compact, but not compact.
\end{thing4}

\begin{thing4}{Exercise 1.26*}\label{exer:1.26}\leavevmode
Construct an example of a Hausdorff topological space which is compact, but not sequentially compact.
\end{thing4}

\begin{thing3}{Definition 1.14}\leavevmode
	A \textit{\textbf{topological group}} is a topological space with group operations $G \times G \longrightarrow G$, $x,t \mapsto xy$ and $G\longrightarrow G$, $x \mapsto  x^{-1}$ which are continuous. In a similar way, one defines \textit{\textbf{topolocial vector spaces}}, \textit{\textbf{topological rings}} and so on.
\end{thing3}

\begin{thing4}{Exercise 1.27*}\label{exer:1.27}\leavevmode
Let $G$ be a compact topological group, acting on a topological space $M$ in such a way that the map $M \times G \longrightarrow M$ is continuous. Prove that the quotient space is Hausdorff.
\end{thing4}

\begin{proof}[Solution]\leavevmode
\end{proof}


\begin{thing4}{Exercise 1.28}[Continuous function maps compact to compact]\label{exer:1.28}\leavevmode
Let $f:X\to Y$ be a continuous map of topological spaces with $X$ compact. Prove that $f(X)$ is also compact.
\end{thing4}

\begin{proof}[Solution]\leavevmode
	Choose an open cover $\{U_\alpha\}$ of $f(X)$. Then  $\left\{f^{-1}\left( U_\alpha \right) \right\}$ is an open cover of $X$ since $f$ is continuous, and thus it has an open subcover $\{f^{-1}(U_\beta)\}$. I claim that $\{U_\beta\}$ is a cover of $f(X)$: if there was a point $f(x) \not \in \bigcup U_\beta $, then $x$ couldn't be in any of the  $f^{-1}(U_\beta)$, which cover $X$. 
\end{proof}

\begin{thing4}{Exercise 1.29}[Compact subset of Hausdorff is closed]\label{exer:1.29}\leavevmode
Let $Z \subset Y$ be a compact subset of a Hausdorff topological space. Prove that it is closed.
\end{thing4}

\begin{proof}[Solution]\leavevmode
Recall that a set is closed if it contains all its limit points (any point that is not a limit point has a neighbourhood not intersecting the set, making the complement open).

Let $z_0$ be a limit point of $Z$. Choose for every point $z \in Z$ neighbourhoods $U_z \ni z$ and $V_z \ni z_0$ such that $U_z \cap V_z=\varnothing$. If $z_0 \not\in Z$, then $\{U_z\}$ is an open cover of $Z$, so there exists a finite subcover $U_{z_1},\ldots,U_{z_n}$. The set $\bigcup_{i=1}^n V_{z_i}$ is an open neighbourhood of $z_0$ that does not intersect $Z$, a contradiction.
\end{proof}

\begin{thing4}{Exercise 1.30}\label{exer:1.30}\leavevmode
Let $f:X \to Y$ be a continuous, bijective map of topological spaces, with $X$ compact and $Y$ Hausdorff. Prove that it is a homeomorphism.
\end{thing4}

\begin{proof}[Solution]\leavevmode
We need to see that $f^{-1}$ is continuous, i.e. that $(f^{-1})^{-1}(U)$ is open for any $U \subset Y$ open. Since $f$ is bijective,  $(f^{-1})^{-1}(U)=f(U)$; so we must check $f$ is open. Equivalently, we can check $f$ is closed:  if $f(F)$ is closed for any closed $F \subset X$, then for any open set $U\subset X$, we see $f(X\setminus U)=Y\setminus f(U)$ is closed.

To see $f$ is closed note that since $X$ is compact and  $f$ is bijective, $f(X)=Y$ is also compact by  Exercise \hyperref[exer:1.28]{1.28}. By Exercise \hyperref[exer:1.22]{1.22} a closed subset $F$ of $X$ is compact. Again by continuity, $f(F)$ is compact in  $Y$. Finally by Exercise \hyperref[exer:1.29]{1.29}, since $Y$ is Hausdorff and $f(F)$ is compact, it must be closed.
\end{proof}

\begin{thing3}{Definition 1.15}\leavevmode
	A topological space $M$ is called \textit{\textbf{pseudocompact}} if any continuous function $f:M\to \mathbb{R}$ is bounded.
\end{thing3}

\begin{thing4}{Exercise 1.31}\label{exer:1.31}\leavevmode
Prove that any compact topological space is pseudocompact.
\end{thing4}

\begin{proof}[Solution]\leavevmode
We must show that any continuous function $f:M \to \mathbb{R}$ is bounded, in the sense that its image contained in a ball of finite radius (c.f. Exercise \hyperref[exer:1.33]{1.33}). The image of any such function is compact by Exercise \hyperref[exer:1.28]{1.28}. But compact sets of $\mathbb{R}$ are bounded: if for every $r>0$, the image  $f(X)$ is not contained in the ball of radius $r$ centered at zero, $B_r(0)$, then $\{B_r(0)\}$ is an open cover of $f(X)$ (since its union is all of  $\mathbb{R}$) without a finite subcover.
\end{proof}

\begin{thing4}{Exercise 1.32}\label{exer:1.32}\leavevmode
Show that for any continuous function $f:M\longrightarrow \mathbb{R}$ on a compact space there exists $x \in M$ such that  $f(z)=\operatorname{sup}_{z \in  M} f(z)$.
\end{thing4}

\begin{proof}[Solution]\leavevmode
As in Exercise \hyperref[exer:1.31]{1.31}, the image of $f$ is a bounded set of $ \mathbb{R}$, which means the supremum is finite. To see it is attained at a point in $M$ notice that $f(M)$ is compact and thus closed. This means that all its limit points belong to  $f(M)$: if a limit point $z_0$ of the closed set $f(M)$ is not in $f(M)$, then $z_0$ has no neighbourhood contained in $\mathbb{R}\setminus f(M)$, but $\mathbb{R}\setminus f(M)$ is open. In particular  $\operatorname{sup}_{z \in M}f(z)$ is in $f(M)$.

(Extra: proof for sequentially compact.) We construct a sequence of points in $f(X)$ whose limit is  $\operatorname{sup}_{z \in M}f(z)$. (Take balls of radius $1/n$.) This gives a sequence in $X$ which must have a convergent subsequence within $M$. The limit of such sequence is the desired point.
\end{proof}

\begin{thing4}{Exercise 1.33}\label{exer:1.33}\leavevmode
Consider $\mathbb{R}^n$ as a metric space, with the standard (Euclidean) metric. Let $Z \subset \mathbb{R}^n$ be a closed, bounded set (\textit{\textbf{bounded}} means contained in a ball of finite radius). Prove that $Z$ is sequentially compact.
\end{thing4}

\begin{proof}[Solution]\leavevmode
	First consider the case $n=1$. If  $\{z_i\}$ is a sequence contained in a closed and bounded set $Z$, then  $\operatorname{sup}_{i \in \mathbb{N}}z_i$ is an element in $Z$. Taking balls of radius $1/m$ with center in the supremum we construct a subsequence \(\{z_{i_j}\}\) of  $\{z_i\}$ converging to the $\operatorname{sup}$.

	Now suppose that $Z \subset\mathbb{R}^n$ is closed and bounded. Note that any projection \(\pi:\mathbb{R}^n\to \mathbb{R}\) of a closed and bounded set \(Z\) is bounded---this follows from the fact that the absolute value of any coordinate is less or equal than the norm of a vector: \(|x_i|\leq \|x\|\). Then projecting gives a sequence from every coordinate, each of which is contained in a bounded set and thus must have a convergent subsequence (though the limit need not be an element of \(\pi(Z)\)).
	%Then \(Z\) is contained in a closed rectangle  \([a_1,b_1]\times\ldots\times[a_n,b_n]\). Choose a sequence $\{z_i\}\subset Z$.
	\iffalse Projecting gives a sequence in every coordinate, {\color{2}each of which is contained in a closed and bounded set in \(\mathbb{R}\)}, so that they must have convergent subsequences.
	%{\color{2}But this doesn't give a convergent subsequence in the product since the subsequences may have completely disjoint indices.}

	\begin{claim}\leavevmode
		Any projection \(\pi:\mathbb{R}^n\to \mathbb{R}\) of a closed and bounded set \(Z\) is closed and bounded.
	\end{claim}
	\begin{proof}[Proof of claim]\leavevmode
	Boundedness is immediate from the fact that the projection of a ball is a ball---this follows from the fact that the absolute value of any coordinate is less or equal than the norm of a vector: \(|x_i|\leq \|x\|\).

	For closedness choose a limit point \(y_0\) of \(\pi(Z)\). This gives a sequence \(\{z_i\} \subset Z\) such that \(\{\pi(z_i)\}\) converges to \(y_0\). By contradiction suppose that there is no \(z_0 \in Z\) such that \(\pi(z_0)=y_0\). Then all \(z \in Z\) are \textit{not} limit points of \(\{z_i\}\): \(\pi\) is continuous, so \(\pi(\lim_{i} z_i)=\lim_{i} \pi(z_i)=y_0\). {\color{2}But that's exactly what this exercise is about!!}
	\end{proof}\fi

\iffalse
First let's show that the projection of \(Z\) to any coordinate is a closed. To check this we may use Exercise \hyperref[exer:1.5]{1.5} (disjoint closed sets have non-intersecting neighbourhoods) as follows. Let \(\pi\) any projection \(\mathbb{R}^n \to \mathbb{R}\). Choose a limit point \(y_0 \in \pi(Z)\). By contradiction suppose that there is no point in \(Z\) which is projected onto \(y_0\), that is, \(\pi^{-1}(y_0) \cap Z=\varnothing\). Choose two disjoint open neighbourhoods \(U \supset \pi^{-1}(y_0)\) and \(V \supset Z\).

Note that \(\pi\) is open: the projection of an open ball in \(\mathbb{R}^n\) is an open interval in \(\mathbb{R}\) (because the absolute value of any coordinate is less or equal than the norm of a vector, \(|x_i|\leq \|x\|\)). Then \(\pi(U)\) is an open neighbourhood of \(y_0\), which must intersect \(\pi(Z)\) since \(y_0\) is a limit point of \(\pi(Z)\), a contradiction. This shows \(\pi\) is closed. Also note that \(\pi\) maps bounded sets to bounded sets.

Now choose a sequence $\{z_i\}\subset Z$. Projecting gives a sequence in every coordinate, each of which is contained in a closed and bounded set in \(\mathbb{R}\), so that they must have convergent subsequences. {\color{2}But this doesn't give a convergent subsequence in the product since the subsequences may have completely disjoint indices.}\fi

We to construct a subsequence of the original sequence we must construct subsequences of every coordinate one by one: since the first coordinate gives a bounded sequence in \(\mathbb{R}\), we have a subsequence \(\{z_{i_{j_1}}\}\) of \(\{z_i\}\) for which the first coordinate converges to some number. Then we look at the second coordinate, which is bounded in \(\mathbb{R}\) and gives a subsequence \(\{z_{i_{j_2}}\}\) of \(\{z_{i_{j_1}}\}\) for which the second \textit{and first} coordinates converge. This way we obtain a subsequence \(\{z_{i_{j_n}}\}\) of \(\{z_i\}\) for which all coordinates converge, so that the subsequence must be convergent itself. Since \(Z\) is closed, the limit point must be in \(Z\).
\end{proof}

\begin{remark}\leavevmode
	A more straightforward proof is given by the so-called "lion's chase"---dividing the closed and bounded set in smaller sets.
\end{remark}

\begin{thing4}{Exercise 1.34**}\label{exer:1.34}\leavevmode
Find a pseudocompact Hausdorff topological space which is not compact.
\end{thing4}

\begin{thing3}{Definition 1.16}\leavevmode
A map of topological spaces is called \textit{\textbf{proper}} if a preimage of any compact subset is compact.
\end{thing3}

\begin{thing4}{Exercise 1.35*}\label{exer:1.35}\leavevmode
Let $f:X\to Y$ be a continuous proper, bijective map of metrizable topological spaces. Prove that $f$ is a homeomorphism, or find a counterexample.
\end{thing4}

\begin{thing4}{Exercise 1.36*}\label{exer:1.36}\leavevmode
Let $f:X \to Y$ be a continuous, proper map of metrizable topological spaces. Show that $f$ is closed, or find a counterexample.
\end{thing4}

\section{Manifolds and sheaves}

\subsection{Topological manifolds}

\begin{thing5}{Remark 2.1}\leavevmode
	Manifolds can be smooth (of a given "class of smoothness"), real analytic, or topological (continuous). \textit{\textbf{Topological manifold}} is easiest to define, it is a topological space which is locally homeomorphic to an open ball in $\mathbb{R}^n$.
\end{thing5}

\begin{thing3}{Definition 2.1}\leavevmode
	\textit{\textbf{An action}} of a group on a manifold is silently assumed to be continuous. Let $G$ be a group acting on a set $M$. The \textit{\textbf{stabilizer}} of $x \in M$ is the subgroup of all elements in $G$ that fix $x$. An action is \textit{\textbf{free}} if the stabilizer of every point os trivial. The \textit{\textbf{quotient space}} $M/G$ is the space of orbits, equipped with the following topology: an open set $U \subset M/G $ is open if its preimage in $M$ is open.
\end{thing3}

\begin{thing4}{Exercise 2.1}[!]\label{exer:2.1}\leavevmode
Let $G$ be a finite group acting freely on a Hausdorff manifold $M$. Show that the quotient space $M/G$ is a topological manifold.
\end{thing4}

\begin{proof}[Solution]\leavevmode

\end{proof}

\begin{thing4}{Exercise 2.2}\label{exer:2.2}\leavevmode
Construct an example of a finite group $G$ acting non-freely on a topological manifold $M$ m such that $M/G$ is not a topological manifold.
\end{thing4}

\begin{proof}[Solution]\leavevmode
Consider the manifold $\mathbb{R}^2$ and a group $G$ generated by some rotation about the origin. This makes the origin the only fixed point of $G$. Any other point has a neighbourhood in which it is the only element in its orbit; such a neighbourhood is mapped diffeomorphically onto the quotient, giving a coordinate chart.

However, any neighbourhood of the origin contains at least two $G$-related elements.  Suppose there is a  coordinate chart of $\bar{0} $ in the quotient. The preimage of this open set is an open set in $\mathbb{R}^2$ containing $0$. The composition of the quotient map and the coordinate chart produce a non-injective map from  $\mathbb{R}^2$ to $\mathbb{R}^2$. If such a map was smooth at $0$, by the inverse function theorem there would be a  neighbourhood of $0$ in which the map would be invertible, a contradiction.
\end{proof}

\begin{thing4}{Exercise 2.3}\label{exer:2.3}\leavevmode
Consider the quotient of  $\mathbb{R}^2$ by the action of $\{\pm  1\}$ that maps $x$ to  $-x$. Is the quotient space a topological manifold?
\end{thing4}

\begin{proof}[Solution]\leavevmode
	Yes (it's not \textit{smooth} but it is topological). A homeomorphism between $\mathbb{R}^2/\{\pm 1\}$ and $\mathbb{R}^2$ is $re^{i\theta}\mapsto re^{i2\theta}$ where the angle $\theta$ in the domain is in the interval  $[0,\pi)$ and  $r\in [0,\infty)$. This map is clearly bijective (any equivalence class has a unique representative of the given form) and its continuous inverse is $re^{i\varphi}\mapsto re^{i\varphi/2}$ for $\varphi \in [0,2\pi)$.
\end{proof}

\begin{thing4}{Exercise 2.4*}\label{exer:2.4}\leavevmode
Let $M$ be path connected, Hausdorff topological manifold and $G$ a group of all its homeomorphisms. Prove that $G$ acts transitively.
\end{thing4}

\begin{thing4}{Exercise 2.5**}\label{exer:2.5}\leavevmode
Prove that any closed subgroup $G \subset \mathsf{GL}(n)$ of a matrix group is homeomorphic to a manifold, or find a counterexample.
\end{thing4}

\begin{thing5}{Remark 2.2}\leavevmode
	In the above definition of a manifold, it is not required to be Hausdorff. Nevertheless, in most cases, manifolds are tacitly assumed to be Hausdorff.
\end{thing5}

\begin{thing4}{Exercise 2.6}\label{exer:2.6}\leavevmode
Construct an example of a non-Hausdorff manifold.
\end{thing4}

\begin{thing4}{Exercise 2.7}\label{exer:2.7}\leavevmode
Show that $\mathbb{R}^2/\mathbb{Z}^2$ is a manifold.
\end{thing4}

\begin{proof}[Solution]\leavevmode
Let $\bar{z}$ be a point in $\mathbb{R}^2/\mathbb{Z}^2$. Its preimage is the lattice $\{z+(a,b):a,b \in \mathbb{Z}\}$. A ball of radius  $\frac{1}{2}$ centered at any representative of $\bar{z} $ contains only one representative of any other class, so that the restriction of the projection is bijective (and continuous by definition of quotient topology). The inverse map is also continuous by definition of quotient topology: an open set in the ball on $\mathbb{R}^2$ is mapped to an open set in the quotient because its preimage is open.
\end{proof}

\begin{thing4}{Exercise 2.8}\label{exer:2.8}\leavevmode
Let $\alpha$ be an irrational number. The group $\mathbb{Z}^2$ acts on $\mathbb{R}$ by the formula $t \mapsto  t +m +n\alpha$. Show that this action is free, but the quotient $\mathbb{R}/\mathbb{Z}^2$ is not a manifold.
\end{thing4}

\begin{proof}[Solution]\leavevmode
This action is obviously free since any nonzero pair of integers "moves" any number $t$. The idea is to show that the orbit of every point is dense. This would mean that any open set in the quotient is homeomorphic to $\mathbb{R}$, preventing it from having local neighbourhoods homeomorphic to balls, i.e. being a topological manifold.
\end{proof}

\subsection{Smooth manifolds}

\begin{thing3}{Definition 2.2}\leavevmode
	A \textit{\textbf{cover}} of a topological space $X$ is a family of open sets $\{U_i\}$ such that $\bigcup_{i} U_i=X$. A cover $\{V_i\}$ is a \textit{\textbf{refinement}} of a cover $\{U_i\}$ if every $V_i$ is contained in some $U_i$.
\end{thing3}

\begin{thing4}{Exercise 2.11}\label{exer:2.11}\leavevmode
Show that any two covers of a topological space admit a common refinement.
\end{thing4}

\begin{proof}[Solution]\leavevmode
Let $\{U_i\}$ and $\{U_i'\}$ be covers of a topological space $X$. Then  $\{V_{ij}:=U_i \cap U_j'\}$ is a common refinement. It is obvious that $V_{ij}$ is contained in $U_i$ and $U_j$, so it is a subcover of both covers. And it is also a cover: it $x \in X$ then $x$ must be in some  $U_i$ and some $U_j'$, so that it is in $V_{ij}$.
\end{proof}

\begin{thing3}{Definition 2.3}\leavevmode
	A cover $\{U_i\}$ is an \textit{\textbf{atlas}} if for every $U_i$ we have a map $\varphi_i:U_i\to \mathbb{R}^n$ giving a homeomorphism of $U_i$ with an open subset in  $\mathbb{R}^n$. The \textit{\textbf{transition maps}} 
	\[\phi_{ij}:\varphi_i(U_i\cap U_j)\to \varphi_j(U_i \cap U_j)\]
	are induced by the above homeomorphisms. An atlas is \textit{\textbf{smooth}} if all transition maps are smooth (of class $C^\infty$, i.e., infinitely differentiable), \textit{\textbf{smooth of class}} $C^i$ if all transition functions are of differentiability class $C^i$ and \textit{\textbf{real analytic}} if all transition maps admita a Taylor expansion at each point.
\end{thing3}

\begin{thing3}{Definition 2.4}\leavevmode
A \textit{\textbf{refinement of an atlas}} is a refinement of the corresponding cover $V_i \subset U_i$ equipped with the maps $\varphi_i:V_i\to \mathbb{R}^n$ that are the restricitions of $\varphi_i:U_i \to \mathbb{R}^n$. Two atlases $(U_i,\varphi_i)$ and $(U_i, \psi_i)$ of class $C^\infty$ of $C^i$ (with the same cover) are \textit{\textbf{equivalent}} in this class if, for all $i$, the map $\psi_i \circ \varphi_i^{-1}$ defined on the corresponding open subset in $\mathbb{R}^n$ belongs to the mantioned class. Two arbitratry atlases are \textit{\textbf{equivalent}} if the corresponding covers possess a common refinement giving equivalent atlases.
\end{thing3}

\begin{thing3}{Definition 2.5}\leavevmode
	A \textit{\textbf{smooth structure}} on a manifold (of class $C^\infty$ of $C^i$) is an atlas of class $C^\infty$ or $C^i$ considered up to the above equivalence. A \textit{\textbf{smooth manifold}} is a topological manifold equipped with a smooth structure.
\end{thing3}

\begin{thing5}{Remark 2.3}\leavevmode
	Terrible, isn't is?
\end{thing5}

\begin{thing4}{Exercise 2.12*}\label{exer:2.12}\leavevmode
Construct an example of two nonequivalent smooth structures on $\mathbb{R}^n$.
\end{thing4}

\begin{thing3}{Definition 2.6}\leavevmode
	A \textit{\textbf{smooth function}} on a manifold $M$ is a function $f$ whose restriction to the chart $(U_i,\varphi_i)$ gives a smooth function $f \circ \varphi_i^{-1}:\varphi_i(U_i)\to\mathbb{R}$ for each open subset $\varphi_i(U_i) \subset \mathbb{R}^n$.
\end{thing3}

\begin{thing5}{Remark 2.4}\leavevmode
	There are several ways to define a smooth manifold. The above way is most standard. It is not the most convenient one but you should know it. Two other ways (via sheaves of functions and via Whitney's theorem) are presented further in these handouts.
\end{thing5}

\begin{thing3}{Definition 2.7}\leavevmode
	A \textit{\textbf{presheaf of functions}} on a topological space $M$ is a collection of subrings $\mathcal{F}(U) \subset C(U)$ in the ring $C(U)$ of all functions on $U$, for each open subset $U \subset M$, such that the restriction of every $\gamma \in \mathcal{F}(U)$ to an open subset $U_1 \subset U$ belongs to $\mathcal{F}(U_1)$.
\end{thing3}

\begin{thing3}{Definition 2.8}\leavevmode
	A presheaf of functions $\mathcal{F}$ is called a \textit{\textbf{sheaf of functions}} if these subrings satisfy the following condition. Let $\{U_i\}$ be a cover of an open subset $U\subset M$ (possibly infinite) and $f_i \in \mathcal{F}(U_i)$ a family of functions defined on the open sets of the cover and compatible on the pairwise intersections:
	\[f_i|_{U_i\cap U_j}=f_j|_{U_i \cap U_j}\]
	for every pair of memebers of the cover. Then there exists $f \in \mathcal{F}(U)$ such that $f_i$ is the restriction of $f$ to $U_i$ for all $i$.
\end{thing3}

\begin{thing5}{Remark 2.5}\leavevmode
	A \textit{presheaf of functions} is a collection of subrings of functions on open subsets, compatible with restrictions. A \textit{sheaf of functions} is a presheaf allowing "gluing" of a function on a bigger open set if its restriction to smaller open sets lies in the presheaf.
\end{thing5}

\begin{thing3}{Definition 2.9}\leavevmode
	A sequence $A_1 \longrightarrow A_2 \longrightarrow A_3 \longrightarrow \ldots$ of homomorphisms of abelian grous or vector spaces is called \textit{\textbf{exact}} if the image of each map is the kernel of the next one.
\end{thing3}

\begin{thing4}{Exercise 2.13}\label{exer:2.13}\leavevmode
Let $ \mathcal{F}$ be a presheaf of functions. Show that $\mathcal{F}$ is a sheaf if and only if for every open cover $\{ U_i\}$ of an oen subset $U\subset M$ the sequence of restriction maps
\[\begin{tikzcd}0\arrow[r]&\mathcal{F}(U)\arrow[r,"\varphi_1"]&\prod_{i}\mathcal{F}(U_i)\arrow[r,"\varphi_2"]&\prod_{i \neq  j}\mathcal{F}(U_i \cap U_j)\end{tikzcd}\]
 is exact, with $\eta \in \mathcal{F}(U_i)$ mapped to $\eta|_{U_i \cap U_j}$ and $-\eta|_{U_j \cap U_i}$.
\end{thing4}

\begin{proof}[Solution]\leavevmode
The key observation is that elements of $\ker \varphi_2$ are collections of functions $f_i \in \mathcal{F}(U_i)$ satisfying compatibility in pairwise intesections, i.e.,
\[f_i|_{U_i \cap U_j}=f_j|_{U_i \cap U_j}.\]
{\color{2}To achieve this} I think we must define $\varphi_2$ by
\[(\ldots,\quad f_i,\quad \ldots,\quad \quad f_j,\quad \ldots)\longmapsto (\ldots,\quad f_i|_{U_i \cap U_j}-f_j|_{U_i \cap U_j},\quad \ldots).\]
Then elements in $\ker \varphi_2$ satisfy the desired compatibility condition.
\iffalse
{\color{2}Indeed:} if $f_i \in \mathcal{F}(U_i)$ and $f_j\in \mathcal{F}(U_j)$ are coordinates of some $F \in \prod_{i}\mathcal{F}(U_i)$, the image of $F$ under  $\varphi_2$ is of the form
\[(\ldots,\;\;\;f_i|_{U_i \cap U_j},\;\;\;\ldots,\;\;\;-f_i|_{U_j \cap U_i},\;\;\;\ldots,\;\;\;f_j|_{U_i \cap U_j},\;\;\;\ldots,\;\;\;-f_j|_{U_j \cap U_i},\;\;\;\ldots)\]
which we assume to be the zero element of $\prod_{i \neq j}\mathcal{F}(U_i \cap U_j)$.\fi

$(\implies )$ Suppose the the sequence above is exact. To show $\mathcal{F}$ is a sheaf fix $f_i \in \mathcal{F}(U_i)$ for every $i$ satisfying $f_i|_{U_i \cap U_j} = f_j |_{U_i \cap U_j}$ for every $i,j$. This is equivalent to choosing an element in $\ker \varphi_2$. By exactness this element is in $\operatorname{img} \varphi_1$. This means that $f_i$ is the restriction of some $f \in \mathcal{F}(U)$ for every $i$ as desired.

$(\impliedby)$ Suppose $\mathcal{F}$ is a sheaf. Injectivity of $\varphi_1$ is immediate: if $f \in \mathcal{F}(U)$ is mapped to zero under $\varphi_1$, meaning $f_i|_{U_i}=0$ for all $i$, it must be zero since $\{U_i\}$ is a cover. Exactness in the second ring is equivalent to the definition of sheaf by the remarks above.
\end{proof} 

\begin{thing4}{Exercise 2.14}\label{exer:2.14}\leavevmode
Show that the following spaces of functions on $\mathbb{R}^n$ define sheaves of functions.
\begin{enumerate}[label=(\alph*)]
\item Space of continuous functions.
\item Space of  smooth functions.
\item Space of functions of differentiability class $C^i$.
\item (*) Space of functions which are pointwise limits of sequences of continuous functions.
\item Space of functions vanishing outside a set of measure 0.
\end{enumerate}
\end{thing4}

\begin{proof}[Solution]\leavevmode
Injectivity of $\varphi_1$ is immediate in all cases: a function that vanishes on every subset of an open cover vanishes identically.
\begin{enumerate}[label=(\alph*)]
\item Define a global function $f$ on $U$ by  $x \mapsto f_i(x)$ for any $f_i \in \mathcal{F}(U_i)$ such that $x \in U_i$. Continuity follows from continuity of $f_i$, and the fact that $f$ is well-defined follows from the gluing condition of $\mathcal{F}$.

\item Like above: smoothness follows from smoothness of $f_i$.

\item Like above.

\item 

\item {\color{2}Not sure} (uncountable union of measure-zero sets may have positive measure).
\end{enumerate}
\end{proof}

\begin{thing4}{Exercise 2.15}\label{exer:2.15}\leavevmode
Show that the following spaces of functions on $\mathbb{R}^n$ are presheaves, but not sheaves
\begin{enumerate}[label=(\alph*)]
\item Space of constant functions.
\item Space of bounded functions.
\item Space of functions vanishing outside of a bounded set.
\item Space of continuous functions with finite  $\int |f|$.
\end{enumerate}
\end{thing4}

\begin{proof}[Solution]\leavevmode
The presheaf condition, that the restriction of a function to 
	\begin{enumerate}[label=(\alph*)]
\item Open sets with two connected components may not glue to a global constant function.

\item Unbounded functions may be bounded in open subsets! Take the open set $(0,\infty)\subset \mathbb{R}$ and the cover $U_i=(1/i,\infty)$. Define the bounded function $f_i(x)=1/x$ in every $U_i$.

	In $\mathbb{R}^n$ we may do the same trick using the half-spaces with bounded last coordinate $U_i=\{(x_1,\ldots,x_n):x_n \geq 1/i\}$ and taking  $f_i(x)=1/\|x\|$.

\item Let the open set $U$ be all of $\mathbb{R}^n$. An open cover is given by balls $B_i$ of radius $2/3$ with center in  $i \in\mathbb{Z}^n$.  For every $i \in \mathbb{Z}^n$ define functions $f_i$ that vanish only outside a ball of very small radius, say $1/6$, with center in $i$. These functions coincide (they vanish) in the intersections of the cover, but the function obtained by gluing cannot vanish outside any bounded set: it is non-zero in the union of balls of radius $1/6$ with centers in  $\mathbb{Z}^n$.

\item Item (c) works if we manage to make the functions continuous. This can be done partition of unity. Also we must require that the values of the integrals in the smaller balls of radius $1/6$ do not tend to zero (this way the global integral is an infinite sum of numbers that do not tend to zero, so it cannot be finite).

	{\color{9}Just consider constant functions!}
\end{enumerate}
\end{proof}

\begin{thing3}{Definition 2.10}\leavevmode
	A \textit{\textbf{ringed space}} $(M,\mathcal{F})$ is a topological space equipped with a sheaf of functions. A \textit{\textbf{morphism}} $(M,\mathcal{F}) \xrightarrow{\psi}(N,\mathcal{F})$ of ringed spaces is a continuous map $M \xrightarrow{\psi}N$ such that, for every open subset $U \subset N$ and every function $f \in \mathcal{F}'(U)$, the function $f \circ \Psi$ belongs to the ring $\mathcal{F}(\Psi^{-1}(U))$. An \textit{\textbf{isomorphism}} of ringed spaces is a homeomorphism $\Psi$ such that $\Psi$ and $\Psi^{-1}$ are morphisms of ringed spaces.
\end{thing3}

\begin{thing5}{Remark 2.6}\leavevmode
	Usually the term ``ringed space" stands for a more general concept, where the ``sheaf of functions" is an abstract ``sheaf of rings", not necesarily a subsheaf in the sheaf of all functions on  $M$. The above definition is simpler, but less standard.
\end{thing5}

\begin{thing4}{Exercise 2.16}\label{exer:2.16}\leavevmode
Let $M, N$ be open subsets in $\mathbb{R}^n$ and let  $\Psi:M \to N$ be a smooth map. Show that $\Psi$ defines a morphism of spaces ringed by smooth functions.
\end{thing4}

\begin{proof}[Solution]\leavevmode
Let $\mathcal{F}$ be the sheaf of smooth functions on $M$ and  $\mathcal{F}'$ on $N$. Choose an open subset $U\subset M$ and $f \in \mathcal{F}'(U)$. Since $\Psi$ is smooth and composition of smooth functions is smooth, $f \circ \Psi$ is a smooth map.
\end{proof}

\begin{thing4}{Exercise 2.17}\label{exer:2.17}\leavevmode
Let $M$ be a smooth manifold of some class and let $\mathcal{F}$ be the space of functions of this class. Show that $\mathcal{F}$ is a sheaf.
\end{thing4}

\begin{proof}[Solution]\leavevmode
Let $U$ be an open set of $M$. To show $\mathcal{F}$ is a presheaf notice that the restriction of a function of class $C^i$ to an open subset is also of class $C^i$. To show $\mathcal{F}$ is a sheaf fix an open set $U \subset M$, an open cover $\{U_j\}$ of $U$, and a collection of functions $f_j \in \mathcal{F}(U_j)$. As in Exercise \hyperref[exer:2.14]{2.14}, differentiability class $C^i$ is a local condition and thus gluing the $f_j$ produces a $C^i$ function on $U$.
\end{proof}

\begin{thing4}{Exercise 2.18}[!]\label{exer:2.18}\leavevmode
Let $M$ be a topological manifold, and let $(U_i,\varphi_i)$ and $(V_j,\psi_j)$ be smooth structures on $M$. Show that these structures are equivalent if and only if the corresponding sheaves of smooth functions coincide.
\end{thing4}

\begin{proof}[Solution]\leavevmode
First let's clarify what is the sheaf of smooth functions associated to a smooth structure. Let $U \subset M$ be open.  The ring $\mathcal{F}(U)$ associated to the atlas $(U_i,\varphi_i)$ consists of functions $f:U \to \mathbb{R}$ such that $f \circ \varphi_i^{-1}$ is smooth for all $i$.

Also recall that equivalence of smooth structures means that there is a common refinement of the covers $\{U_i\}$ and $\{V_j\}$ such that $\psi_k\circ \varphi_k^{-1}$ is smooth for all $k$ indexing the refinement.

$(\implies )$ Suppose that $(U_i,\varphi_i)$ and $(V_j,\psi_j)$ are equivalent. The corresponding sheaves $\mathcal{F}_1$ and $\mathcal{F}_2$ coincide because functions are smooth with respect to one atlas iff they are smooth with respect to the other. Indeed: fix $U \subset M$ open and a function $f \in\mathcal{F}_1(U)$. Then $f \in \mathcal{F}_2(U)$ since
\[f \circ \psi^{-1}_j=f \circ (\varphi_i^{-1}\circ \varphi)\circ\psi_j^{-1}=(f \circ \varphi_i^{-1})\circ (\varphi\circ\psi_j^{-1}).\]
which is smooth.

$(\impliedby)$. Suppose that $\mathcal{F}_1$ and $\mathcal{F}_2$ coincide. Let $W_{ij}:=U_i \cap V_j$ be a common refinement of $\{U_i\}$ and $\{V_j\}$. Set $\varphi_{ij}=\varphi_i|_{W_{ij}}$ and $\psi_{ij}=\psi_j|_{W_{ij}}$. \textbf{We must show that $\psi_{ij}\circ \varphi_{ij}^{-1}$ is smooth.} Idea: to use the fact that the sheaves coincide we can use the coordinate functions of the charts, which are real-valued functions and thus must be elements of the sheaves.

Notice that $\psi_{ij}$ consists of $n:=\dim M$ coordinate functions  $\psi_{ij}^\ell:M \to \mathbb{R}$. Each of this functions is smooth with respect to the smooth structure $(V_j,\psi_j)$ since it is the projection onto the $\ell$-th coordinate, that is,
\[\psi_{ij}^\ell \circ \psi_{ij}^{-1}(x_1,\ldots,x_\ell,\ldots,x_n)=x_\ell.\]
Since the sheaf of smooth functions with respect to the smooth structure $(U_i,\varphi_i)$ is the same, $\psi_{ij}^\ell \circ \varphi_{ij}$ must be smooth for all $\ell$, making $\psi_{ij}\circ \varphi_{ij}^{-1}$ smooth.
\end{proof}

\begin{thing5}{Remark 2.7}\label{rk:2.7}\leavevmode
This exercise implies that the following definition is equivalent to the one stated earlier.
\end{thing5}

\begin{thing3}{Definition 2.11}\label{def:2.11}\leavevmode
	Let $(M,\mathcal{F})$ be a topological manifold equipped with a sheaf of functions. It is said to be a \textit{\textbf{smooth manifold of class}} $C^\infty$ or  $C^i$ if every point in $(M,\mathcal{F})$ has an open neighbourhood isomorphic to the ringed space $(\mathbb{R}^n, \mathcal{F}')$, wherre $\mathcal{F}'$ is a ring of functions on $\mathbb{R}^n$ of this class.
\end{thing3}

\begin{thing3}{Definition 2.12}\leavevmode
A \textit{\textbf{coordinate system}} on an open subset $U$ of a manifold $(M,\mathcal{F})$ is an isomorphism between $(U,\mathcal{F})$ and an open subset in $(\mathbb{R}^n,\mathcal{F}')$, where $\mathcal{F}'$ are functions of the same class on $\mathbb{R}^n$.
\end{thing3}

\begin{thing5}{Remark 2.8}\leavevmode
In order to avoid complicated notation, from now on we assume that all manifolds are Hausdorff and smooth (of class $C^\infty)$. The case of other differentiability classes can be considered in the same manner.
\end{thing5}

\begin{thing4}{Exercise 2.19}[!]\label{exer:2.19}\leavevmode
Let $(M,\mathcal{F})$ and $(N,\mathcal{F}')$ be manifolds and let $\Psi:M \to N$ be a continuous map. Show that the following conditions are equivalent.
\begin{enumerate}[label=(\roman*)]
\item In local coordinates $\Psi$ is given by a smooth map
\item $\Psi$ is a morphism of ringes spaces.
\end{enumerate}
\end{thing4}

\begin{proof}[Solution]\leavevmode
(i)$\implies $(ii). Suppose that in local coordinates $\Psi$ is given by a smooth map. Showing that $\Psi$ is a morphism of ringed is spaces is to show that for any open set $U \subset N$ and smooth function $f \in\mathcal{F}'(U)$, the function $f \circ \Psi$ is smooth on $\Psi^{-1}(U)$. The latter means that for each chart $(U_i,\varphi_i)$ of $\Psi^{-1}(U)$, the composition  $(f \circ\Psi)\circ \varphi_i^{-1}$ is smooth.

\[\begin{tikzcd}[column sep=large,row sep=large]
	\mathbb{R}&M\arrow[r,"\Psi"]\arrow[d,swap,"\varphi"]\arrow[l,swap,"f \circ \Psi",bend right]& N\arrow[d,"\psi"]\arrow[r,"f",bend left]&\mathbb{R}\\
	&\mathbb{R}^m\arrow[r,"\psi\circ\Psi\circ \varphi^{-1}",swap]&\mathbb{R}^n
\end{tikzcd}\]
The definition of $f$ being smooth in $U$ is that  $f \circ \psi^{-1}_j$ is smooth in any chart $(V_j,\psi_j)$. Starting from $\mathbb{R}^m$, we can go right instead of up to see that
\[(f \circ \Psi)\circ \varphi^{-1}=(f \circ \psi^{-1}) \circ (\psi \circ \Psi \circ\varphi^{-1}),\]
which is smooth.

{\color{2}Misha: this is almost trivial, should be easier.}

(ii)$\implies $ (i). Now suppose that the pullback of smooth functions (defined on open sets) by $\Psi$ is smooth. Choose the coordinate functions $\psi^\ell$ of a local chart $\psi$. Then $\psi^\ell \circ \Psi \circ \varphi^{-1}$ is smooth for all $\ell$ and for any local chart $(U,\varphi)$ of $M$, making $\psi \circ \Psi \circ \varphi^{-1}$ smooth as well.

{\color{2}Misha: and this shouldn't be that easy.}
\end{proof}

\begin{thing5}{Remark 2.9}\label{rk:2.9}\leavevmode
An isomorphism of smooth manifolds is called a \textit{\textbf{diffeomorphism}}. As follows from this exercise, a diffeomorphism is a homeomorphism that maps smooth functions onto smooth ones. {\color{14}Because the inverse map pulls back smooth functions to smooth ones, so the map itself maps smooth functions to smooth ones.}
\end{thing5}

\subsection{Embedded manifolds}

\begin{thing3}{Definition 2.13}\leavevmode
A \textit{\textbf{closed embedding}} $\phi:N \hookrightarrow M$ of topological spaces is an injective map from $N$ to a closed subset $\phi(N)$ inducing a homeomorphism of $N$ and $\phi(N)$. An \textit{\textbf{open embedding}} $\phi:N \hookrightarrow M$ is a homeomorphism of $N$ and an open subset of $M$. is an image of a closed embedding.
\end{thing3}

\begin{thing3}{Definition 2.14}\leavevmode
Let $M$ be a smooth manifold. $N \subset M$ is called \textit{\textbf{smoothly embedded submanifold of dimension $m$}} if for every point $x \in N$ there is a neighbourhood $U \subset M$ diffeomorphic to an open ball $B \subset \mathbb{R}^n$, such that this diffeomorphism maps $ U \cap N$ onto a linear subspace of $B$ dimension $m$.
\end{thing3}

\begin{thing4}{Exercise 2.22}\label{exer:2.22}\leavevmode
Let $(M,\mathcal{F})$ be a smooth manifold and let $N \subset M$ be a smoothly embedded submanifold. Consider the space $\mathcal{F}'(U)$ of smooth functions on $U \subset N$ that are extendable to functions on $M$ defined on some neighbourhood of $U$.
\begin{enumerate}[label=(\alph*)]
\item Show that $\mathcal{F}'$ is a sheaf.
\item  Show that this sheaf defines a smooth structure on $N$.
\item Show that the natual embedding $(N, \mathcal{F}') \to (M, \mathcal{F})$ is a morphism of manifolds.
\end{enumerate}
\textbf{Hint.} To prove that $\mathcal{F}$ is a sheaf, you might need partition of unity introduced below. Sorry.
\end{thing4}
\begin{proof}[Solution]\leavevmode
\begin{enumerate}[label=(\alph*)]
\item To see that $\mathcal{F}'$ is a presheaf fix an open set $U\subset N$ and a function $f \in \mathcal{F}'(U)$. This means that $f$ can be extended to a function $\tilde{f}$ on $M$ defined on some neighbourhood of $U$. Then the restriction of $f $ to any open subset  $U_1 \subset U$ can be extended to the same function \(\tilde{f}\) on $M$ defined on the same neighbourhood of $U$, which is also a neighbourhood of $U_1$. This says that $f|_{U_1}\in\mathcal{F}'(U_1)$.

	To check that $\mathcal{F}'$ is a sheaf consider a cover $ \{ U_i\}$ of $U$ and chose $f_i \in \mathcal{F}(U_i)$ for all $i$ satisfying
	\[f_i|_{U_i\cap U_j}=f_j|_{U_i \cap U_j},\qquad \forall i,j.\]
	This means that every $f_i$ can be extended to a function $\tilde{f}_i$ on $M$ defined on some neighbourhood $\tilde{U}_i \subset M$ of $U_i$. Consider $\tilde{U}=\bigcup_{i} \tilde{U}_i$; we must construct a smooth function on all of \(\tilde{U}\) from the \(\tilde{f}_i\).
	
	The natural choice is to try to define a function $\tilde{f}:\tilde{U}\to \mathbb{R}$ given by $x\mapsto \tilde{f}_i(x)$ for any $i$ such that $x \in \tilde{U}_i$. This may not work since the \(\tilde{f}_i\) may not coincide in the intersections \(\tilde{U}_i\cap \tilde{U}_j\) outside $N$.

	{\color{2}Suppose there is a partition of unity} \(\{\nu_i\}\) subordinate to the cover \(\{\tilde{U}_i\}\). Then each \(\tilde{f}_i\nu_i\) is a smooth function defined on \(\tilde{U}\), and so is the function \(F=\sum_i \tilde{f}_i\nu_i\).

	To conclude we must show that the restriction of \(F\) to any \(U_j\) coincides with \(f_j\). Let \(x \in U_j\) for some $j$. Then
	\begin{align*}
	F(x)&=\sum_i\tilde{f}_i(x)\tilde{\nu}_i(x)=\sum_if_i(x)\tilde{\nu}_i(x)=f_j(x)\sum_i\tilde{\nu}_i(x)=f_j(x)
	\end{align*}
since the original functions \(f_i\) coincide in the intersections.
	\item Suppose that $N$ is a smoothly embedded sumbanifold of dimension $m$.

According to Remark \hyperref[rk:2.7]{2.7} and Definition \hyperref[def:2.11]{2.11} we must show that every point in \((N,\mathcal{F}')\) has an open neighbourhood isomorphic to the ringed space \((\mathbb{R}^m,\mathcal{F}'')\), where \(\mathcal{F}''\) is a sheaf of smooth functions on \(\mathbb{R}^m\).
		%According to Remark \hyperref[rk:2.9]{2.9}, this means that there are local homeomorphisms between $N$ and \(\mathbb{R}^m\) that map smooth functions to smooth ones.

	Since $N$ is a smoothly embedded submanifold, at every point of $N$ there is a neighbourhood $U$ of $M$ homeomorphic to a ball $B$ in \(\mathbb{R}^n\) such that \(U \cap N\) is mapped to a linear subspace of \(B\). Since \(M\) is a smooth manifold we may suppose (restricting to a smaller open set if necessary) that the same \((U,\mathcal{F})\) is isomorphic to  \((\mathbb{R}^n,\mathcal{F}'')\).

	Let's check that \((U\cap N, \mathcal{F}')\) is isomorphic to \((\mathbb{R}^n,\mathcal{F}'')\). Suppose that \(U\) is isomorphic to  \(\mathbb{R}^m\) via \(\varphi\). It's clear that \(U \cap N\) is homeomorphic to an open subset \(V\) of \(\mathbb{R}^n\) via \(\varphi|_{U \cap N}\).

	Let \(V:=\varphi(U \cap N)\) and \(f'' \in \mathcal{F}''(V)\). Then $f''$ may be smoothly extended to a function on \(\varphi(U)\cong \mathbb{R}^m\): define an extension \(\tilde{f}''(x,y)=f''(x)\); then the partial derivatives with respect to the new variables vanish. Then \(\tilde{f}''\) corresponds to a smooth function on \(U\) by the isomorphism  \((U,\mathcal{F})\cong (\mathbb{R}^m,\mathcal{F}'')\). This shows that the function $f''$ corresponds to a function on \(U \cap N\) that may be extended to a neighbourhood of  \(M\), meaning that it is an element of \(\mathcal{F}'(U \cap N)\).

	Conversely, a function \(f' \in \mathcal{F}'(U\cap N)\) may be smoothly extended to a function on some open set of \(M\) by definition. Intersecting such a set with \(U\) and restricting smooth functions we may suppose it is isomorphic to \((\mathbb{R}^n, \mathcal{F}'')\). Then \(\varphi\) maps the extension of \(f'\) to a smooth function on \(\mathbb{R}^n\), whose restriction to \(V\) is an element of \(\mathcal{F}''(V)\).
	%\(f' \circ \varphi^{-1}|_{V}\) is a function on \(\mathcal{F}''(V)\).

\item To check that the natural embedding \((N,\mathcal{F}') \xrightarrow{\Psi} (M, \mathcal{F})\) is a morphism of manifolds we must check that it is a continuous map satisfying \(f \circ\Psi \in \mathcal{F}'(\Psi^{-1}(U))\) for any open set \(U \subset M\) and \(f \in \mathcal{F}(U)\).

	Continuity is immediate since {\color{2}$N$ is equipped with the subspace topology}. The second condition is also immediate by definition of \(\mathcal{F}'\).
\end{enumerate}
\end{proof}

\begin{thing4}{Exercise 2.23}\label{exer:2.23}\leavevmode
Let \(N_1,N_2\) be two manifolds and let \(\varphi_i:N_i\to M\) be smooth embeddings. Suppose that the image of \(N_1\) coincides with that of \(N_2\). Show that \(N_1\) and \(N_2\) are isomorphic.
\end{thing4}

\begin{proof}[Solution]\leavevmode
%First notice that the condition that the image of \(N_1\) coincides with that of \(N_2\) makes them have the same dimension $m$. Indeed: there are local homeomorphisms \(\varphi_i\) mapping \(N_i \cap U\) to linear subspaces of dimensions \(m_i\) of \(\mathbb{R}^n\), and their composition gives a diffeomorphism between such subspaces. 
According to Remark \hyperref[rk:2.9]{2.9}, to see that \(N_1\cong N_2\) we must show that there are local homeomorphisms between \(N_2\) and \(N_2\) that map smooth functions to smooth ones.

The smooth structures of \(N_i\) are given by Exercise \hyperref[exer:2.22]{2.22}: they are the sheaves \(\mathcal{F}'_i\), that is, \((N_i,\mathcal{F}'_i)\) are locally isomorphic to \((\mathbb{R}^{n_i},\mathcal{F}'')\) for some \(n_i\). Notice that \(n_1=n_2\) because the image of these embeddings in \(M\) coincides: that means that there is a neighbourhood \(U \subset M\) diffeomorphic to a ball \(B\) in \(\mathbb{R}^m\) such that \(N_i \cap U\) is mapped to the same linear subspace of \(B\) of dimension \(n_1=n_2:=n\).

Local homeomorphisms between \(N_1\) and \(N_2\) may be obtained by composing the local homeomorphisms with \(\mathbb{R}^{n}\) given by each smooth structure:

\[\begin{tikzcd}
	V_1 \subset N_1\arrow[r,dashed]\arrow[d]&N_2\supset V_2\arrow[d]\\
	\mathbb{R}^n\arrow[r]&\mathbb{R}^n
\end{tikzcd}\]


The fact that these local homeomorphisms map smooth functions to smooth functions follows from the fact that they are compositions of diffeomorphisms and that the dimensions coincide. (Since the dimensions coincide we can suppose that the map \(\mathbb{R}^{n}\to \mathbb{R}^{n}\) is linear and thus smooth.)

{\color{2}This should also be done better.}
\end{proof}
\begin{thing5}{Remark 2.10}\leavevmode
By the above problem, in order to define a smooth structure on $N$, it sufficies to embed $N$ into \(\mathbb{R}^n\). As it will be clear in the next handout, every manifold is embeddable into \(\mathbb{R}^n\) (assuming it admits partition of unity). Therefore, in place of a smooth manifold, we can use ``manifolds that are smoothly embedded into \(\mathbb{R}^n\)".
\end{thing5}


\subsection{Partition of unity}

\begin{thing4}{Exercise 2.26}\label{exer:2.26}\leavevmode
Show that an open ball \(\mathbb{B}^n \subset \mathbb{R}^n\) is diffeomorphic to \(\mathbb{R}^n\).
\end{thing4}

\begin{thing4}{Definition 2.15}\label{def:2.15}\leavevmode
A cover \(\{U_\alpha\}\) of a topological space \(M\) is called \textit{\textbf{locally finite}} if every point in \(M\) possesses a neighbourhood that intersects only a finite number of \(U_\alpha\).
\end{thing4}

\begin{thing4}{Exercise 2.27}\label{exer:2.27}\leavevmode
Let \(\{U_\alpha\}\) be a locally finite atlas on \(M\), and \(U_\alpha\xrightarrow{\phi_\alpha}\mathbb{R}^n\) homeomorphisms. Consider a cover \(\{V_i\}\) of \( \mathbb{R}^n\) given by open balls of radius $n$ centered in integer points, and let \(\{ W_\beta\}\) be a cover of \(M\) obtained as union of \(\phi^{-1}_\alpha(V_i)\). Show that \(\{W_\beta\}\) is locally finite.
\end{thing4}

\begin{proof}[Solution]\leavevmode
 The result follows from the local finiteness of both \(\{U_\alpha\}\) in \(M\) and \(\{V_i\}\) in \(\mathbb{R}^n\) as follows. (Local finiteness of \(\{V_i\}\) follows from definition of \(\{V_i\}\).)

%{\color{2}I suppose that} \(W_\beta=\phi^{-1}_\alpha(V_i)\) for fixed \(\alpha\) and \(i\). Then \(W_\beta\) is contained in \(U_\alpha\).
Since \(\{U_\alpha\}\) is locally finite, for a given point $x$ of \(M\) there is a neighbourhood which intersects only a finite number of the \(U_\alpha\). Moreover, since  \(\{V_i\}\) is locally finite, each \(\phi_\alpha(x)\) has a neighbourhood intersecting only finitely many \(V_i\). In conclusion, for each \(\alpha\) there are finitely many \(W_\beta\) determined by the \(V_i\), and there are finitely many \(U_\alpha\) which intersect at a given point.
%The same goes if \(W_\beta=\bigcup_{i} \phi_\alpha(V_i)\) for fixed \(\alpha\). If \(W_\beta = \bigcup_{\alpha}\phi^{-1}_\alpha(V_i) \) for fixed $i$ we cannot be certain since \(W_\beta\) may be contained in an uncountable union of the \(U_\alpha\).
\end{proof}

\begin{thing4}{Exercise 2.28}\label{exer:2.28}\leavevmode
Let \(\{U_\alpha\}\) be an atlas on a manifold \(M\).
\begin{enumerate}[label=(\alph*)]
\item Construct a refinement \(\{W_\beta\}\) of \(\{U_\alpha\}\) such that a closure of each \(W_\beta\) is compact in \(M\).
\item Prove that such a refinement can be chosen locally finite if \(\{U_\alpha\}\) is locally finite.
\end{enumerate}
\textbf{Hint.} Use the previos exercise.
\end{thing4}

\begin{proof}[Solution]\leavevmode
\begin{enumerate}[label=(\alph*)]
\item The refinement is the cover \(\{W_\beta\}\) from Exercise \hyperref[exer:2.27]{2.27}. 
The closure of \(W_\beta=\phi^{-1}_\alpha(V_i)\) is mapped by \(\phi_\alpha\) to the closure of its image, \(\phi_\alpha(U_\alpha)\cap V_i\). (This is because  \(\phi_\alpha\) is a homeomorphism; by Exercise \hyperref[exer:1.6]{1.6} limit points of the domain map to limit points of the image.) The closure of \(\phi_\alpha(U_\alpha)\cap V_i\) is compact (since it is closed and bounded), and thus its image under \(\phi^{-1}\) is also compact.

\item This is immediate from Exercise \hyperref[exer:2.27]{2.27}.
\end{enumerate}
\end{proof}

\begin{thing4}{Exercise 2.29}\label{exer:2.29}\leavevmode
Let \(K_1, K_2\) be non-intersecting compact subsets of a Hausdorff topological space \(M\). Show that there exist a pair of open subsets \(U_1 \supset K_1\), \(U_2 \supset K_2\) satisfying \(U_1 \cap U_2=\varnothing\).
\end{thing4}

\begin{proof}[Solution]\leavevmode
	(With some help from {\color{6}ChatGPT}). Fix a point \(y \in K_2\). Since \(M\) is Hausdorff, for every \(x \in K_1\) there are disjoint neighbourhoods \(U_{xy} \ni x\) and \(V_{xy} \ni y\). This means that \( \{U_{xy}\}_{x \in X}\) is an open cover of \(K_1\), which must have a finite subcover \(U_{x_1y},\ldots,U_{x_{n_y}y}\). These open sets correspond to open sets \(V_{x_1y},\ldots,V_{x_{n_y}y}\), {\color{6}the intersection of which is a neighbourhood of $y$ disjoint from \(\bigcup_{i=1}^{n_{y}}U_{x_iy}\).}

Denote this intersection by \(V_y:=\bigcap_{i=1}^{n_{y}}V_{x_iy}\). Then \(\{V_y\}_{y \in Y}\) is an open cover of \(Y\), which must have a finite subcover \(V_{y_1},\ldots,V_{y_m}\). Each \(V_{y_j}\) is associated to an open cover of \(K_1\), from which it is disjoint. The intersection of (the unions of) these \(m\) covers of \(K_1\) is an open set containing \(K_1\), and it is disjoint from \(\bigcup_{j=1}^mV_{y_j} \supset K_2\).
\end{proof}
\iffalse
\begin{proof}[Solution]\leavevmode
Since \(M\) is Hausdorff, for every pair of points \(x \in K_1\) and \(y \in K_2\) there are open sets \(U \ni x\) and \(V \ni y\) such that \(U \cap V= \varnothing\). In other words, for every point \((x,y) \in K_1 \times K_2\) there is an open neighbourhood \(U\times V\) that does not intersect the diagonal \(\Delta \subset M \times M\). Since \(K_1 \times K_2\) is compact, there is a finite cover \(W= \bigcup_{i} U_i \times V_i\) of \(K_1 \times K_2\) that does not intersect the diagonal.

This is not enough for finding two disjoint neighbourhoods of \(K_i\): we must ensure that \(U_i \cap V_j=\varnothing\) for \(i \neq  j\). This is the same as showing that \(\left(\bigcup_{i} U_i \times V_i\right)\cup \left( \bigcup_{i} V_i \times U_i \right) \) does not intersect the diagonal.

This means that there are pairs \((x_1,y_1,),\ldots,(x_k,y_k)\) and open sets (disjoint from the diagonal) \(U_i\times V_i \ni (x_i,y_i)\) so that \(\mathcal{U}:=\bigcup_{i}^k U_i \supset K_1\) and  \(\mathcal{V}:=\bigcup_{i}^k V_i\supset K_2\) are neighbourhoods of \(K_1\) and \(K_2\).

Note that \(\mathcal{U} \cap \mathcal{V}\) may not be empty: it could happen that \(U_i \cap V_j \neq  \varnothing\) for \(i \neq  j\). In other words, \(\mathcal{U} \times \mathcal{V}\) may intersect \(\Delta\). However, since \(\Delta\) is closed, the set \((\mathcal{U}\times \mathcal{V})\setminus\Delta\) remains open in \(M \times M\) (it is the intersection of two open subsets).

Is the projection open? If so, then \(\pi_i(\mathcal{U} \times \mathcal{V} \setminus \Delta)\) would be the solution. But is it? Maybe finally using compactness… the projection of each \(K_i\) is compact, but we don't know whether the projection of \(\Delta\) is closed. So \(K_1\) is compact in \(\mathcal{U}\). Which means that \(\mathcal{U}\setminus K_1\) is open. But what about \(\mathcal{U} \setminus\pi(\Delta)\)? {\color{6}the argument here is that simply removing the points that ``appear twice" does not stop the sets from being open.}


This is because the cartesian product of two unions contains more elements than the union of cartesian products. We may consider instead \(\pi_1(W):= \mathcal{U}\) and \(\pi_2(W):= \mathcal{V}\) where \(\pi_i\) are the projections.

I know that \(U_i \times V_i\) and \(U_j \times V_j\) do not intersect \(\Delta\). But \(U_i \times V_j\) is not

Our objective is to show that there is an open neighbourhood \(W= \bigcup_{i} U_i \times V_i\) of \(K_1 \times K_2\) that does not intersect the diagonal. Suppose there is no such neighbourhood. Then every neighbourhood \(W\) of \(K_1\times K_2\) contains a point in \(\Delta\).

Since the diagonal is closed and \(K_1 \times K_2\) is compact.

If we show that \(K_1 \times K_2\) is compact, this would yield a finite cover open cover




	Suppose that for every open subsets \(U_1 \supset K_1\) and \(U_2 \supset K_2\) there is a point \(z \in U_1 \cap U_2\). We can separate \(z\) from \(K_1\) by considering \(\bigcup_{i} U\)

What about ``sequence converging to set" or ``limit set of a set"? It means that every  neighbourhood of this set \(L\) contains points of \(K_1\) other than those of \(L_1\). If I show: \textbf{claim}: if \(K_1\) is compact and \(L\) is a limit set of \(K_1\) then \(L \subset K_1\), am I done?
\end{proof}
\fi

\begin{thing4}{Exercise 2.30}[!]\label{exer:2.30}\leavevmode
Let \(U \subset M\) be an open subset with compact closure, and \(V \supset M\setminus U\) another open subset. Prove that there exists \(U' \subset U\) such that the closure of \(U'\) is contained in \(U\), and \(V \cup  U'=M\).

\textbf{Hint.} Use the previous exercise.
\end{thing4}

\begin{proof}[Solution]\leavevmode
	(Without using the previous exercise.) Notice that the sets \(U\) and  \(V\) are a cover of \(M\).
\end{proof}

\begin{proof}[Solution]\leavevmode
We can split \(U\) into two parts: \(W:=U\cap V\) and \(W':=U\setminus V\). Since the closure of \(U\) is compact, so are the closures of \(W\) and \(W'\), but the may not be disjoint, so we cannot apply Exercise \hyperref[exer:2.29]{2.29}. We must produce a smaller set

What about \(W'=U \setminus V\) and any point in \(U\) outside the close of \(W'\)? If the closure of \(W'\) contained all of \(U\),  

By Exercise \hyperref[exer:2.29]{2.29} there are two open sets \(W_1 \supset W\) and \(W_1' \supset W'\) 

What about the complement of the closure of \(U\)? It is an open subset of \(M\).

\end{proof}

\begin{thing3}{Definition 2.16}\label{def:2.16}\leavevmode
Let \(U \subset V\) be two open subsets of \(M\) such that the closure of \(U\) is contained in \(V\). In this case we write \(U \Subset V\).
\end{thing3}

\begin{thing4}{Exercise 2.31}[!]\label{exer:2.31}\leavevmode
Let \(\{U_\alpha\}\) be a countable locally finite cover of a Hausdorff topological space, such that a closure of each \(U_\alpha\) is compact. Prove that there exists another cover \(\{V_\alpha\}\) indexed by the same set, such that \(V_\alpha \Subset U_\alpha\).

\textbf{Hint.} Use induction and the previous exercise.
\end{thing4}

\begin{proof}[Solution]\leavevmode
In order to use Exercise \hyperref[exer:2.30]{2.30} consider for every \(\alpha\) the set \(W_\alpha=\bigcup_{\beta \neq  \alpha} U_\beta\). Then \(W_\alpha \supset M\setminus U_\alpha\), so that there exists \(U'_\alpha \Subset U_\alpha\) and \(W_\alpha \cup U'_\alpha=M\). It remains to show that \(\{U'_\alpha\}\) is a cover. Let \(x \in M\) be any point. Then for each \(\alpha\) either \(x \in U'_\alpha\), in which case we are done, or \(x \in W_\alpha\)
\end{proof}

\begin{thing3}{Definition 2.17}\label{def:2.17}\leavevmode
A \textit{\textbf{function with compact support}} is a function which vanishes outside of a compact set.
\end{thing3}

\begin{thing3}{Definition 2.18}\label{def:2.18}\leavevmode
	Let \(M\) be a smooth manifold and let \(\{U_\alpha\}\) be a locally finite cover of \(M\). A \textit{\textbf{partition of unity}} subordinate to the cover \(\{U_\alpha\}\) is a familty of smooth functions \(f_i:M\to [0,1]\) with compact support indexed by the same indices as the \(U_i\)'s and satisfying the following conditions.
	\begin{enumerate}[label=(\alph*)]
	\item Every function \(f_i\) vanishes outside \(U_i\).
	\item \(\sum_i f_i=1\).
	\end{enumerate}
\end{thing3}

\begin{thing5}{Remark 2.11}\label{rk:2.11}\leavevmode
Note that the sum \(\sum_if_i=1\) makes sense only when \(\{U_\alpha\}\) is locally finite.
\end{thing5}

\begin{thing4}{Exercise 2.34}\label{exer:2.34}\leavevmode
Show that all derivatives of \(e^{-\frac{1}{x^2}}\) at 0 vanish.
\end{thing4}

\begin{proof}[Solution]\leavevmode
	First notice that the function \(e^{-x^{-2}}\) is not defined at 0. However, the limit as \(x \to 0\) is zero, so that defining the function to be 0 at \(x=0\) preserves continuity. The same will happen with its derivatives.

	The first derivative is
	\[\frac{d}{dx}e^{-x^{-2}}=2x^{-3}e^{-x^{-2}}.\]
Since exponential decay is faster than polynomial decay, the limit as \(x \to 0\) is zero.

	The second derivative is
\begin{align*}
	\frac{d^2}{dx^2}e^{-x^{-2}}&=2\Big(x^{-3}{\color{6}\frac{d}{dx}e^{-x^{-2}}}-3x^{-4}e^{-x^{-2}}\Big)\\
&=2\Big(x^{-3}{\color{6}2x^{-3}e^{-x^{-2}}}-3x^{-4}e^{-x^{-2}}\Big)\\
&=P_2(x)e^{-x^{-2}}
\end{align*}
where \(P_2(x)\) is some polynomial, so that again the limit as \(x \to 0\) is zero. Proceeding by induction suppose that the \(n\)-th derivative is the product of some polynomial \(P_n(x)\) times \(e^{-x^{-2}}\). Then the \((n+1)\)-th derivative is
\begin{align*}
	\frac{d^{n+1}}{dx^{n+1}}e^{-x^{-2}}&=\frac{d}{dx}\left(\frac{d^n}{dx^n}e^{-x^{-2}}\right)\\
&=\frac{d}{dx}\left(P_n(x)e^{-x^{-2}}\right)\\
&=P'_n(x)e^{-x^{-2}}+P_n(x)\frac{d}{dx}e^{-x^{-2}}\\
&=P'_n(x)e^{-x^{-2}}+P_n(x)P_1(x)e^{-x^{-2}}.
\end{align*}
\end{proof}

\begin{thing4}{Exercise 2.35}\label{exer:2.35}\leavevmode
Define the following function \(\lambda\) on \(\mathbb{R}^n\) 
\[\lambda(x)=\begin{cases}
	e^{\frac{1}{|x|^2-1}}{\color{2}-1}\qquad &\text{ if } |x|<1 \\
	0\qquad &\text{ if } |x|\geq 1
\end{cases}\]
Show that \(\lambda\) is smooth and that all its derivatives vanish at the points of the unit sphere.
\end{thing4}

\begin{proof}[Solution]\leavevmode
\(\lambda\) is smooth in the unit open ball since it is a composition of smooth functions: real exponent \(e^x\), \(1/x\) for nonzero  $x$ and \(|x|^2-1\). Outside the unit closed ball it is trivially smooth since any point has a neighbourhood in which \(\lambda\) is constant 0. To show that \(\lambda\) is smooth in the unit sphere we must check 
\end{proof}
\end{document}
